\documentclass{swfucoursework}

\swfusetup{%
  Title     ={Linux应用}, % 课程名称
  School    ={大数据与智能工程学院},%学院
  Author    ={艾学习}, % 姓名
  ID        ={20251152xxx}, % 学号
  Class     ={计算机科学与技术2025班},%班级
  Tutor     ={王晓林}, %指导教师
  Term      ={2025 -- 2026 -- 01},%学期
  CommentsDate ={2025年11月1日},%
  Comments  ={Good!},%
  Mark      ={B},%
  Sig       ={wangxiaolin},% default wangxiaolin
}

\begin{document}

\maketitle

\tableofcontents
\clearpage

\section{实习目的}\label{ux5b9eux4e60ux76eeux7684}

熟练掌握Linux平台的使用,掌握基本命令及基本的shell编程,
了解Linux平台上的常用软件开发环境及开发步骤。

\section{实验要求}\label{ux5b9eux9a8cux8981ux6c42}

\begin{enumerate}
\def\labelenumi{\arabic{enumi}.}
\tightlist
\item
  在Linux平台完成所有实验\\
\item
  在Linux平台完成实验报告\\
\item
  努力尝试用英文撰写实验报告\\
\item
  将实验作业及报告以tgz格式打包,并上传到指定教学网站\\
\item
  迟交报告将被扣分
\end{enumerate}

\section{实验主要内容(含工具、方法等)}\label{ux5b9eux9a8cux4e3bux8981ux5185ux5bb9ux542bux5de5ux5177ux65b9ux6cd5ux7b49}

详见《实验指导书》。

\begin{itemize}
\tightlist
\item
  https://cs6.swfu.edu.cn/\textasciitilde wx672/lecture\_notes/linux/bash/shell\_basics.html
\item
  https://cs6.swfu.edu.cn/\textasciitilde wx672/lecture\_notes/linux/c/c\_dev.html
\end{itemize}

\subsection{Basic Commands and
Concepts}\label{basic-commands-and-concepts}

\subsubsection{Try the following
commands}\label{try-the-following-commands}

\begin{minted}[autogobble]{sh}
pwd; ls; cd; mkdir; cat; less; man; echo; help;
mv; cp; rm; 
vi;
\end{minted}

\textbf{Answer:}

\begin{minted}[autogobble]{sh}
# pwd - shows the present working directory.
pwd
# Output: /tmp

# ls - list files in current dir.
ls
# Output: lots of files and directories.

# cd - change directory
cd      # go home
cd /tmp # get into /tmp dir

# mkdir - create a new dir
mkdir /tmp/coursework # create a new dir coursework inside /tmp
mkdir -p /tmp/coursework/programming/{c,bash,python} # gets a set of dirs

# cat - concatenate files
cat > /tmp/tmp.txt # write into /tmp/tmp.txt
cat /tmp/tmp.txt   # show content of /tmp/tmp.txt
cat >> /tmp/tmp.txt # appned into it
cat /tmp/tmp.txt > /tmp/a.txt # copy tmp.txt to a.txt in /tmp

# less - view a file
less /tmp/a.txt # read a.txt

# man - read manual
man less # read the manual of less

# echo - write to stdout
echo 'hello, world!'
echo $PATH  # output the value of the variable PATH

# help - show help message of bash built-in commands
help echo

# mv - rename/move files
mv /tmp/a.txt /tmp/b.txt

# cp - copy
cp /tmp/b.txt /tmp/c.txt

# rm - remove files
rm /tmp/c.txt
 
# vi - a text editor
\end{minted}

\subsubsection{Try the following CLI
shortcuts}\label{try-the-following-cli-shortcuts}

\begin{itemize}
\tightlist
\item
  C-a, C-e, C-f, C-b, C-n, C-p, C-u, C-k, C-y, C-d, C-r, TAB
\end{itemize}

\textbf{Answer:}

\begin{itemize}
\tightlist
\item
  Ctrl-a: beginning of line
\item
  Ctrl-e: end of line
\item
  Ctrl-f: forward
\item
  Ctrl-b: backward
\item
  Ctrl-n: next
\item
  Ctrl-p: previous
\item
  Ctrl-r: reverse search
\item
  Ctrl-u: cut to beginning
\item
  Ctrl-k: kill (cut to end)
\item
  Ctrl-y: yank (paste)
\item
  Ctrl-d: delete a character
\item
  TAB: magic key, completion
\end{itemize}

\subsubsection{\texorpdfstring{Output redirection
(\mintinline[]{text}{>},
\mintinline[]{text}{>>})}{Output redirection (, )}}\label{output-redirection}

\begin{itemize}
\item
  To show the current time and date on the screen, you can do
  \mintinline[]{text}{date}. What if you do
  \mintinline[]{text}{date > file1}?

  \textbf{Answer:} Output to file1.
\item
  To show a string on the screen, you can do
  \mintinline[]{text}{echo 'Hello, world'}. How to output to file1?

  \textbf{Answer:} \mintinline[]{text}{echo 'Hello, world!' >> file1}.
  This can append to file1.
\end{itemize}

\subsubsection{\texorpdfstring{Wildcard characters
(\mintinline[]{text}{*},
\mintinline[]{text}{?})}{Wildcard characters (, )}}\label{wildcard-characters}

Suppose you have \mintinline[]{text}{file1}, \mintinline[]{text}{file2},
\mintinline[]{text}{hello}, \mintinline[]{text}{hello.c} in
\mintinline[]{text}{/tmp} dir, and two dirs \mintinline[]{text}{f} and
\mintinline[]{text}{h} in \mintinline[]{text}{/tmp}. What do the
following commands do?\\
- \mintinline[]{text}{mv f* f} - \mintinline[]{text}{mv h* h}

\textbf{Answer:}

\begin{itemize}
\tightlist
\item
  move \mintinline[]{text}{file1} and \mintinline[]{text}{file2} to
  \mintinline[]{text}{/tmp/f/}
\item
  move \mintinline[]{text}{hello} and \mintinline[]{text}{hello.c} to
  \mintinline[]{text}{/tmp/h/}.
\end{itemize}

Suppose you have files fa fb fc faa fbb fcc faaa fbbb fccc in
\mintinline[]{text}{/tmp/} dir. What's the output of
\mintinline[]{text}{ls f?}, \mintinline[]{text}{ls f??},
\mintinline[]{text}{ls f???}?

\textbf{Answer:}

\begin{itemize}
\tightlist
\item
  \mintinline[]{text}{ls f?} shows \mintinline[]{text}{fa fb fc}
\item
  \mintinline[]{text}{ls f??} shows \mintinline[]{text}{faa fbb fcc}
\item
  \mintinline[]{text}{ls f???} shows \mintinline[]{text}{faaa fbbb fccc}
\end{itemize}

\mintinline[]{text}{?} means matching \emph{any one character}.

\subsubsection{\texorpdfstring{Understanding
\mintinline[]{text}{ls -l}}{Understanding }}\label{understanding-ls--l}

\textbf{Answer:}

\begin{minted}[autogobble]{text}
-rw------- 1 sam sam    57 Apr 17  1998 weather.txt
drwxr-xr-x 6 sam sam   102 Oct  9  1999 web_page
-rw-rw-r-- 1 sam sam 27648 Feb 11 20:41 web_site.tar
-rw------- 1 sam sam   574 Dec 16  1998 xmas_file.txt
╷────┬──── ╷ ─┬─ ─┬─ ──┬── ─────┬────── ──────┬──────
│    │     │  │   │    │        │             │
│    │     │  │   │    │        │         File Name
│    │     │  │   │    │        │
│    │     │  │   │    │        └─── Modification Time
│    │     │  │   │    │
│    │     │  │   │    └──────────── Size (in bytes)
│    │     │  │   │
│    │     │  │   └───────────────── Group
│    │     │  │
│    │     │  └───────────────────── Owner
│    │     │
│    │     └──────────────────────── Number of hard links
│    │
│    └────────────────────────────── File Permissions
│
└─────────────────────────────────── File types
\end{minted}

\paragraph{File types}\label{file-types}

• ``d'' --- directory • ``-'' --- regular file • ``l'' --- soft link •
``c'' --- character device • ``b'' --- block device • ``s'' --- socket •
``p'' --- named pipe (FIFO)

\paragraph{File modes}\label{file-modes}

\begin{minted}[autogobble]{text}
drwxr-xr-x 2 wx672 wx672 4096 Sep 26 17:59 f/
drwxr-xr-x 2 wx672 wx672 4096 Sep 26 20:49 f-test/
drwxr-xr-x 2 wx672 wx672 4096 Sep 26 18:00 h/
 └┼┘└┼┘└┼┘
  │  │  │
  │  │  │
  │  │  └─── Other's permission
  │  └────── Group's permission
  └───────── Owner's permission
\end{minted}

\begin{itemize}
\tightlist
\item
  ``rwx'' --- readable, writable, executable
\item
  ``r-x'' --- readable, not writable, executable
\item
  ``r--'' --- readable, not writable, not executable
\item
  ``---'' --- not readable, not writable, not executable
\end{itemize}

\subsubsection{\texorpdfstring{File modes
(\mintinline[]{text}{chmod})}{File modes ()}}\label{file-modes-chmod}

Comment on the following commands:

\textbf{Answer:}

\begin{minted}[autogobble]{text}
chmod 777 f # everyone can rwx
chmod 700 f # owner can rwx, anyone else can do nothing
chmod 600 f # owner can rw-, anyone else can do nothing
chmod 000 f # nobody can do anything
chmod 755 f # owner can rwx, anyone else has r-x
chmod a+rwx f # same as 777
chmod a-rwx f # same as 000
chmod go-rwx f # group member and other users can do nothing
chmod u+x f # # add executable permission to owner
\end{minted}

\subsubsection{Shell variables}\label{shell-variables}

Show the values of these variables: PATH, PWD, HOME, USER.

\textbf{Answer:}

\begin{minted}[autogobble]{sh}
echo $PATH
echo $PWD
echo $HOME
echo $USER
\end{minted}

What does \mintinline[]{text}{PATH="./:$PATH"} do?

\textbf{Answer:} Change the value of \mintinline[]{text}{PATH} by
prepending it with \mintinline[]{text}{./}.

\comments %can optionally take an argument

\end{document}
