% Created 2019-05-28 Tue 21:55
% Intended LaTeX compiler: pdflatex
\documentclass{wx672ctexart}
		\usepackage{wx672common}
\usepackage{wx672hyperref}
\usepackage{wx672minted}
\author{王晓林}
\date{\today}
\title{Debian快速安装指导}
\hypersetup{
 pdfauthor={王晓林},
 pdftitle={Debian快速安装指导},
 pdfkeywords={},
 pdfsubject={},
 pdfcreator={Emacs 26.1 (Org mode 9.2.3)}, 
 pdflang={Cn}}
\begin{document}

\maketitle
\tableofcontents


\section{UEFI}
\label{sec:org902f24c}
现在(2018年10月)的电脑都很新潮,在主板上几乎都用UEFI取代了传统的BIOS。关于UEFI设置,我的经验是:
\begin{itemize}
\item 把下列和Windows相关的选项都关掉(disable):
\begin{itemize}
\item \texttt{Secure boot}
\item \texttt{QuickBoot/FastBoot}
\item \texttt{Intel Smart Response Technology (SRT)}
\item \texttt{FastStartup}
\end{itemize}
\item 如果在下面的安装过程中(硬盘分区的时候)看不到硬盘,那么你需要在UEFI设置里找到Intel Rapid Storage
Technology (Intel RST),把它设置为AHCI。
\end{itemize}
\section{安装}
\label{sec:orgbaf76a9}
安装Debian系统的大致步骤如下:
\begin{enumerate}
\item 先准备好一个安装盘(LiveUSB)。 【注意】 \textbf{一定要是64位的LiveUSB!}
\begin{enumerate}
\item \textbf{下载:} \url{http://mirrors.163.com/debian-cd/current/amd64/iso-cd/}
\begin{itemize}
\item 该目录中有三个(也许更多)iso文件,比如:
\begin{verbatim}
debian-9.5.0-amd64-netinst.iso        14-Jul-2018 19:12    291M
debian-9.5.0-amd64-xfce-CD-1.iso      14-Jul-2018 19:12    640M
debian-mac-9.5.0-amd64-netinst.iso    14-Jul-2018 19:12    294M
\end{verbatim}

只下载最小的那一个就好。
\end{itemize}
\item \textbf{制作U盘:} 在Debian/Ubuntu平台上制作启动U盘非常简单,(以Debian为例)敲个命令就行了:
\begin{minted}[mathescape=true,linenos=true,numbersep=5pt,frame=lines,framesep=2mm]{sh}
sudo cp debian-9.5.0-amd64-netinst.iso /dev/sdX
sudo sync
\end{minted}
【注意】
\begin{enumerate}
\item 把 \texttt{debian-9.5.0-amd64-netinst.iso} 换成你下载的iso文件的名字。文件名太长容易敲错,
那么就尽量用TAB键来补全。比如说,对于上面的长命令,你只需要键入前几个字母:
\begin{minted}[mathescape=true,linenos=true,numbersep=5pt,frame=lines,framesep=2mm]{sh}
sudo cp deb
\end{minted}
然后按一下 \texttt{TAB} 键,长长的文件名就自动都补全了。
\item 把 \texttt{X} 换成你的U盘对应的字母。用 \texttt{lsblk} 看一眼,你就知道是哪个字母了,通常不是b就是c。
\end{enumerate}
\end{enumerate}
\item 拔掉网线,从U盘重启系统,开始安装。大概半个小时,一个“最小系统”就装好了。
\item 完事后,拔掉U盘,重启系统。

【注意】
\begin{itemize}
\item 安装的时候不要选择中文语言环境,因为后面的安装配置工作都是在非图形环境下进行,
采用中文的话,你很可能会遭遇乱码。
\item 硬盘分区的时候,
\begin{itemize}
\item 如果你是装Linux单系统,就非常简单,没啥可说的;
\item 如果你是要装双系统(保留原来的Windows),那么,有三点烦,
\begin{enumerate}
\item 可用空间不够怎么办?删掉哪个分区?如何压缩原来的Windows分区?总之,烦!
\item 以后,霸道的Windows每次升级、更新,都会让你的Linux消失……
\item 装了双系统之后,通常(不争气的)你会选用熟悉的Windows系统,渐渐地,过不了多久,你就忘了电脑上还有一个Linux系统。
\end{enumerate}
所以,我很不愿意搭理装双系统的人。
\end{itemize}
\end{itemize}
\end{enumerate}

\section{安装配置}
\label{sec:orgcfbfe32}
最小系统装好之后,拔出U盘,重启系统,现在我们讲讲之后的事情……
\begin{enumerate}
\item 第一件事当然是把网线插好,启动你崭新的Debian,在屏幕提示下,输入用户名、密码。
之后,你就可以通过输入命令来让电脑为你工作了。

\begin{quote}
【注意】如果你的笔记本比较新潮,比如我新买的华为Honor Magicbook,没提供有线网接口,而且我们刚装好的最小系统里没有本机的无线网卡驱动,那么,请先参看\hyperref[sec:org3378f41]{本文末尾的附录:没有有线网卡怎么办?}
联网之后再继续。
\end{quote}

好了,假设你解决了所有的网络问题,现在我们可以继续了……一个“最小系统”干不了多少事情,所以我们先要安装更多的应用程序。注意,安装配置系统是管理员的工作,所以下面的操作都需要以管理员的身份来进行。如果你没有以管理员的身份登录(命令行提示符是 \texttt{\$} ),那么你现在就输入命令:
\begin{verbatim}
$ su
\end{verbatim}

然后输入管理员密码,你就成为管理员了(命令行提示符变成了 \texttt{\#} )。注意看命令行的提示符:
\begin{itemize}
\item \texttt{\$} 代表普通用户;
\item \texttt{\#} 代表root。
\end{itemize}

【注意】如果你在安装最小系统的时候没有设置root密码,那么现在显然无法用 \texttt{su} 命令,那么你直接用 \texttt{sudo} 命令就好了。

后面的安装配置工作显然是需要联网的,所以,先检查一下你的网络状况:
\begin{minted}[mathescape=true,linenos=true,numbersep=5pt,frame=lines,framesep=2mm]{sh}
ip a
\end{minted}
上面这行命令会列出你所有的网卡。仔细看一下,是否有一块网卡叫 \texttt{ethX} (\texttt{X} 是个数字,
通常就是 \texttt{0}. 如果是 \texttt{1} 或者 \texttt{2} 也没关系),或者叫 \texttt{enpXsY} (\texttt{X} 和 \texttt{Y} 都是数字)。仔细看看这块网卡是否已经获取到了IP地址。
如果你能看到类似下面这行信息,那就没问题了。
\begin{verbatim}
inet 192.168.1.110/24 brd 192.168.1.255 scope global dynamic eth0
\end{verbatim}

上面一行中的 \texttt{192.168.1.110} 就是我的 \texttt{eth0} 网卡获取到的IP地址。如果你看不到这样一行,那么先检查一下网线是否插好了,然后敲命令:
\begin{minted}[mathescape=true,linenos=true,numbersep=5pt,frame=lines,framesep=2mm]{sh}
dhclient enpXsY
\end{minted}
【注意】 \texttt{enpXsY} 是你的有线网卡的名字,也许是 \texttt{ethX} 。把 \texttt{X,Y} 换成相应的数字。

上面这条命令就是用来获取IP地址的。没什么意外的话,你马上就可以获取到IP了。之后,再敲 \texttt{ip a} 命令确认一下。还可以 \texttt{ping} 一下 \texttt{cs2.swfu.edu.cn} 看看网络是否联通了。

【注意】如果你用的是无线网卡,那么关于联网密码设置问题,请先参看\hyperref[sec:orgc0a8978]{本文末尾的附录:无线联网时的密码设置}。

\item 修改 \texttt{sources.list} 文件
\begin{minted}[mathescape=true,linenos=true,numbersep=5pt,frame=lines,framesep=2mm]{sh}
nano /etc/apt/sources.list
\end{minted}
把这个文件里原有的内容全部删除掉,然后添加下面这三行:
\begin{verbatim}
deb http://mirrors.163.com/debian testing main non-free contrib
deb http://mirrors.163.com/debian testing-updates main non-free contrib
deb http://mirrors.163.com/debian testing-proposed-updates main non-free contrib
\end{verbatim}

\item 存盘退出后,刷新一下软件包列表,并更新你的最小系统:
\begin{minted}[mathescape=true,linenos=true,numbersep=5pt,frame=lines,framesep=2mm]{sh}
apt update && apt dist-upgrade
\end{minted}
网络顺畅的话,这一步要花十几分钟的时间。
\item 现在,“机房装了什么,我就要装什么”。那就先把机房系统的软件清单弄到手,在\href{http://cs2.swfu.edu.cn/\~wx672/debian-install/list.laptop}{这里}。
这是我个人Debian笔记本电脑上的软件包列表。用 \texttt{wget} 把\href{http://cs2.swfu.edu.cn/\~wx672/debian-install/list.laptop}{这个软件清单}下载:
\begin{minted}[mathescape=true,linenos=true,numbersep=5pt,frame=lines,framesep=2mm]{sh}
cd
wget http://cs2.swfu.edu.cn/~wx672/debian-install/list.laptop
\end{minted}
【注意】这一步要以普通用户的身份来做,不要用root!注意看命令行的提示符:
\begin{itemize}
\item \texttt{\$} 代表普通用户;
\item \texttt{\#} 代表root;
\end{itemize}
\item 然后,开始大批量安装软件包:
\begin{minted}[mathescape=true,linenos=true,numbersep=5pt,frame=lines,framesep=2mm]{sh}
apt install `cat list.laptop`
\end{minted}
【注意】这行命令中用到的单引号是 \texttt{`} ,也就是键盘左上角的那个,而不是 \texttt{'} ,回车键左边这一个。95\%的同学会在这里出错。

如果网络顺畅的话,这一步大概需要半个小时。通常,安装过程是不需要人为干预的。但有的软件包在安装过程中,会停下来问你「Yes/no」。这种时候,你最好耐心把屏幕提示看明白。一般来讲,
直接按「回车」就好。
\item 一切顺利的话,网卡、声卡、显卡……都不需要额外的操心。但如果运气不太好的话(这通常是人品问题,因为你以学习的名义向家里要钱,最终却为了玩游戏而买了个声卡、显卡都特新潮的游戏机),
那么……假设你幡然悔悟了,可以看看本文末尾的附录:\hyperref[sec:org57926af]{关于硬件配置}。
\item 如果像我一样,你也是\hyperref[sec:orgc0a8978]{用USB无线网卡完成的安装},那么现在你应该可以拔掉USB无线网卡了。同时把刚才添加进 \texttt{/etc/network/interfaces} 文件的四行删除,或者注释掉。重启系统之后,用
\texttt{nmtui} 来连接无线网:
\begin{minted}[mathescape=true,linenos=true,numbersep=5pt,frame=lines,framesep=2mm]{sh}
sudo nmtui
\end{minted}
这是个界面挺友好的小工具,不用人教,你就会用。
\end{enumerate}

\subsection{配置}
\label{sec:orgadd5732}

\subsubsection{sudo}
\label{sec:org070412a}
\texttt{sudo} 的时候总要问密码,是不是很烦?那就不让它问了:
\begin{enumerate}
\item 变身root
\begin{minted}[mathescape=true,linenos=true,numbersep=5pt,frame=lines,framesep=2mm]{sh}
su
\end{minted}
输入密码,变成root。然后,
\item 建立一个新文件
\begin{minted}[mathescape=true,linenos=true,numbersep=5pt,frame=lines,framesep=2mm]{sh}
nano /etc/sudoers.d/your-user-name
\end{minted}
【注意】把 \texttt{your-user-name} 改成你自己的用户名。
\item 在里面写这么一行:
\begin{verbatim}
your-user-name  ALL = NOPASSWD: ALL
\end{verbatim}

【注意】把 \texttt{your-user-name} 改成你自己的用户名。
\item 改一下权限:
\begin{minted}[mathescape=true,linenos=true,numbersep=5pt,frame=lines,framesep=2mm]{sh}
chmod 0440 /etc/sudoers.d/your-user-name
\end{minted}
这以后 \texttt{sudo} 就不再问密码了。
\item 退回普通用户身份:
\begin{minted}[mathescape=true,linenos=true,numbersep=5pt,frame=lines,framesep=2mm]{sh}
exit
\end{minted}
\end{enumerate}
\subsubsection{dotfile}
\label{sec:orgc7ea5bf}
现在你的系统和机房的差不多一样了,唯一的差别就是你还没配置呢。
配置是个琐碎事,比较省事的办法就是把我的配置文件拷贝过来。最省事的拷贝方式就是
git( \textbf{以普通用户的身份来做} ):
\begin{minted}[mathescape=true,linenos=true,numbersep=5pt,frame=lines,framesep=2mm]{sh}
cd
git clone http://cs2.swfu.edu.cn/~wx672/dotfile/.git
#或者
git clone https://gitlab.swfu.edu.cn/wx672/dotfile.git
#或者
git clone https://github.com/wx672/dotfile.git
\end{minted}

上面三个网址应该都可以。 \texttt{git} 是著名的源代码管理工具,也就是版本控制工具。用它来管理配置文件也非常顺手。上面的命令完成之后, \texttt{ls} 一下,应该可以看到,你的 \texttt{\$HOME} 目录里多了一个子目录 \texttt{dotfile} ,里面放的都是杂七杂八的配置文件。

现在把 \texttt{dotfile} 目录里所有以 \texttt{dot.} 开头的文件和目录都链接到 \texttt{\$HOME} 目录里,
\begin{enumerate}
\item 先确保你在 \texttt{\$HOME}:
\begin{minted}[mathescape=true,linenos=true,numbersep=5pt,frame=lines,framesep=2mm]{sh}
cd
\end{minted}
\item 把旧的 \texttt{.bash*} 文件都删掉:
\begin{minted}[mathescape=true,linenos=true,numbersep=5pt,frame=lines,framesep=2mm]{sh}
rm -f .bash*
\end{minted}
\item 做链接:
\begin{minted}[mathescape=true,linenos=true,numbersep=5pt,frame=lines,framesep=2mm]{sh}
ln -sf dotfile/dot.* .
ln -sf dotfile/help/dot.* .
\end{minted}
现在 \texttt{ls} 一下,你会发现 \texttt{\$HOME} 目录里有了很多 \texttt{dot.} 开头的文件。
\item 把所有的 \texttt{dot.} 都变成 \texttt{.}, 也就是把文件名前面的 \texttt{dot} 都去掉,只留下 \texttt{.}:
\begin{minted}[mathescape=true,linenos=true,numbersep=5pt,frame=lines,framesep=2mm]{sh}
rename 's/dot//' dot.*
\end{minted}
现在用 \texttt{ls -al} 检查一下,我们需要的配置文件(也就是‘点’开头的文件)应该都在 \texttt{\$HOME} 目录里了。
\item 顺手把我准备好的「帮助墙纸」也下载下来吧。墙纸上列出了我们最常用的快捷键。
\begin{minted}[mathescape=true,linenos=true,numbersep=5pt,frame=lines,framesep=2mm]{sh}
wget -O .keys.png http://cs2.swfu.edu.cn/~wx672/tex-fun/keys/keys-1.png
\end{minted}
 等以后在XWindow环境下,按 \texttt{Super-F1} 应该就可以弹出这张墙纸了。
( \texttt{Super} 键就是键盘左下方的 \texttt{Win} 键)
\item 我的Emacs配置里用到了很多插件,自然你也需要它们,否则Emacs不能正常工作。
\begin{enumerate}
\item 先把我的插件包下载下来
\begin{minted}[mathescape=true,linenos=true,numbersep=5pt,frame=lines,framesep=2mm]{sh}
wget http://cs2.swfu.edu.cn/~wx672/debian-install/elpa.tgz
\end{minted}
\item 放到Emacs的配置文件目录里
\begin{minted}[mathescape=true,linenos=true,numbersep=5pt,frame=lines,framesep=2mm]{sh}
mv elpa.tgz ~/.emacs.d/
\end{minted}
\item 然后解压缩
\begin{minted}[mathescape=true,linenos=true,numbersep=5pt,frame=lines,framesep=2mm]{sh}
cd ~/.emacs.d
tar zxf elpa.tgz
\end{minted}
\item 测试一下
\begin{minted}[mathescape=true,linenos=true,numbersep=5pt,frame=lines,framesep=2mm]{sh}
emacs --debug-init
\end{minted}
如果报错,就把出错信息发给我(wx672ster@gmail.com)。  
当然,如果你能自己解决问题那再好不过了。
\end{enumerate}
\end{enumerate}
\subsubsection{Auto login}
\label{sec:orge980901}
简单起见,我们只讲“怎么做”,先不管“为什么”。(需要sudo,也就是用root身份来做)
\begin{enumerate}
\item 拷贝配置文件
\begin{minted}[mathescape=true,linenos=true,numbersep=5pt,frame=lines,framesep=2mm]{sh}
sudo cp -r ~/dotfile/etc/systemd/system/getty@tty1.service.d/ /etc/systemd/system/
\end{minted}
注意, \texttt{\textasciitilde{}} (也就是波浪线), 它代表你的 \texttt{\$HOME} 目录。
\item 修改
\begin{minted}[mathescape=true,linenos=true,numbersep=5pt,frame=lines,framesep=2mm]{sh}
sudo nano /etc/systemd/system/getty@tty1.service.d/override.conf
\end{minted}
在这个 \texttt{override.conf} 文件里应该只有如下三行:
\begin{verbatim}
[Service]
ExecStart=
ExecStart=-/sbin/agetty --autologin wx672 --noclear %I $TERM
\end{verbatim}
你只要把其中的 \texttt{wx672} 改成你自己的用户名就可以了。
\end{enumerate}

\subsubsection{中文语言环境}
\label{sec:org3fbebb1}
注意,我们其实并不需要一套纯正的中文环境,我们只是需要输入和阅读中文。
其它方面,比如窗口菜单、提示信息、man page,我觉得还是看英文比较好。

千万别说“我英文差,还是用中文算了”。要知道,就是因为你
“这个差、那个不行、这个不懂、那个不会……”所以你才来上学的,不是吗?
既然知道“差”,那就该好好学习,提高它。
英文是用熟的,如果你总是回避它,就总也不会长进了。

好了,下面我们来配置一个简单的中文环境。相关中文字体我们已经安装好了。下面只需要:
\begin{enumerate}
\item 安装中文字体和输入法。
\begin{minted}[mathescape=true,linenos=true,numbersep=5pt,frame=lines,framesep=2mm]{sh}
cd
wget http://cs2.swfu.edu.cn/~wx672/debian-install/list.chinese
sudo apt install `cat list.chinese`
\end{minted}

\item 选择 \texttt{locale}
\begin{minted}[mathescape=true,linenos=true,numbersep=5pt,frame=lines,framesep=2mm]{sh}
sudo dpkg-reconfigure locales
\end{minted}
在这一长串列表中,只要选中
\begin{itemize}
\item[{$\boxtimes$}] \texttt{en\_US.UTF-8 UTF-8}
\item[{$\boxtimes$}] \texttt{zh\_CN.GB18030 GB18030}
\item[{$\boxtimes$}] \texttt{zh\_CN.UTF-8 UTF-8}
\end{itemize}
就可以了。默认语言环境选 \texttt{None} 。
\item 拷贝一个小配置文件:
\begin{minted}[mathescape=true,linenos=true,numbersep=5pt,frame=lines,framesep=2mm]{sh}
sudo cp ~/dotfile/etc/default/locale /etc/default
\end{minted}
\item 顺带再拷贝一个小文件:
\begin{minted}[mathescape=true,linenos=true,numbersep=5pt,frame=lines,framesep=2mm]{sh}
sudo cp ~/dotfile/etc/default/keyboard /etc/default
\end{minted}
这是把你的 \texttt{CapsLock} 键变成 \texttt{Ctrl} 键,
因为Unix用户经常要用 \texttt{Ctrl} 键,从来不用 \texttt{CapsLock} 。

好了,现在安装配置的工作基本就结束了。你可以重启一下系统。
系统重启后,看到的应该就是学院机房里那个没有桌面的“桌面系统”了。
不记得快捷键了?按 \texttt{Super-F1} 。

中文输入法,我选用的是 \texttt{fcitx}, 因为感觉它的bug要少一些,比较稳定。
如果你需要配置它的话,就:
\begin{minted}[mathescape=true,linenos=true,numbersep=5pt,frame=lines,framesep=2mm]{sh}
fcitx-configtool
\end{minted}
你最好和我一样,用 \texttt{Shift-space} 来激活输入法,因为 \texttt{Ctrl-space} 在Emacs里有特殊用途。

注意:fcitx依赖于dbus-x11,而显然fcitx软件包的维护者忽略了这个小细节。那么我们就自己把它装上呗:
\begin{minted}[mathescape=true,linenos=true,numbersep=5pt,frame=lines,framesep=2mm]{sh}
sudo apt install dbus-x11
\end{minted}
\end{enumerate}

\section{附录:没有有线网卡怎么办?}
\label{sec:org3378f41}
办法很多:
\begin{enumerate}
\item 用Android手机的USB Tethering功能。以我自己的手机系统为例(LineageOS 15.1/Android 8.1),
很简单,
\begin{enumerate}
\item 用USB线连接手机和电脑;
\item 在手机的「系统设置」里有个搜索框,在里面输入“tethering”,马上就能找到“Hotspot \&
Tethering”,激活里面的USB Tethering功能就行了;
\item 在电脑上,敲命令 \texttt{ip a} 应该能看到一块有线网卡。比如,
\begin{verbatim}
3: enp2s0f4u2: <BROADCAST,MULTICAST,UP,LOWER_UP> mtu 1500 qdisc pfifo_fast state UNKNOWN group default qlen 1000
   link/ether 26:b1:c7:c5:02:1f brd ff:ff:ff:ff:ff:ff
\end{verbatim}
从上面的屏幕输出信息可以看到,这块有线网卡的名字是 \texttt{enp2s0f4u2} 。然后,以root身份,
敲下面这条命令:
\begin{minted}[mathescape=true,linenos=true,numbersep=5pt,frame=lines,framesep=2mm]{sh}
sudo dhclient enp2s0f4u2
\end{minted}
你就可以获得一个IP地址了,也就是说,你已经成功联网了。
\end{enumerate}
\item 去找一个USB无线网卡试试。我找到一个Realtek的指甲盖大小的USB无线网卡,不需要驱动,插上就能用。我也尝试过两个比较古老的tp-link无线网卡,不好使。
\item 另外,如果你真的和我一样,用的是华为Honor Magicbook,那么也许你不必去找USB网卡,可以先试试能否让内置网卡工作。Magicbook的内置网卡是Intel的。既然完成后面的安装步骤之后它能正常工作,那我想,现在使使劲应该也能解决问题吧。但毕竟我还没有亲自尝试过,所以只能先给出一些想法:
\begin{itemize}
\item 之所以内置网卡暂时不工作,我怀疑是我们用来安装最小系统的iso文件不够新。它是以Debian稳定版(stretch)为基础做出来的,其中的内核(4.9)和相应固件(firmware-iwlwifi)都偏旧,
可能尚不支持这么新潮(2018年)的硬件。所以,可以试试把内核和相应固件从稳定版更新到测试版(buster)。在没有网络连接的情况下,显然这需要我们另找办法下载,并手动安装一些软件包,包括:
\begin{itemize}
\item \href{https://packages.debian.org/buster/linux-image-amd64}{linux-image-amd64}
\item \href{https://packages.debian.org/buster/firmware-iwlwifi}{firmware-iwlwifi}
\item 还有若干被上述两个软件包依赖的软件包
\end{itemize}
\item 一些参考链接:
\begin{itemize}
\item \href{https://unix.stackexchange.com/questions/283722/how-to-connect-to-wifi-from-command-line}{How to connect to WiFi from command line?}
\item \href{https://askubuntu.com/questions/974/how-can-i-install-software-or-packages-without-internet-offline}{How can I install software packages without Internet?}
\item \href{https://commandlinefanatic.com/cgi-bin/showarticle.cgi?article=art016}{Installing Debian without a Network}
\item \href{https://wiki.debian.org/WiFi}{Debian Wiki --- WiFi}
\end{itemize}
\end{itemize}
\item 如果上述办法都不成功,那么这招肯定行,就是笨点。直接去下面这些镜像站下载完整的安装盘。
\begin{itemize}
\item \url{http://mirrors.163.com/debian-cd/current/amd64/iso-dvd/}
\item \url{http://mirrors.ustc.edu.cn/debian-cd/current/amd64/iso-dvd/}
\end{itemize}

完整的DVD安装盘包含3个iso文件,你可以先下载第一个试试。如果里面有了你需要的无线网卡驱动和相关程序,那么激活网卡之后,你就可以直接网络安装了,无需下载其它的iso文件了。
\end{enumerate}

\subsection{无线联网时的密码设置}
\label{sec:orgc0a8978}
无线联网时通常是要输入密码的,所以我们需要修改一个配置文件 \texttt{/etc/network/interfaces} ,很简单,编辑这个小文件:
\begin{minted}[mathescape=true,linenos=true,numbersep=5pt,frame=lines,framesep=2mm]{sh}
sudo nano /etc/network/interfaces
\end{minted}
\texttt{nano} 是个很简单的编辑器,用起来应该不会有什么困难吧。 
\texttt{nano} 窗口的最下两行都是快捷键提示,最重要的两个是:
\begin{enumerate}
\item 存盘: \texttt{Ctrl-o}
\item 退出: \texttt{Ctrl-x}
\end{enumerate}

在这个文件的最后加上如下几行:
\begin{verbatim}
iface tmp inet dhcp
wireless-essid MY-ESSID
wpa-ssid MY-ESSID
wpa-psk PASSWORD
\end{verbatim}
【注意】把 \texttt{MY-ESSID} 和 \texttt{PASSWORD} 换成你自己的无线网络的名字和密码。

然后,用下面这条命令来连接无线网:
\begin{minted}[mathescape=true,linenos=true,numbersep=5pt,frame=lines,framesep=2mm]{sh}
sudo ifup WLANCARD=tmp
\end{minted}
【注意】把 \texttt{WLANCARD} 换成你自己的无线网卡的名字,网卡的名字通常是w开头的,比如我的无线网卡名字就是 \texttt{wlp1s0} ,那么我用的联网命令就是:
\begin{minted}[mathescape=true,linenos=true,numbersep=5pt,frame=lines,framesep=2mm]{sh}
sudo ifup wlp1s0=tmp
\end{minted}

\section{附录:关于硬件配置}
\label{sec:org57926af}
首先,当然是要搞清楚你到底有哪些硬件。很简单:
\begin{minted}[mathescape=true,linenos=true,numbersep=5pt,frame=lines,framesep=2mm]{sh}
lspci
#想看更详细的信息,就:
lspci -vvv
\end{minted}

总之, \texttt{lspci} 能列出你所有外围设备的详细信息。然后,如果你的有线或无线网卡是Realtek,或者Atheros牌子的,那么你需要安装相应的firmware(固件)。
\begin{minted}[mathescape=true,linenos=true,numbersep=5pt,frame=lines,framesep=2mm]{sh}
#如果是Realtek网卡,就:
sudo apt install firmware-realtek
#如果是Atheros网卡,就:
sudo apt install firmware-atheros
#如果是Intel网卡,就:
sudo apt install firmware-iwlwifi
\end{minted}

并不是所有的网卡都需要安装相应的固件,甚至上面提到的Realtek, Atheros, Intel网卡,即使不装固件,网卡也可能工作,但未必那么稳定。所以,既然有固件,那还是装上比较好。同样,针对声卡、显卡,Debian库里也有很多固件。下面这条命令可以列出库里所有的固件包:
\begin{minted}[mathescape=true,linenos=true,numbersep=5pt,frame=lines,framesep=2mm]{sh}
aptitude search ^firmware
\end{minted}
大概也就三十几个吧。找找有没有和你的硬件相关的。怎么知道是否相关呢?看看固件包的详细信息呗,比如:
\begin{minted}[mathescape=true,linenos=true,numbersep=5pt,frame=lines,framesep=2mm]{sh}
apt show firmware-atheros
\end{minted}
于是就知道了这个固件适用于哪些网卡。

关于显卡,听说Nvidia显卡比较难伺候,好在我从来没碰到过,因为只有游戏本才配置这么贵的显卡。如果你(曾经人品不好)不幸碰到了,那么,省事起见,我建议你暂时不要用它,就用主板上内置的(通常是Intel)显卡就好。直到有一天你成了一个熟练的Linux用户之后,再把它激活。
\section{附录:\LaTeX{} (非必须)}
\label{sec:org393216c}
在Linux平台,你不用非要学习使用\LaTeX{}来排版你的文章、报告、论文,
因为你已经有了一套开源的office软件。如果前面的事情你都顺利完成了,那么现在只需要按
\texttt{Super-o} (键盘上那个Win键,我们叫它Super键)
就可以调出著名的Libreoffice了。然后,你完全可以像在Windows平台上那样写东西。

但是,「你们这些使用Linux的人,不就是“装逼、扮酷”嘛」,既然他嫌你酷,那么你就再酷一点嘛。
TeXLive是一套优秀而庞大的排版系统,我们只需要安装使用它提供的少数十几个软件包就够了。

我个人用到的\LaTeX{}软件包列表在\href{http://cs2.swfu.edu.cn/\~wx672/debian-install/list.texlive}{这里}:
\begin{verbatim}
$ wget http://cs2.swfu.edu.cn/~wx672/debian-install/list.texlive
$ sudo apt install `cat list.texlive`
\end{verbatim}

上面这两行命令和我们前面用到的很相似吧。第一行是下载 \texttt{list.texlive} 文件,
也就是我的TeXLive软件包列表。第二行是安装文件里的所有软件包。
安装好以后,如果想“酷”,那么你要做如下几件事情:
\begin{enumerate}
\item 熟悉Emacs的使用。为什么非要用Emacs啊?因为它为编辑\LaTeX{}文件提供了最好的支持。而且,我不想在这里唠唠叨叨,如果你想看我为Emacs做的广告,可以看我在「知乎」上写的一个小答案:
\begin{itemize}
\item \url{https://www.zhihu.com/question/30955165/answer/70799403}
\end{itemize}

顺带贩卖一下我为Debian做的广告:
\begin{itemize}
\item \url{https://www.zhihu.com/question/19676224/answer/29321011}
\end{itemize}

\item 学习一点关于\LaTeX{}的基础知识,我觉得两三个小时应该够了吧。我推荐 \texttt{lshort}:
\begin{verbatim}
texdoc -l lshort
\end{verbatim}

上面这条命令会列出几个相关的PDF文件,你要关注的是前两个:
\begin{verbatim}
1 /usr/share/texlive/texmf-dist/doc/latex/lshort-english/lshort.pdf
2 /usr/share/texlive/texmf-dist/doc/latex/lshort-chinese/lshort-zh-cn.pdf
\end{verbatim}

我鼓励你看英文原版,至少应该中英对照着看吧。
\item 如果你打算尝试用\LaTeX{}来写你的毕业论文,那么我为你提供了点小帮助:
\begin{itemize}
\item \url{https://github.com/wx672/texmf/tree/master/doc/latex/swfu/swfcthesis}
\item \url{http://cs2.swfu.edu.cn/\~wx672/texmf/doc/latex/swfu/swfcthesis/}
\end{itemize}
上面两个链接里的内容是一样的,看哪个都行。有问题可以向我求助。

Happy LaTeXing!
\end{enumerate}
\end{document}