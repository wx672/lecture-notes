\documentclass{wx672ctexart}

\usepackage{wx672bib}
\addbibresource{os.bib}

\title{《Linux应用》教学大纲}
\author{王晓林}

\pagestyle{plain}

\begin{document}

\maketitle

\begin{description}
\item[Linux原理及应用]
\item[每周课次: 4]
\item[学分:] 4
\item[前导课程:] C编程
\item[课程简介:] 本课程涵盖用户层面(应用)与内核层面(原理)的Linux知识。用户层面的内容
  包括:
  \begin{itemize}
  \item 命令行基础
  \item Bash编程
  \item 软件开发环境介绍
  \item 系统管理
  \item 网络管理
  \end{itemize}
  在内核层面,我们将了解Linux是如何解决操作系统设计方面的主要问题的。
  \begin{multicols}{2}
    \begin{itemize}
    \item 系统启动的过程
    \item 内存寻址
    \item 中断处理
    \item 系统调用接口函数
    \item 进程管理
    \item CPU调度算法
    \item Ext2文件系统
    \item 输入/输出管理
    \end{itemize}
  \end{multicols}
  \item[教程与参考数目:]\hfill
  \nocite{cooper10bash,web:debianhandbook,web:debkernelhandbook,bovet2005understanding,tanenbaum2008modern,bovet2005understanding}
  \printbibliography[heading=none]{}
\end{description}

\end{document}

%%% Local Variables:
%%% mode: latex
%%% TeX-master: t
%%% End:
