% Created 2013-03-06 Wed 20:13
\documentclass[12pt,a4paper]{article}
\usepackage[top=2cm,bottom=3cm,left=3cm,right=3cm]{geometry}
\usepackage{listings}
\usepackage{graphicx}
\graphicspath{{./figs/}{../figs/}{./}{../}} %note that the trailing / is required
\usepackage{indentfirst}
\usepackage[indentafter,pagestyles]{titlesec}
\usepackage{xltxtra}
\usepackage{xeCJK}
\usepackage{hyperref}
\usepackage[utf8]{inputenc}
\usepackage{fixltx2e}
\usepackage{longtable}
\usepackage{float}
\usepackage{wrapfig}
\usepackage{soul}
\usepackage{marvosym}
\usepackage{wasysym}
\usepackage{latexsym}
%\usepackage{amsmath,amsfonts,amssymb}
\tolerance=1000
\setlength{\parindent}{2.5em}
\setmainfont{DejaVu Serif}
\setsansfont{DejaVu Sans}
\setmonofont{DejaVu Sans Mono}
\setCJKmainfont[BoldFont={WenQuanYi Zen Hei}]{SimSun}
\setCJKfamilyfont{hei}{WenQuanYi Zen Hei}
\setCJKfamilyfont{song}{SimSun}
\XeTeXlinebreaklocale "zh"
\XeTeXlinebreakskip = 0pt plus 1pt
\renewcommand{\contentsname}{目录}
\renewcommand{\listfigurename}{插图目录}
\renewcommand{\listtablename}{表格目录}
\renewcommand{\abstractname}{摘要}
\renewcommand{\appendixname}{附录}
\renewcommand{\indexname}{索引}
\renewcommand{\figurename}{图}
\renewcommand{\tablename}{表}
\renewcommand{\refname}{参考文献} % article.cls 
\providecommand{\alert}[1]{\textbf{#1}}

\title{《Linux应用基础》考核大纲}
\author{王晓林}
\date{2015年9月6日}

\begin{document}

\maketitle

\begin{itemize}
\item 课程编号:A05010, A05011  
\item 课程名称:Linux应用基础
\item 任课教师:王晓林
\item 适应专业:电信、信计、计算机等本科专业
\item 授课学时:32/48
\item 考试方式:闭卷机试
\item 命题规则:按《西南林学院考试命题规则》(试行)(1996年10月修订)执行
\item 考试时间:120分钟
\item 推荐教材:《Advanced Bash Scripting Guide》
\item 课程性质及教学目的:本课为通信和计算机专业的选修课. 通过学习, 要求学生全面了解Linux系
  统的工作环境,学会在Linux下完成日常工作和开发工作。
\end{itemize}
  
各章节考核目标如下:
\clearpage

\section{GNU/Linux简介}

\subsection{考核知识点}

\begin{itemize}
\item UNIX, GNU, Linux的诞生、创始人
\item 开源软件
\end{itemize}

\subsection{考核要求}

\begin{itemize}
\item 了解UNIX, GNU, Linux的基本历史常识
\item 了解开源软件
\end{itemize}

\section{Shell基础}

\subsection{考核知识点}

\begin{itemize}
\item UNIX文件系统基本结构
\item 路径、目录、特殊文件
\item 常用的命令行命令
\item Shell编程
\end{itemize}

\subsection{考核要求}

\begin{itemize}
\item 理解UNIX文件系统目录结构
\item 熟悉常用命令:ls, cd, pwd, cp, rm, man, help, ln, cat, less, mkdir\ldots{}
\item 会编简单的程序
\item 能看懂复杂的程序
\end{itemize}

\section{常用软件工具}

\subsection{考核知识点}

\begin{itemize}
\item 编辑器: emacs, vim
\item 网络工具: firefox, lftp, wget, mutt, pidgin \ldots{}
\item 办公自动化: openoffice.org
\item 图像处理: gimp, imagemagick, dia, xfig, inkscape, hugin
\end{itemize}

\subsection{考核要求}

\begin{itemize}
\item 能熟练使用emacs和vim
\end{itemize}

\section{软件开发环境}

\subsection{考核知识点}

\begin{itemize}
\item GCC
\item make
\item gdb
\item 可视化编程: Qt4, glade, gambas, Tcl/Tk
\end{itemize}

\subsection{考核要求}

\begin{itemize}
\item 了解GCC, Makefile, GDB的使用
\end{itemize}

\section{Debian GNU/Linux系统管理}

\subsection{考核知识点}

\begin{itemize}
\item 最小系统的安装
\item apt
\item Debian 管理工具
\end{itemize}

\subsection{考核要求}

\begin{itemize}
\item 会安装系统
\item 会安装、卸载软件
\item 会更新系统和软件
\end{itemize}

\vspace{10em}
\begin{table}[h]
  \begin{tabular}{p{20em}ll}
    &撰稿人(职称):&王晓林(讲师)\\
    &审核人(职称):&\\
    &审定人(职称):&\\
    &&\\
    &制定日期:&2015年9月6日\\
  \end{tabular}
\end{table}

\end{document}