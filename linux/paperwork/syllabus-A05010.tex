% Created 2013-03-06 Wed 20:00
\documentclass[12pt,a4paper]{article}
\usepackage[top=2cm,bottom=3cm,left=3cm,right=3cm]{geometry}
\usepackage{listings}
\usepackage{graphicx}
\graphicspath{{./figs/}{../figs/}{./}{../}} %note that the trailing / is required
\usepackage{indentfirst}
\usepackage[indentafter,pagestyles]{titlesec}
\usepackage{xltxtra}
\usepackage{xeCJK}
\usepackage{hyperref}
\usepackage[utf8]{inputenc}
\usepackage{fixltx2e}
\usepackage{longtable}
\usepackage{float}
\usepackage{wrapfig}
\usepackage{soul}
\usepackage{marvosym}
\usepackage{wasysym}
\usepackage{latexsym}
%\usepackage{amsmath,amsfonts,amssymb}
\tolerance=1000
\setlength{\parindent}{2.5em}
\setmainfont{DejaVu Serif}
\setsansfont{DejaVu Sans}
\setmonofont{DejaVu Sans Mono}
\setCJKmainfont[BoldFont={WenQuanYi Zen Hei}]{SimSun}
\setCJKfamilyfont{hei}{WenQuanYi Zen Hei}
\setCJKfamilyfont{song}{SimSun}
\XeTeXlinebreaklocale "zh"
\XeTeXlinebreakskip = 0pt plus 1pt
\renewcommand{\contentsname}{目录}
\renewcommand{\listfigurename}{插图目录}
\renewcommand{\listtablename}{表格目录}
\renewcommand{\abstractname}{摘要}
\renewcommand{\appendixname}{附录}
\renewcommand{\indexname}{索引}
\renewcommand{\figurename}{图}
\renewcommand{\tablename}{表}
\renewcommand{\refname}{参考文献} % article.cls 
\providecommand{\alert}[1]{\textbf{#1}}

\title{《Linux应用基础》课程大纲}
\author{王晓林}
\date{2015年9月6日}

\begin{document}

\maketitle

\begin{itemize}
\item 课程编号: A05010
\item 学时: 32 (理论: 16; 实验: 16)
\item 学分: 1.5
\item 实习: 0
\item 面向专业: 信息工程
\end{itemize}

\tableofcontents
\clearpage

\section{课程大纲}

\subsection{课程内容}

\begin{enumerate}
\item GNU/Linux 和开源运动
\begin{itemize}
\item 什么是开源
\item 什么是GNU?
\item 什么是 Linux?
\item 开源软件能干什么?
\item 怎样学习Linux?
\end{itemize}
\item Shell基础
\begin{itemize}
\item UNIX文件系统
\item 路径,目录,特殊文件
\item 基本 shell 命令
\item Shell 编程
\end{itemize}
\item 常用软件工具
\begin{itemize}
\item 编辑器: emacs, vim
\item 网络工具: firefox, lftp, wget, mutt, pidgin \ldots{}
\item 办公自动化: openoffice.org
\item 图像处理: gimp, imagemagick, dia, xfig, inkscape, hugin
\end{itemize}
\item 软件开发环境
\begin{itemize}
\item GCC
\item make
\item gdb
\item 可视化编程: Qt4, glade, gambas, Tcl/Tk
\end{itemize}
\item Debian GNU/Linux 系统管理
\begin{itemize}
\item 最小系统的安装
\item apt
\item Debian 管理工具
\end{itemize}
\end{enumerate}
\subsection{实验内容}
参见《实验教学大纲》。

\subsection{实习}
无
\subsection{考核}

\begin{itemize}
\item 考试: 80\%
\item 作业: 20\%
\end{itemize}

\subsection{参考教材}

教材:
\begin{itemize}
\item 《完美应用Ubuntu/LAMP技术大系》,UbuntuChina,电子工业出版社,2008;
\end{itemize}
参考资料:
\begin{itemize}
\item \emph{Advanced Bash-Scripting Guide}, Mendel Cooper, Version 4.1.01, 25 October 2006
\item \emph{Debian system administration guide}
\end{itemize}

\section{课程说明}

\subsection{课程性质和要求}

目前《操作系统原理》课程都是以开源的Linux为范本进行教学。因此学生必须要有一定的Linux应用基
础能力。本课程介绍给同学如下内容:
\begin{itemize}
\item GNU/Linux的过去、现在、和未来
\item Bash
\item Linux下的软件开发环境
\item Linux系统管理和网络管理
\end{itemize}

\subsection{课程重点}

\begin{itemize}
\item Shell命令行
\item 软件工具,开发环境
\item Linux下的C编程
\end{itemize}

\subsection{作业、实习要求}
按时交作业。

\subsection{与其它课程的关系}

\begin{itemize}
\item 前期课程:大学计算机基础
\item 后期课程:网络课程,编程课程,操作系统课程
\end{itemize}

\subsection{课时安排}

\begin{center}
  \begin{tabular}{lrr}
    \hline
    课程内容              &  理论学时  &  实验学时  \\
    \hline
    GNU/Linux 与开源运动  &         2  &            \\
    Shell 基础            &        6  &         6  \\
    常用软件工具          &         2  &         4  \\
    软件开发环境          &         4  &         4  \\
    Debian系统管理        &         2  &         2  \\
    \hline
  \end{tabular}
\end{center}

\subsection{特殊说明}
本课程以应用为主,最好全部授课安排在机房进行

\vspace{10em}
\begin{table}[h]
  \begin{tabular}{p{20em}ll}
    &撰稿人(职称):&王晓林(讲师)\\
    &审核人(职称):&\\
    &审定人(职称):&\\
    &&\\
    &制定日期:&2015年9月6日\\
  \end{tabular}
\end{table}

\clearpage
\section{实验教学大纲}

\begin{itemize}
\item 课程编号: A05010
\item 学时: 32 (理论: 16; 实验: 16)
\item 学分: 1.5
\item 实习: 0
\item 授课对象: 信息工程
\end{itemize}

\subsection{实验教学的目的和要求}
本课程的目的就是让学生熟悉Linux下的工作环境和开发环境,为后续课程打下坚实的基础。

\subsection{实践教学大纲}

\begin{center}
  \begin{tabular}{lr}
    \hline
    实验安排        &  学时  \\
    \hline
    shell基础       &     6  \\
    常用软件工具    &     4  \\
    软件开发环境    &     4  \\
    Debian系统管理  &     2  \\
    \hline
  \end{tabular}
\end{center}

\subsection{实验设备要求}

\begin{itemize}
\item Debian/Ubuntu PC
\end{itemize}

\subsection{实验内容}

\begin{enumerate}
\item Shell基础
  \begin{itemize}
  \item UNIX文件系统
  \item 路径,目录,特殊文件
  \item 基本 shell 命令
  \item Shell 编程
  \end{itemize}
\item 常用软件工具
  \begin{itemize}
  \item 编辑器: emacs, vim
  \item 网络工具: firefox, lftp, wget, mutt, pidgin \ldots{}
  \item 办公自动化: openoffice.org
  \item 图像处理: gimp, imagemagick, dia, xfig, inkscape, hugin
  \end{itemize}
\item 软件开发环境
  \begin{itemize}
  \item GCC
  \item make
  \item gdb
  \item 可视化编程: Qt4, glade, gambas, Tcl/Tk
  \end{itemize}
\item Debian GNU/Linux 系统管理
  \begin{itemize}
  \item 最小系统的安装
  \item apt
  \item Debian 管理工具
  \end{itemize}
\end{enumerate}
\subsection{实验报告要求}
按规定格式完成,不得延误

\subsection{成绩考核}

\begin{itemize}
\item 实验报告满分100,60分及格
\end{itemize}

\subsection{实验指导和参考书目}
教材:
\begin{itemize}
\item 《完美应用Ubuntu/LAMP技术大系》,UbuntuChina,电子工业出版社,2008;
\end{itemize}
参考资料:
\begin{itemize}
\item \emph{Advanced Bash-Scripting Guide}, Mendel Cooper, Version 4.1.01, 25 October 2006
\item \emph{Debian system administration guide}
\end{itemize}

\subsection{特别说明}
本课程以应用为主,最好全部授课安排在机房进行
  
\section{课程简介}

\begin{itemize}
\item 课程编号: A05010
\item 学时: 32 (理论: 16; 实验: 16)
\item 学分: 1.5
\item 实习: 0
\item 面向专业: 计算机科学与技术,电子信息工程,信息与计算机技术
\item 前期课程:英语,大学计算机基础
\item 课程性质和要求目前《操作系统原理》课程都是以开源的Linux为范本进行教学。因此学生必须要
  有一定的Linux应用基础能力。本课程介绍给同学如下内容:
  \begin{itemize}
  \item GNU/Linux的过去、现在、和未来
  \item Bash
  \item Linux下的软件开发环境
  \item Linux系统管理和网络管理
  \end{itemize}
\item 课程重点
  \begin{itemize}
  \item Shell命令行
  \item 软件工具,开发环境
  \item Linux下的C编程
  \end{itemize}
\item 参考教材
  \begin{itemize}
  \item 教材:
    \begin{itemize}
    \item 《完美应用Ubuntu/LAMP技术大系》,UbuntuChina,电子工业出版社,2008;
    \end{itemize}
  \item 参考资料:
    \begin{itemize}
    \item \emph{Advanced Bash-Scripting Guide}, Mendel Cooper, Version 4.1.01, 25 October
      2006
    \item \emph{Debian system administration guide}
    \end{itemize}
  \end{itemize}
\end{itemize}

\vspace{10em}
\begin{table}[h]
  \begin{tabular}{p{20em}ll}
    &撰稿人(职称):&王晓林(讲师)\\
    &审核人(职称):&\\
    &审定人(职称):&\\
    &&\\
    &制定日期:&2015年9月6日\\
  \end{tabular}
\end{table}

\end{document}