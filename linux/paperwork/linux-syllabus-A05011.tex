\documentclass{wx672ctexart}

\usepackage{wx672hyperref,wx672bib}
\addbibresource{os.bib}

% \newcommand{\Sign}{\begin{flushright}%
%   \begin{tabular}{r@{:\,}l}%
%     撰稿人(职称)&王晓林(讲师)\\%
%     审核人(职称)&赵家刚(副教授)\\%
%     审定人(职称)&狄光智(副教授)\\%
%     制定日期&2012年11月1日%
%   \end{tabular}%
% \end{flushright}}

\title{《Linux应用基础》课程大纲}
\author{王晓林}

\begin{document}

\maketitle
\tableofcontents

\begin{itemize}
\item 课程编号:41100165
\item 学时:32 (理论: 16; 实验: 16)
\item 周学时:2
\item 学分:2
\item 实习:0
\item 面向专业: 计算机科学与技术,电子信息工程
\end{itemize}

\section{课程大纲}

\subsection{课程内容}

\begin{enumerate}
\item GNU/Linux和开源运动
\begin{itemize}
\item 什么是开源
\item 什么是GNU?
\item 什么是 Linux?
\item 开源软件能干什么?
\item 怎样学习Linux?
\end{itemize}
\item Shell基础
\begin{itemize}
\item UNIX文件系统
\item 路径,目录,特殊文件
\item 基本 shell 命令
\item Shell 编程
\end{itemize}
\item 常用软件工具
\begin{itemize}
\item 编辑器: emacs, vim
\item 网络工具: firefox, lftp, wget, mutt, pidgin \ldots{}
\item 办公自动化: openoffice.org
\item 图像处理: gimp, imagemagick, dia, xfig, inkscape, hugin
\end{itemize}
\item 软件开发环境
\begin{itemize}
\item GCC
\item make
\item gdb
\item 可视化编程: Qt4, glade, gambas, Tcl/Tk
\end{itemize}
\item Debian GNU/Linux 系统管理
\begin{itemize}
\item 最小系统的安装
\item apt
\item Debian 管理工具
\end{itemize}
\end{enumerate}
\subsection{实验内容}

参见第\ref{sec:lab}节《实验教学大纲》。

\subsection{实习}
无
\subsection{考核}

\begin{itemize}
\item 考试: 50\%
\item 作业: 50\%
\end{itemize}

\subsection{参考教材}

\nocite{cooper10bash,web:debianhandbook}
\printbibliography[heading=none]{}


\section{课程说明}

\subsection{课程性质和要求}

目前《操作系统原理》课程都是以开源的Linux为范本进行教学。因此学生必须要有一定的Linux应用基
础能力。本课程介绍给同学如下内容:
\begin{itemize}
\item GNU/Linux的过去、现在、和未来
\item Bash
\item Linux下的软件开发环境
\item Linux系统管理和网络管理
\end{itemize}

\subsection{课程重点}

\begin{itemize}
\item Shell命令行
\item 软件工具,开发环境
\item Linux下的C编程
\end{itemize}

\subsection{作业、实习要求}
作业迟交一天扣分10\%。

\subsection{与其它课程的关系}

\begin{itemize}
\item 前期课程:大学计算机基础
\item 后期课程:网络课程,编程课程,操作系统课程
\end{itemize}

\subsection{课时安排}

\begin{center}
  \begin{tabular}{lrr}
    \hline
    课程内容              &  理论学时  &  实验学时  \\
    \hline
    Shell 基础            &        4  &         4  \\
    常用软件工具          &         4  &         4  \\
    软件开发环境          &         4  &         4  \\
    Debian系统管理        &         4  &         4  \\
    \hline
  \end{tabular}
\end{center}

\subsection{特殊说明}
本课程以应用为主,最好全部授课安排在机房进行

%\Sign{}

\section{实验教学大纲}
\label{sec:lab}

\begin{itemize}
\item 课程编号: 41100165
\item 学时: 32 (理论: 16; 实验: 16)
\item 学分: 2
\item 实习: 0
\item 授课对象: 计算机科学与技术,电子信息工程,信息与计算机技术
\end{itemize}

\subsection{实验教学的目的和要求}
本课程的目的就是让学生熟悉Linux下的工作环境和开发环境,为后续课程打下坚实的基础。

\subsection{实践教学大纲}

\begin{center}
  \begin{tabular}{lr}
    \hline
    实验安排        &  学时  \\
    \hline
    shell基础       &    4  \\
    常用软件工具    &     4  \\
    软件开发环境    &     4  \\
    Debian系统管理  &     4  \\
    \hline
  \end{tabular}
\end{center}

\subsection{实验设备要求}

\begin{itemize}
\item Debian PC
\end{itemize}

\subsection{实验内容}

\begin{enumerate}
\item Shell基础
  \begin{itemize}
  \item UNIX文件系统
  \item 路径,目录,特殊文件
  \item 基本 shell 命令
  \item Shell 编程
  \end{itemize}
\item 常用软件工具
  \begin{itemize}
  \item 编辑器: emacs, vim
  \item 网络工具: firefox, lftp, wget, mutt, pidgin \ldots{}
  \item 办公自动化: openoffice.org
  \item 图像处理: gimp, imagemagick, dia, xfig, inkscape, hugin
  \end{itemize}
\item 软件开发环境
  \begin{itemize}
  \item GCC
  \item make
  \item gdb
  \item 可视化编程: Qt4, glade, gambas, Tcl/Tk
  \end{itemize}
\item Debian GNU/Linux 系统管理
  \begin{itemize}
  \item 最小系统的安装
  \item apt
  \item Debian 管理工具
  \end{itemize}
\end{enumerate}

\subsection{实验报告要求}
按规定格式完成,延误一天扣分10\%。

\subsection{成绩考核}

\begin{itemize}
\item 实验报告满分100,60分及格
\end{itemize}

\subsection{实验指导和参考书目}

\nocite{cooper10bash,web:debianhandbook}
\printbibliography[heading=none]{}


\subsection{特别说明}
本课程以应用为主,最好全部授课安排在机房进行

%\Sign{}

\section{课程简介}

\begin{itemize}
\item 课程编号: 41100165
\item 学时: 32 (理论: 16; 实验: 16)
\item 学分: 2
\item 实习: 0
\item 面向专业: 计算机科学与技术,电子信息工程,信息与计算机技术
\item 前期课程:英语,大学计算机基础
\item 课程性质和要求目前《操作系统原理》课程都是以开源的Linux为范本进行教学。因此学生必须要
  有一定的Linux应用基础能力。本课程介绍给同学如下内容:
  \begin{itemize}
  \item GNU/Linux的过去、现在、和未来
  \item Bash
  \item Linux下的软件开发环境
  \item Linux系统管理和网络管理
  \end{itemize}
\item 课程重点
  \begin{itemize}
  \item Shell命令行
  \item 软件工具,开发环境
  \item Linux下的C编程
  \end{itemize}
\item 参考教材\hfill \nocite{cooper10bash,web:debianhandbook}
  \printbibliography[heading=none]{}
\end{itemize}

\end{document}
%%% Local Variables:
%%% mode: latex
%%% TeX-master: t
%%% End:
