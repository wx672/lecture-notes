% Created 2024-06-30 Sun 16:02
% Intended LaTeX compiler: lualatex
\documentclass{swfulabreport}

\usepackage{hyperref}
\usepackage{amsmath,amsfonts,amssymb}
\author{张三}
\date{\today}
\title{《计算机网络》课程实习}
\hypersetup{
 pdfauthor={张三},
 pdftitle={《计算机网络》课程实习},
 pdfkeywords={},
 pdfsubject={},
 pdfcreator={Emacs 29.3 (Org mode 9.6.15)}, 
 pdflang={Cn}}
\begin{document}

\maketitle

\section{Packet analysis}
\label{sec:orgb13b384}
Upon running the following command:

\begin{minted}[mathescape=true,linenos=true,numbersep=5pt,frame=lines,framesep=2mm]{sh}
sudo tcpdump -ilo -nnvvvxXKS -s0 port 3333
\end{minted}

the following packet is captured:

\begin{verbatim}
08:34:10.790666 IP (tos 0x0, ttl 64, id 12824, offset 0, flags [DF], proto TCP (6),
length 64)
    127.0.0.1.46668 > 127.0.0.1.3333: Flags [P.], seq 2400005725:2400005737, ack 373279396,
    win 512, options [nop,nop,TS val 3259949783 ecr 3259896343], length 12
        0x0000:  4500 0040 3218 4000 4006 0a9e 7f00 0001  E..@2.@.@.......
        0x0010:  7f00 0001 b64c 0d05 8f0d 2e5d 163f caa4  .....L.....].?..
        0x0020:  8018 0200 fe34 0000 0101 080a c24e e2d7  .....4.......N..
        0x0030:  c24e 1217 6865 6c6c 6f20 776f 726c 640a  .N..hello.world.
\end{verbatim}

\begin{enumerate}
\item Tell me the meaning of each option used in the previous command.
\begin{itemize}
\item \textbf{-i}:
\item \textbf{-nn}:
\item \textbf{-vvv}:
\item \textbf{-x}:
\item \textbf{-X}:
\item \textbf{-S}:
\item \textbf{-K}:
\item \textbf{-s0}:
\end{itemize}

\item Please analyze this captured packet and explain it to me as detailed as you can.     
\begin{description}
\item[{Answer}] 
\end{description}
\end{enumerate}

\section{HTTP}
\label{sec:org060c569}
\begin{enumerate}
\item Write a simple script showing how HTTP works (you need \texttt{curl});
\begin{minted}[mathescape=true,linenos=true,numbersep=5pt,frame=lines,framesep=2mm]{sh}
#!/bin/bash

\end{minted}

\item Record your HTTP demo session with \texttt{ttyrec}.
\end{enumerate}

\section{Socket programming}
\label{sec:org2570ebe}

\subsection{TCP}
\label{sec:orgaf956c8}

\begin{minted}[mathescape=true,linenos=true,numbersep=5pt,frame=lines,framesep=2mm]{c}
/* A simple TCP server written in C */

// Your code
\end{minted}

\begin{minted}[mathescape=true,linenos=true,numbersep=5pt,frame=lines,framesep=2mm]{c}
/* A simple TCP client written in C */

// Your code
\end{minted}

\subsection{UDP}
\label{sec:orgeab4237}

\begin{minted}[mathescape=true,linenos=true,numbersep=5pt,frame=lines,framesep=2mm]{c}
/* A simple UDP server written in C */

// Your code
\end{minted}

\begin{minted}[mathescape=true,linenos=true,numbersep=5pt,frame=lines,framesep=2mm]{c}
/* A simple UDP client written in C */

// Your code
\end{minted}

\section{Questions}
\label{sec:orgbf4dc21}
List at least 5 problems you've met while doing this work. When
listing your problems, you have to tell me:
\begin{enumerate}
\item Description of this problem. For example,
\begin{itemize}
\item What were you trying to do before seeing this problem?
\end{itemize}
\item How did you try solving this problem? For example,
\begin{itemize}
\item Did you google? web links?
\item Did you read the man page?
\item Did you ask others for hints?
\end{itemize}
\end{enumerate}

\subsection{Problems}
\label{sec:org81554a6}

\subsubsection{Problem 1}
\label{sec:org7f96be3}

\subsubsection{Problem 2}
\label{sec:org34f7e1c}

\subsubsection{Problem 3}
\label{sec:orgf7e9237}

\subsubsection{Problem 4}
\label{sec:org7b84558}

\subsubsection{Problem 5}
\label{sec:orgaed36c2}
\end{document}

%%% Local Variables:
%%% mode: LaTeX
%%% TeX-master: t
%%% End:
