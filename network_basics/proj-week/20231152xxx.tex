\documentclass{swfulabreport}

\swfusetup{%
  Title     ={Hacking with Linux networking CLI tools}, % 课程名称
  Author    ={Who Am I}, % 
  ID        ={20231152xxx}, % 学号
  Class     ={计科2023级},%
  Tutor     ={王晓林}, %指导教师
  Year      ={2024},%
  Month     ={12},%
  Date      ={18},%
  Reviewer  ={王晓林},%
  ReviewDate={2024/12/19},%
  Mark      =,%
  LabDate   ={2024/12/16 -- 2024/12/17},%
  LabDays   ={2},%
  Lab       ={生物楼219},%经管楼219
  Heads     ={120},%
  Grouping  ={不分组},%
  Why   ={掌握Linux平台上常用网络工具的使用,并理解基本的网络编程过程。

\begin{itemize}
\tightlist
\item
  掌握用tcpdump捕获并分析数据包
\item
  掌握用netcat完成简单的网络会话
\item
  掌握基本的网络编程
\end{itemize}},%实验目的
  Req   ={\begin{itemize}
\tightlist
\item
  在Linux平台完成所有实验
\item
  在Linux平台完成实验报告
\item
  努力尝试用英文撰写实验报告
\item
  将实验作业及报告以tgz格式打包,并上传到指定教学网站
\item
  迟交报告将被扣分
\end{itemize}},%实验要求
  What  ={详见\href{https://cs6.swfu.edu.cn/~wx672/lecture_notes/network_basics/proj-week/proj-week.html}{《实验指导书》}。},%实验内容
  Sched ={\begin{itemize}
\tightlist
\item
  2024/12/16: 学习使用tmux,ttyrec等实验所需的命令行工具。 学习使用ip,
  tcpdump, netcat, curl, ss, nmap等网络工具。
\item
  2024/12/17: 完成用tcpdump捕获、分析数据包,用netcat实现网络协议会话,
  以及网络编程等规定实验项目。完成所有实验,并撰写实验报告。
\end{itemize}},%课程实习安排
  Sumup ={不要少于300字。},%实习总结
}

\begin{document}

\maketitle

\section{Packet analysis}\label{packet-analysis}

\begin{minted}[autogobble]{sh}
sudo tcpdump -ilo -nnvvvxXKS -s0 port 3333
\end{minted}

Upon running the above command, the following packet is captured:

\begin{minted}[autogobble]{text}
08:34:10.790666 IP (tos 0x0, ttl 64, id 12824, offset 0, flags [DF],
proto TCP (6), length 64)

127.0.0.1.46668 > 127.0.0.1.3333: Flags [P.], seq 2400005725:2400005737,
ack 373279396, win 512, options [nop,nop,...], length 12

  0x0000:  4500 0040 3218 4000 4006 0a9e 7f00 0001  E..@2.@.@.......
  0x0010:  7f00 0001 b64c 0d05 8f0d 2e5d 163f caa4  .....L.....].?..
  0x0020:  8018 0200 fe34 0000 0101 080a c24e e2d7  .....4.......N..
  0x0030:  c24e 1217 6865 6c6c 6f20 776f 726c 640a  .N..hello.world.
\end{minted}

\begin{enumerate}
\def\labelenumi{\arabic{enumi}.}
\item
  Tell me the meaning of each option used in the previous command.

  \begin{itemize}
  \tightlist
  \item
    \textbf{-i}:
  \item
    \textbf{-nn}:
  \item
    \textbf{-vvv}:
  \item
    \textbf{-x}:
  \item
    \textbf{-X}:
  \item
    \textbf{-S}:
  \item
    \textbf{-K}:
  \item
    \textbf{-s0}:
  \end{itemize}
\item
  Please analyze this captured packet and explain it to me as detailed
  as you can.

  \begin{itemize}
  \tightlist
  \item
    \textbf{Answer:}
  \end{itemize}
\end{enumerate}

\section{HTTP}\label{http}

\begin{enumerate}
\def\labelenumi{\arabic{enumi}.}
\tightlist
\item
  Write a simple script showing how HTTP works (you need
  \mintinline[]{text}{curl}).
\end{enumerate}

\begin{minted}[autogobble]{sh}
#!/bin/bash
\end{minted}

\begin{enumerate}
\def\labelenumi{\arabic{enumi}.}
\setcounter{enumi}{1}
\tightlist
\item
  Record your HTTP demo session with \mintinline[]{text}{ttyrec}.
\end{enumerate}

\section{Socket programming}\label{socket-programming}

\subsection{TCP}\label{tcp}

\begin{minted}[autogobble]{c}
  /* A simple TCP server written in C */

  // Your code
\end{minted}

\begin{minted}[autogobble]{c}
  /* A simple TCP client written in C */

  // Your code
\end{minted}

\subsection{UDP}\label{udp}

\begin{minted}[autogobble]{c}
  /* A simple UDP server written in C */

  // Your code
\end{minted}

\begin{minted}[autogobble]{c}
  /* A simple UDP client written in C */

  // Your code
\end{minted}

\section{Questions}\label{questions}

List at least 5 problems you've met while doing this work. When listing
your problems, you have to tell me:

\begin{enumerate}
\def\labelenumi{\arabic{enumi}.}
\item
  Description of this problem. For example,

  \begin{itemize}
  \tightlist
  \item
    What were you trying to do before seeing this problem?
  \end{itemize}
\item
  How did you try solving this problem? For example,

  \begin{itemize}
  \tightlist
  \item
    Did you google? web links?
  \item
    Did you read the man page?
  \item
    Did you ask others for hints?
  \end{itemize}
\end{enumerate}

\subsection{Problem 1}\label{problem-1}

\subsection{Problem 2}\label{problem-2}

\subsection{Problem 3}\label{problem-3}

\subsection{Problem 4}\label{problem-4}

\subsection{Problem 5}\label{problem-5}

\end{document}
