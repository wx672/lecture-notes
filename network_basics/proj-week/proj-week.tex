% Created 2021-12-31 Fri 10:55
% Intended LaTeX compiler: pdflatex
\documentclass{article} [NO-DEFAULT-PACKAGES] \usepackage{wx672hyperref}
\usepackage{amsmath,amsfonts,amssymb}
\usepackage{graphicx}
\usepackage{tabularray}
\usepackage{wx672minted}
\pagestyle{plain}
\author{Xiaolin Wang}
\date{\today}
\title{Hacking with Linux networking command line tools}
\hypersetup{
 pdfauthor={Xiaolin Wang},
 pdftitle={Hacking with Linux networking command line tools},
 pdfkeywords={ssh, tmux, http, ftp, smtp, iptables, network, networking},
 pdfsubject={Network lab tutorial week},
 pdfcreator={Emacs 27.1 (Org mode 9.4.6)}, 
 pdflang={English}}
\begin{document}

\maketitle
\tableofcontents


\section{Caution}
\label{sec:orgbcfde5a}
\begin{itemize}
\item You must submit your report as a \texttt{tar ball} in which the following files should be
included:

\begin{enumerate}
\item Your report in either \texttt{Emacs Org} or \texttt{Markdown} format, and a HTML file
generated from your \texttt{org} or \texttt{md} file.

Tips: 
\begin{itemize}
\item In Emacs, press \texttt{C-c C-e h h} to export HTML file from your org file;

\item For \texttt{Markdown} to HTML, you can try \texttt{markdown}, \texttt{pandoc}, \texttt{cmark}, whatever.

\item This page itself is generated from an \href{proj-week.org}{org file (proj-week.org)}. You can take it
as an example.

\item Report template: \href{20221159xxx.org}{\texttt{org file}}, \href{20221159xxx.html}{\texttt{html file}}, \href{20221159xxx.md}{\texttt{markdown file}}
\end{itemize}

\item your bash script for a HTTP demostration;

\item a \texttt{ttyrec} file recording your operations (\texttt{man ttyrec});
\end{enumerate}

\noindent\rule{\textwidth}{0.5pt}
\begin{enumerate}
\item Here's how :: 
\begin{enumerate}
\item make a directory, e.g. 20221159xxx. In this directory, try very hard to make all
the files available.

\begin{minted}[mathescape=true,linenos=true,numbersep=5pt,frame=lines,framesep=2mm]{sh}
mkdir  20221159xxx
cd 20221159xxx
emacsclient tmux-http.sh # write your script
emacsclient 20221159xxx.org # this is your report
vim 20221159xxx.md # in markdown format
ttyrec http-demo.ttyrec # make your demo screencast
\end{minted}

\item make a tar ball.
\begin{verbatim}
cd ..
tar zcf 20221159xxx.tgz 20221159xxx
ls -l # make sure your tar ball is smaller than 1MB in size         
\end{verbatim}

\item upload the \texttt{tgz} file to our \href{https://cs6.swfu.edu.cn/moodle/mod/assign/view.php?id=536}{moodle site}.
\end{enumerate}
\end{enumerate}

\noindent\rule{\textwidth}{0.5pt}
\end{itemize}

\begin{itemize}
\item Here is a short tutorial about writing lab report: \href{tutorial.ttyrec}{\texttt{tutorial.ttyrec}}. To view it:

\begin{verbatim}
ttyplay  tutorial.ttyrec
\end{verbatim}


Feel free to make your own \texttt{ttyrec} file while doing this lab work. For example:

\begin{verbatim}
ttyrec  20221159xxx-http.ttyrec
ttyrec  20221159xxx-email.ttyrec
ttyrec  20221159xxx-ftp.ttyrec
\end{verbatim}

\item \textbf{Deadline:} \textit{<2021-10-31 Sun> } 

\begin{itemize}
\item Submit your report as a \texttt{tgz} file \href{https://cs6.swfu.edu.cn/moodle/mod/assign/view.php?id=536}{here}. In your \texttt{tgz} file, there must be:
\begin{enumerate}
\item your report in \texttt{org} or \texttt{markdown} format
\item your report in HTML format
\item your bash script for demostrating a HTTP session
\item one or more \texttt{ttyrec} files for demostrating whatever you did
\end{enumerate}

\item Late reports will be penalized 20\% per day.
\end{itemize}

\item MS-word file will \textbf{NOT} be accepted. Cheating will result in automatic failure of this
work.
\end{itemize}

\section{\texttt{tmux, nc, ip, tcpdump, ss, nmap, curl}}
\label{sec:orga451444}

Here are the bash scripts I used in the class for demostrating how some protocols work.

\begin{itemize}
\item \href{https://cs6.swfu.edu.cn/\~wx672/lecture\_notes/network\_basics/scripts/tmux-demo-3way.handshake.sh}{TCP three-way handshake}
\item \href{https://cs6.swfu.edu.cn/\~wx672/lecture\_notes/network\_basics/scripts/tmux-demo-udp.sh}{UDP}
\item \href{https://cs6.swfu.edu.cn/\~wx672/lecture\_notes/network\_basics/scripts/tmux-demo-smtp.sh}{SMTP} (need a SMTP server)
\item \href{https://cs6.swfu.edu.cn/\~wx672/lecture\_notes/network\_basics/scripts/tmux-demo-ftp.sh}{FTP} (need a FTP server)
\end{itemize}

\noindent\rule{\textwidth}{0.5pt}

\begin{description}
\item[{Your tasks}] (Deadline: \textit{<2021-10-31 Sun>})

\begin{enumerate}
\item Run the above scripts to get familar with these tools, and get a thorough understanding about these protocols;

\item Packet analysis. Upon running the following command:

\begin{verbatim}
sudo tcpdump -ilo -nnvvvxXKS -s0 port 3333
\end{verbatim}


the following packet is captured:

\begin{verbatim}
08:34:10.790666 IP (tos 0x0, ttl 64, id 12824, offset 0, flags [DF],
proto TCP (6), length 64)
127.0.0.1.46668 > 127.0.0.1.3333: Flags [P.], seq 2400005725:2400005737,
ack 373279396, win 512, options [nop,nop,TS val 3259949783 ecr 3259896343],
length 12
    0x0000:  4500 0040 3218 4000 4006 0a9e 7f00 0001  E..@2.@.@.......
    0x0010:  7f00 0001 b64c 0d05 8f0d 2e5d 163f caa4  .....L.....].?..
    0x0020:  8018 0200 fe34 0000 0101 080a c24e e2d7  .....4.......N..
    0x0030:  c24e 1217 6865 6c6c 6f20 776f 726c 640a  .N..hello.world.
\end{verbatim}

\begin{enumerate}
\item Tell me the meaning of each option used in the previous command;

\item Please analyze this captured packet and explain it to me as detailed as you can.
\end{enumerate}

\item Write a similar script showing how HTTP works (you need \texttt{curl});

\item Record your HTTP demo session with \texttt{ttyrec}.
\end{enumerate}
\end{description}
\end{document}