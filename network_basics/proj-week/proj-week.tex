% Created 2024-06-30 Sun 16:07
% Intended LaTeX compiler: lualatex
\documentclass{article}
\usepackage{wx672babel}
\usepackage{hyperref}
\usepackage{amsmath,amsfonts,amssymb}
\usepackage{wx672minted}
\pagestyle{plain}
\author{Xiaolin Wang}
\date{\today}
\title{Hacking with Linux networking command line tools}
\hypersetup{
 pdfauthor={Xiaolin Wang},
 pdftitle={Hacking with Linux networking command line tools},
 pdfkeywords={ssh, tmux, http, ftp, smtp, iptables, network, networking},
 pdfsubject={Network lab tutorial week},
 pdfcreator={Emacs 29.3 (Org mode 9.6.15)}, 
 pdflang={English}}
\begin{document}

\maketitle
\tableofcontents


\section{Caution}
\label{sec:org78dbaf4}
You must submit your report as a \texttt{tar ball} in which the following files
should be included in:

\begin{itemize}
\item Your report in either \texttt{Emacs Org} or \texttt{Markdown} format, and a PDF file
generated from your \texttt{org} or \texttt{md} file.

Tips: 
\begin{itemize}
\item In Emacs, press \texttt{C-c C-e l p} to export a PDF file from your org file.

\item For \texttt{Markdown} to PDF, you can try \texttt{markdown}, \texttt{pandoc}, \texttt{cmark},
whatever. For example:

\begin{minted}[mathescape=true,linenos=true,numbersep=5pt,frame=lines,framesep=2mm]{shell}
pandoc input.md -o output.pdf --pdf-engine=lualatex
\end{minted}

\item This HTML page itself is generated from an \href{proj-week.org}{org file
(proj-week.org)}. You can take it as an example.

\item Report template: \href{20231152xxx.org}{\texttt{org file}}, \href{20231152xxx.pdf}{\texttt{html file}}, \href{20231152xxx.md}{\texttt{markdown file}}.
\end{itemize}

\item your source codes (\href{https://cs6.swfu.edu.cn/\~wx672/lecture\_notes/network\_basics/scripts/}{examples}).

\item a screencast file in \texttt{ttyrec} format recording your operations (\texttt{man ttyrec}).

Here's how:

\noindent\rule{\textwidth}{0.5pt}
\begin{enumerate}
\item make a directory, e.g. 20231152xxx, in this dir try very hard to make all
the files available.

\begin{minted}[mathescape=true,linenos=true,numbersep=5pt,frame=lines,framesep=2mm]{sh}
mkdir 20231152xxx       # create a new directory
cd 20231152xxx
vim tmux-http.sh        # write your script
vim tcpServer.c         # Implement the TCP server in C
vim tcpClient.c         # Implement the TCP client in C
vim 20231152xxx.md      # write your report in markdown format, or
vim 20231152xxx.org     # in org format
ttyrec http-demo.ttyrec # make your demo screencast
\end{minted}

\item make a tar ball.
\begin{minted}[mathescape=true,linenos=true,numbersep=5pt,frame=lines,framesep=2mm]{sh}
cd ..                                                      
tar zcf 20231152xxx.tgz 20231152xxx                        
ls -l # make sure your tar ball is smaller than 1MB in size
\end{minted}

\item submit the \texttt{tgz} file to our \href{https://cs6.swfu.edu.cn/moodle/mod/assign/view.php?id=760}{moodle site}.
\end{enumerate}

\noindent\rule{\textwidth}{0.5pt}
\end{itemize}

Here is a short tutorial about writing lab report: \href{tutorial.ttyrec}{\texttt{tutorial.ttyrec}}. To view it:

\begin{minted}[mathescape=true,linenos=true,numbersep=5pt,frame=lines,framesep=2mm]{sh}
ttyplay  tutorial.ttyrec
\end{minted}

Feel free to make your own \texttt{ttyrec} files while doing this lab work. For example:

\begin{verbatim}
ttyrec  20231152xxx-http.ttyrec
ttyrec  20231152xxx-email.ttyrec
ttyrec  20231152xxx-ftp.ttyrec
\end{verbatim}

\begin{description}
\item[{\textbf{Bonus point}}] Manage your project with \texttt{git}. \texttt{man gittutorial} to
learn the very basics of it.
\end{description}

\subsection{Deadline: \textit{<2024-07-07 Sun>}}
\label{sec:org4f38793}

\begin{itemize}
\item Submit your report as a \texttt{tgz} file \href{https://cs6.swfu.edu.cn/moodle/mod/assign/view.php?id=777}{here}. In your \texttt{tgz} file, there
must be:

\begin{itemize}
\item your report in \texttt{org} or \texttt{markdown} format.
\item your report in PDF format.
\item your bash script for demostrating a HTTP session.
\item one or more \texttt{ttyrec} files for demostrating whatever you did.
\end{itemize}

\item Late reports will be penalized 20\% per day.

\item MS-word file will \textbf{NOT} be accepted. Cheating will result in automatic failure of this work.
\end{itemize}

\section{\texttt{tmux, nc, ip, tcpdump, ss, nmap, curl}}
\label{sec:orgef2e17f}

Here are the bash scripts I used in the class for demostrating how some protocols work.

\begin{itemize}
\item \href{https://cs6.swfu.edu.cn/\~wx672/lecture\_notes/network\_basics/scripts/tmux-demo-3way.handshake.sh}{TCP three-way handshake}
\item \href{https://cs6.swfu.edu.cn/\~wx672/lecture\_notes/network\_basics/scripts/tmux-demo-udp.sh}{UDP}
\item \href{https://cs6.swfu.edu.cn/\~wx672/lecture\_notes/network\_basics/scripts/tmux-demo-smtp.sh}{SMTP} (need a SMTP server)
\item \href{https://cs6.swfu.edu.cn/\~wx672/lecture\_notes/network\_basics/scripts/tmux-demo-ftp.sh}{FTP} (need a FTP server)
\end{itemize}

\noindent\rule{\textwidth}{0.5pt}

\subsection{Your tasks}
\label{sec:org2619ca1}

\begin{enumerate}
\item Run the above scripts to get familar with these tools, and get a thorough understanding about these protocols;

\item Packet analysis. Upon running the following command:

\begin{minted}[mathescape=true,linenos=true,numbersep=5pt,frame=lines,framesep=2mm]{sh}
sudo tcpdump -ilo -nnvvvxXKS -s0 port 3333
\end{minted}

the following packet is captured:

\begin{verbatim}
08:34:10.790666 IP (tos 0x0, ttl 64, id 12824, offset 0, flags [DF],
proto TCP (6), length 64)
127.0.0.1.46668 > 127.0.0.1.3333: Flags [P.], seq 2400005725:2400005737,
ack 373279396, win 512, options [nop,nop,TS val 3259949783 ecr 3259896343],
length 12
    0x0000:  4500 0040 3218 4000 4006 0a9e 7f00 0001  E..@2.@.@.......
    0x0010:  7f00 0001 b64c 0d05 8f0d 2e5d 163f caa4  .....L.....].?..
    0x0020:  8018 0200 fe34 0000 0101 080a c24e e2d7  .....4.......N..
    0x0030:  c24e 1217 6865 6c6c 6f20 776f 726c 640a  .N..hello.world.
\end{verbatim}

\begin{enumerate}
\item Tell me the meaning of each option used in the previous command.

\item Please analyze this captured packet and explain it to me as detailed as you can.
\end{enumerate}

\item Write a similar script showing how HTTP works (you need \texttt{curl}).

\item Record your HTTP demo session with \texttt{ttyrec}.
\end{enumerate}

\section{Socket programming}
\label{sec:org50b25e8}

The followings are the \href{https://cs6.swfu.edu.cn/\~wx672/lecture\_notes/network\_basics/src/}{Python programs} I used in the class for demostrating
socket programming. Your tasks

\begin{enumerate}
\item Try these programs with a remote server IP instead of 127.0.0.1.
\item Rewrite them in C.
\end{enumerate}

\subsection{TCP}
\label{sec:org111600d}

\subsubsection{A simple TCP server written in Python3}
\label{sec:org5d09a7f}

\begin{minted}[mathescape=true,linenos=true,numbersep=5pt,frame=lines,framesep=2mm]{python}
#!/usr/bin/python3

### A simple TCP server ###

from socket import *
serverPort = 12000
serverSocket = socket(AF_INET,SOCK_STREAM)
serverSocket.bind(('',serverPort))
serverSocket.listen(0)
print(serverSocket.getsockname())
print('The server is ready to receive')
while 1:
    connectionSocket, addr = serverSocket.accept()
    print(connectionSocket.getsockname())
    sentence = connectionSocket.recv(1024)
    capitalizedSentence = sentence.upper()
    connectionSocket.send(capitalizedSentence)
    connectionSocket.close()
\end{minted}

\subsubsection{A simple TCP client written in Python3}
\label{sec:orgb17ff84}

\begin{minted}[mathescape=true,linenos=true,numbersep=5pt,frame=lines,framesep=2mm]{python}
#!/usr/bin/python3

### A simple TCP client ###

from time import *
from socket import *
serverName = '127.0.0.1'
serverPort = 12000
clientSocket = socket(AF_INET, SOCK_STREAM)
clientSocket.connect((serverName,serverPort))
print(clientSocket.getsockname())
sentence = input('Input lowercase sentence:')
clientSocket.send(bytes(sentence,'utf-8'))
modifiedSentence = clientSocket.recv(1024)
print('From Server:', str(modifiedSentence,'utf-8'))
clientSocket.close()
\end{minted}

\subsubsection{A simple TCP demo script}
\label{sec:orgfc84ea5}

\begin{minted}[mathescape=true,linenos=true,numbersep=5pt,frame=lines,framesep=2mm]{sh}
#!/bin/bash

### A simple TCP demo script ###

set -euC

tmux rename-window "TCP demo"

#    Window setup
# +--------+--------+
# | server | client |
# +--------+--------+
# |      watch      |
# +-----------------+
# |     tcpdump     |
# +-----------------+
#
tmux split-window -h
tmux split-window -fl99
tmux split-window -l12

tmux send-keys -t{top-left} "./tcpServer.py" 

tmux send-keys -t{top-right} "./tcpClient.py"

tmux send-keys -t{up-of} "watch -tn.1 'ss -ant \"( sport == 12000 or dport == 12000 )\"'" C-m

tmux send-keys "sudo tcpdump -ilo -vvvnnxXSK -s0 port 12000" C-m
\end{minted}

\subsection{UDP}
\label{sec:orga187657}

\subsubsection{A simple UDP server written in Python3}
\label{sec:org05f4664}

\begin{minted}[mathescape=true,linenos=true,numbersep=5pt,frame=lines,framesep=2mm]{python}
#!/usr/bin/python3

### A simple UDP server ###

from socket import *
serverPort = 12000
serverSocket = socket(AF_INET, SOCK_DGRAM)
serverSocket.bind(('', serverPort))
print("The server is ready to receive")
while 1:
    message, clientAddress = serverSocket.recvfrom(2048)
    modifiedMessage = message.upper()
    serverSocket.sendto(modifiedMessage, clientAddress)
\end{minted}

\subsubsection{A simple UDP client written in Python3}
\label{sec:org2fe842a}

\begin{minted}[mathescape=true,linenos=true,numbersep=5pt,frame=lines,framesep=2mm]{python}
#!/usr/bin/python3

### A simple UDP client ###

from socket import *
serverName = '127.0.0.1'
serverPort = 12000
clientSocket = socket(AF_INET, SOCK_DGRAM)
message = input('Input lowercase sentence:')
clientSocket.sendto(bytes(message,'utf-8'),(serverName, serverPort))
modifiedMessage, serverAddress = clientSocket.recvfrom(2048)
print(str(modifiedMessage,'utf-8'))
clientSocket.close()
\end{minted}

\subsubsection{A simple UDP demo script}
\label{sec:org235522c}

\begin{minted}[mathescape=true,linenos=true,numbersep=5pt,frame=lines,framesep=2mm]{sh}
#!/bin/bash

### A simple UDP demo script ###

set -euC

tmux rename-window "UDP demo"

#    Window setup
# +--------+--------+
# | server | client |
# +--------+--------+
# |     tcpdump     |
# +-----------------+
#
tmux split-window -h
tmux split-window -fl99

tmux send-keys -t{top-left}  "./udpServer.py" 
tmux send-keys -t{top-right} "./udpClient.py"

tmux send-keys "sudo tcpdump -ilo -vvvnnxXK port 12000" C-m
\end{minted}
\end{document}
