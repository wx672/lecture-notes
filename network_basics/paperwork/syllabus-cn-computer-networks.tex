% Created 2014-09-18 Thu 20:37
\documentclass{wx672article}

\usepackage{wx672ctex, wx672bib}
\addbibresource{net.bib}

\title{《计算机网络》课程大纲}
\author{王晓林}
\date{2015-03-06}

\begin{document}

\maketitle
\tableofcontents
\clearpage

\begin{itemize}
\item 课程编号: 41100009
\item 学时: 32 (理论: 24; 实验: 8)
\item 学分: 3
\item 实习: 0 
\item 面向专业: 计算机科学与技术,电子信息工程,信息与计算机技术,信息工程
\end{itemize}

\section{课程大纲}
\label{sec-1}

\subsection{课程内容}
\label{sec-1-1}

\begin{enumerate}
\item 网络简介 (发展历史,定义,分类,拓扑,应用)
\item 链路层(以太网,ARP,CSMA/CD)
\item 网络层(路由的基本过程,IP地址,IPv6)
\item 传输层(TCP,UDP)
\item 应用层(HTTP,DNS,FTP,SMTP)
\end{enumerate}

\subsection{实验内容}
\label{sec-1-2}


\begin{enumerate}
\item 基本Linux网络命令(ip, ping, tcpdump, nmap, nc, iptables)
\item Packettracer网络模拟
\end{enumerate}

\subsection{实习}
\label{sec-1-3}

无

\subsection{考核}
\label{sec-1-4}

\begin{itemize}
\item 考试: 80\%
\item 作业: 20\%
\end{itemize}

\subsection{参考教材}
\label{sec-1-5}

\nocite{tanenbaum2011computer,fall2011tcp,kurose2013computer,bautts2005linux,hunt2002tcp,hall2009beej}
\printbibliography[heading=none]{}

\section{课程说明}
\subsection{课程性质和要求}
\label{sec-1}

《计算机网络》是一门重要的专业基础课,向学生介绍TCP/IP相关的网络基本概念。通过学习本课程,
学生应该能清晰地了解如下内容:
\begin{itemize}
\item 数据通信的基本常识
\item 网络架构
\item ISO/OSI参考模型
\item TCP/IP协议族
\item 网络管理
\item 网络安全
\end{itemize}

\subsection{课程重点}
\label{sec-2-2}

\begin{itemize}
\item TCP/IP协议栈
\item 以太网工作的过程
\item 路由的基本过程
\item TCP连接的工作原理
\end{itemize}

\subsection{作业、实习要求}

作业迟交一天扣分10\%。

\subsection{与其它课程的关系}

\begin{itemize}
\item 前期课程:Linux应用基础(非必需)
\item 后期课程:路由原理,网络管理
\end{itemize}

\subsection{课时安排}
\label{sec-2-5}

\begin{center}
  \begin{tabular}{lrr}
    \hline
    课程内容 & 理论学时 & 实验学时\\
    \hline
    简介 & 2 & 0\\
    链路层 & 4 & 0\\
    网络层 & 6 & 2\\
    传输层 & 6 & 4\\
    应用层 & 6 & 2\\
    \hline
  \end{tabular}
\end{center}

\subsection{特殊说明}
\label{sec-2-6}

无

\subsection{实践教学大纲}

\begin{center}
  \begin{tabular}{lr}
    \hline
    实验安排 & 学时\\\hline    
    Linux网络命令&4\\
    Packettracer网络模拟&4\\\hline
  \end{tabular}
\end{center}

\subsection{实验设备要求}
\label{sec-3-3}

\begin{itemize}
\item Debian/Ubuntu PC 装有必要软件
\end{itemize}

\subsection{实验内容}
\label{sec-3-4}

\begin{itemize}
\item 参见\href{http://cs2.swfu.edu.cn/~wx672/lecture_notes/network_basics/net-tools/}{《计
    算机网络实验指导》}。
\end{itemize}

\subsection{实验报告要求}
\label{sec-3-5}

按规定格式完成,迟交报告每天扣分10\%。

\subsection{成绩考核}
\label{sec-3-6}

\begin{itemize}
\item 实验报告满分100,60分及格
\end{itemize}

\subsection{实验指导和参考书目}
\label{sec-3-7}

\begin{itemize}
\item \href{http://cs2.swfu.edu.cn/~wx672/lecture_notes/network_basics/net-tools/}{《计算机
    网络实验指导》}
\end{itemize}

\subsection{特别说明}
\label{sec-3-8}

无

\section{课程简介}
\label{sec-4}

\begin{itemize}
\item 课程编号: 41100009
\item 学时: 32 (理论: 24; 实验: 8)
\item 学分: 3
\item 实习: 0
\item 面向专业: 计算机科学与技术,电子信息工程,信息与计算机技术,信息工程
\item 前期课程:英语,Linux应用基础(非必需)
\item 课程性质和要求:《计算机网络》是一门重要的专业基础课。深入理解网络的工作原理对学生在
  网络管理与开发方面具有重大意义。 本课程介绍给同学如下内容:
  \begin{itemize}
  \item 网络架构
  \item ISO/OSI参考模型
  \item TCP/IP协议族
  \end{itemize}
\item 参考教材\hfill
  \nocite{tanenbaum2011computer,fall2011tcp,kurose2013computer,bautts2005linux,hunt2002tcp,hall2009beej}
  \printbibliography[heading=none]{}
\end{itemize}


\end{document}

%%% Local Variables:
%%% mode: latex
%%% TeX-master: t
%%% End:
