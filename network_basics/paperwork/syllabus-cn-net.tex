\documentclass{wx672article} % $HOME/texmf/tex/latex/wx672article.cls


\usepackage{wx672bib}
\addbibresource{net.bib}

\title{计算机网络教学大纲}
\author{}\date{}

\pagestyle{empty}

\begin{document}

%\maketitle

\begin{description}
\item[计算机网络]
\item[每周课次: 4]
\item[学分:] 4
\item[前导课程:] 基本的C、Python编程知识
\item[课程简介:] 本课程涵盖计算机网络体系结构,网络设计的基本思想和实际问题,网络编程。本
  课程将专注于网络的设计、实现、分析,以及大型网络系统的评估。本课程的目标是让学生深入理解
  网络的相关概念,具体包括:
  \begin{itemize}
  \item 数据通信的基本概念
  \item 网络架构
  \item TCP/IP协议栈
  \item 网络管理与安全
  \end{itemize}
\item[课程主要内容:] 本课程的主要内容包括
  %http://www.csee.usf.edu/~kchriste/class2/syllab2.html
  \begin{itemize}
  \item[第1周:] 网络协议分层,与服务模块
  \item[第2周:] 互联网的技术概念
  \item[第3周:] 应用层协议与C/S模型
  \item[第4周:] 网络编程基础
  \item[第5周:] TCP和UDP
  \item[第6周:] TCP拥塞控制
  \item[第7/8周:] IP与路由
  \item[第9周:] 链路层协议
  \item[第10周:] 交换与桥接
  \item[第11/12周:] 无线网络基础
  \item[第13/14周:] 网络安全
  \item[第15周:] 防火墙
  \item[第16周:] SNMP网络管理
  \end{itemize}
\item[教材与参考书目:]\hfill
  \nocite{tanenbaum2011computer,fall2011tcp,kurose2013computer,bautts2005linux,hunt2002tcp,hall2009beej}
  \printbibliography[heading=none]{}
\end{description}

\end{document}

%%% Local Variables:
%%% mode: latex
%%% TeX-master: t
%%% End:
