% Created 2016-05-13 Fri 09:00
\documentclass[11pt]{article}
\usepackage[utf8]{inputenc}
\usepackage[T1]{fontenc}
\usepackage{fixltx2e}
\usepackage{graphicx}
\usepackage{longtable}
\usepackage{float}
\usepackage{wrapfig}
\usepackage{rotating}
\usepackage[normalem]{ulem}
\usepackage{amsmath}
\usepackage{textcomp}
\usepackage{marvosym}
\usepackage{wasysym}
\usepackage{amssymb}
\usepackage{hyperref}
\tolerance=1000
\usepackage{fullpage}
\usepackage{indentfirst}
\usepackage[indentafter,pagestyles]{titlesec}
\usepackage{minted}
\usepackage{xltxtra}
\usepackage{xeCJK}
\tolerance=1000
\setlength{\parindent}{2.5em}
\setmainfont{DejaVu Serif}
\setsansfont{DejaVu Sans}
\setmonofont{DejaVu Sans Mono}
\setCJKmainfont[BoldFont={WenQuanYi Zen Hei}]{SimSun}
\setCJKfamilyfont{hei}{WenQuanYi Zen Hei}
\setCJKfamilyfont{song}{SimSun}
\XeTeXlinebreaklocale "zh"
\XeTeXlinebreakskip = 0pt plus 1pt
\graphicspath{{./figs/}{../figs/}{./}{../}}
\renewcommand{\contentsname}{目录}
\renewcommand{\listfigurename}{插图目录}
\renewcommand{\listtablename}{表格目录}
\renewcommand{\abstractname}{摘要}
\renewcommand{\appendixname}{附录}
\renewcommand{\indexname}{索引}
\renewcommand{\figurename}{图}
\renewcommand{\tablename}{表}
\author{王晓林}
\date{2015-03-06}
\title{《计算机网络》考核大纲}
\hypersetup{
  pdfkeywords={},
  pdfsubject={},
  pdfcreator={Emacs 24.5.1 (Org mode 8.2.10)}}
\begin{document}

\maketitle
\tableofcontents


\begin{itemize}
\item 课程编号:A05055
\item 课程名称:计算机网络
\item 任课教师:王晓林
\item 适应专业:计算机本科专业
\item 授课学时:32/16
\item 考试方式:闭卷笔试
\item 命题规则:按《西南林学院考试命题规则》(试行)(1996年10月修订)执行
\item 考试时间:120分钟
\item 推荐教材:
\begin{itemize}
\item \emph{Computer Networks}, Xie Xiren, Electronics Engineering Press.
\item \emph{Computer Networks}, Andrew S.Tanenbaum, Tsinghua Unversity Press.
\item \emph{Computer Networks Lab Tutorials}, Xie Qian, Electronics Engineering Press.
\item \emph{Computer Network Tutorials}, Du Yu, People's Post and
Telecommunication Press, 1st edition Jan 2002.
\item \emph{Computer Network Practical Tutorials}, Wang Li, Zhang Yuxiang,
Yang Lianghuai, Tsinghua University Press, 1st edition Dec 1999.
\end{itemize}
\item 课程性质及教学目的:本课程为计算机专业本科专业基础课。要求学生全面了解TCP/IP网络的工作原理。
\end{itemize}

各章节考核目标如下:
\section{TCP/IP Introction}
\label{sec-1}
\subsection{考核知识点}
\label{sec-1-1}
\begin{itemize}
\item What's TCP/IP?
\item What's a protocol stack?
\item Relation between adjacent layers?
\item Relation between peer layers?
\item What's RFC?
\end{itemize}
\subsection{考核要求}
\label{sec-1-2}
Know the above key points.
\section{Ethernet and ARP}
\label{sec-2}
\subsection{考核知识点}
\label{sec-2-1}
\begin{itemize}
\item Frame format
\item Address format
\item Broadcast address
\item CSMA/CD
\item How ARP works?
\end{itemize}
\subsection{考核要求}
\label{sec-2-2}
Know the above key points.
\section{IP routing}
\label{sec-3}
\subsection{考核知识点}
\label{sec-3-1}
\begin{itemize}
\item IP address (format, classification, Netmask)
\item Route table
\item How routing works
\item Subnetting
\item IPv6 address format
\end{itemize}
\subsection{考核要求}
\label{sec-3-2}
Know the above key points.
\section{TCP, UDP}
\label{sec-4}
\subsection{考核知识点}
\label{sec-4-1}
\begin{itemize}
\item Sockets
\item 3-way handshake
\item TCP header
\item tcpdump
\item Socket programming
\end{itemize}
\subsection{考核要求}
\label{sec-4-2}
Know the above key points.
\section{Application layer protocols}
\label{sec-5}
\subsection{考核知识点}
\label{sec-5-1}
\begin{itemize}
\item HTTP
\item FTP
\item SMPT, POP3, IMAP
\item DNS
\end{itemize}
\subsection{考核要求}
\label{sec-5-2}
Know the above key points.
% Emacs 24.5.1 (Org mode 8.2.10)
\end{document}