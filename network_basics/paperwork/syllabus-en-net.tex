\documentclass{wx672article} % $HOME/texmf/tex/latex/wx672article.cls


\usepackage{wx672bib}
\addbibresource{net.bib}

\title{Syllabus for Computer Networking}
\author{}\date{}

\pagestyle{empty}

%\newgeometry{top=1cm}

\begin{document}

%\maketitle

\begin{description}
\item[Computer Networking]
\item[Contacts: 4L]
\item[Credits:] 4
\item[Prerequisites:] Basic c/python programming knowledge is not mandatory, but helpful.
\item[Course Description:] The past few years have seen a remarkable growth in the global
  network infrastructure. The Internet has grown from a research curiosity to something we
  all take for granted, and is becoming as essential as the ubiquitous telephone and
  utility networks. It has been able to withstand rapid growth fairly well and its core
  protocols have been robust enough to accommodate applications that were unforeseen by
  the original Internet designers, such as the World Wide Web.

  How does this global network infrastructure work and what are the design principles on
  which it is based? In what ways are these design principles compromised in practice? How
  do we make it work better in today's world? How do we ensure that it will work well in
  the future in the face of rapidly growing scale and heterogeneity? And how should
  Internet applications be written, so they can obtain the best possible performance both
  for themselves and for others using the infrastructure? These are some issues that we
  will grapple with in this course. The course will focus on the design, implementation,
  analysis, and evaluation of large-scale networked systems.

  The goals of this course are to give students clear understanding about comprehensive
  concepts of computer networking. In particular, our goals are to understand:
  \begin{itemize}
  \item Basic Concepts of Data Communication
  \item Network Infrastructure
  \item ISO/OSI RM and TCP/IP Protocol Suite
  \item Network management and security
  \end{itemize}
\item[Course Topics:] This course will cover the following topics:
  %http://www.csee.usf.edu/~kchriste/class2/syllab2.html
  \begin{itemize}
  \item[Week 1:] Protocol layers and service models. OSI and Internet protocols.
  \item[Week 2:] What is the Internet. Concepts of delay, security, and Quality of Service
    (QoS).       
  \item[Week 3:] Application layer protocols and client-server model.
  \item[Week 4:] Sockets programming in C (client-server and web server programs).
  \item[Week 5:] Reliable data transfer. Stop-and-Go evaluation. TCP and UCP semantics and
    syntax.      
  \item[Week 6:] TCP RTT estimation. Principles of congestion control.
  \item[Week 7:] Principles of routing. Link-state and distance vector. IP semantics and
    syntax.      
  \item[Week 8:] Link-state and distance vector routing. Midterm Exam.
  \item[Week 9:] Link layer. Error detection. Multiple access protocols. IEEE 802.3
    Ethernet.
  \item[Week 10:] Switching and bridging. Media. Signal strength. Data encoding.
  \item[Week 11:] Wireless and mobile networks.
  \item[Week 12:] Security. Overview of threats, cryptography, authentication.
  \item[Week 13:] Network firewalls.
  \item[Week 14:] Network management including SNMP. Network troubleshooting.
  \item[Week 15:] Hot topics such as SDN and IoT. Course wrap-up. Review for final exam.
  \item[Week 16:] Comprehensive final exam.
  \end{itemize}
\item[Textbook and References:]\hfill
  \nocite{tanenbaum2011computer,fall2011tcp,kurose2013computer,bautts2005linux,hunt2002tcp,hall2009beej}
  \printbibliography[heading=none]{}
\end{description}

\end{document}

%%% Local Variables:
%%% mode: latex
%%% TeX-master: t
%%% End:
