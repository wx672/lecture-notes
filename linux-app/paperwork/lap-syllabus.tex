\documentclass{wx672ctexart}

\usepackage{wx672bib}
\addbibresource{os.bib}

\author{王晓林}
\date{2021-03-09}
\title{《Linux应用编程》教学大纲}

\begin{document}

\maketitle
\tableofcontents
\clearpage

\begin{itemize}
\item 课程编号: 41100169
\item 学时: 64 (理论: 32; 实验: 32)
\item 学分: 4
\item 实习: 0
\item 面向专业: 计算机科学与技术,电子信息工程,信息与计算机技术
\end{itemize}

\section{课程大纲}
\label{sec-1}

\subsection{课程内容}
\label{sec-1-1}

\begin{enumerate}
\item Getting started
  \begin{itemize}
  \item Commandline introduction
  \item Editors (Vim, Emacs)
  \end{itemize}
\item Shell basics
  \begin{itemize}
  \item Basic operations
  \item Shell programming
  \end{itemize}
\item Linux programming environment
  \begin{itemize}
  \item C programming environment
  \item The tool chain
  \item Header files and macros
  \item Library files
  \item Error handling
  \item The make utility
  \item Version control
  \item Manual pages
  \item A sample GNU package
  \item Pointers in C
  \item Pointers and arrays
  \end{itemize}
\item The Linux environment
\item OS basics
  \begin{itemize}
  \item Hardware
  \item Bootstrapping
  \item Interrupts
  \item System calls
  \end{itemize}
\item Working with files
  \begin{itemize}
  \item Files
  \item Directories
  \end{itemize}
\item Processes and threads
  \begin{itemize}
  \item Virtual memory
  \item Processes
  \item Threads
  \item Signals
  \end{itemize}
\item Inter-process communication
  \begin{itemize}
  \item Pipes and FIFOs
  \item Message queues
  \item Semaphores
  \item Classical IPC problems
  \item Shared memory
  \item Sockets
  \end{itemize}
\end{enumerate}

\subsection{实验内容}
\label{sec-1-2}

参见第\ref{sec:lab}节《Linux应用编程》实验教学大纲。

\subsection{实习}
\label{sec-1-3}

无

\subsection{考核}
\label{sec-1-4}

\begin{itemize}
\item 考试: 50\%
\item 作业: 50\%
\end{itemize}

\subsection{参考教材}
\label{sec-1-5}

\nocite{cs241, matthew2008beginning, cooper10bash, raymond2003art, stevens2013advanced,
  love:2007:lsp:1205435, kerrisk:2010:lpi:1869911, bryant2010computersystems,
  silberschatz11essentials, tanenbaum2008modern, bovet2005understanding}
\printbibliography[heading=none]{}

\section{课程说明}
\label{sec-2}

\subsection{课程性质和要求}
\label{sec-2-1}

《Linux应用编程》是一门重要的专业基础课。熟悉Linux平台上的开发环境对学生在软件编程、开发方
面具有重大意义。 本课程介绍给同学如下内容:
\begin{itemize}
\item Shell basics and programming
\item Linux programming environmnt
\item OS basics
\item File operations
\item Processs and Threads
\item Inter-process communication
\item Sockets
\end{itemize}

\subsection{课程重点}
\label{sec-2-2}

\begin{itemize}
\item Shell basics and programming
\item Linux programming environment
\item OS related C programming
\end{itemize}

\subsection{作业、实习要求}
\label{sec-2-3}
作业迟交一天扣分10\%。

\subsection{与其它课程的关系}
\label{sec-2-4}

\begin{itemize}
\item 前期课程:计算机组成原理,Linux应用基础,C编程,汇编编程
\item 后期课程:Linux系统分析
\end{itemize}

\subsection{课时安排}
\label{sec-2-5}

\begin{center}
  \begin{tabular}{lrr}
    \hline
    课程内容 & 理论学时 & 实验学时\\
    \hline
    Shell basics and programming&6&6\\
    Linux programming environmnt&6&6\\
    OS basics                   &4&4\\
    File operations             &4&4\\
    Processs and Threads        &4&4\\
    Inter-process communication &4&4\\
    Sockets                     &4&4\\    
    \hline
  \end{tabular}
\end{center}

\subsection{特殊说明}
\label{sec-2-6}

无

\section{实验教学大纲}
\label{sec:lab}

\begin{itemize}
\item 课程编号: 41100169
\item 学时: 64 (理论: 32; 实验: 32)
\item 学分: 4
\item 实习: 0
\item 授课对象: 计算机科学与技术,电子信息工程,信息与计算机技术
\end{itemize}

\subsection{实验教学的目的和要求}
\label{sec-3-1}

通过编程实践,让学生熟悉Linux平台的软件开发环境,了解与操作系统相关的编程知识。

\subsection{实践教学大纲}
\label{sec-3-2}

\begin{center}
  \begin{tabular}{lr}
    \hline
    实验安排 & 学时\\
    \hline
    Shell basics&6\\
    Shell programming&4\\
    Linux programming environmnt&6\\
    File operations&4\\
    Processs and Threads&4\\
    Inter-process communication&4\\
    Sockets&4\\
    \hline
  \end{tabular}
\end{center}

\subsection{实验设备要求}
\label{sec-3-3}

\begin{itemize}
\item Debian PC
\end{itemize}

\subsection{实验内容}
\label{sec-3-4}

参见自编《实验指导》:
\begin{itemize}
\item
  \href{https://cs6.swfu.edu.cn/~wx672/lecture_notes/linux/bash/shell_basics.html}{\emph{Play
    with bash}};
\item \href{https://cs6.swfu.edu.cn/~wx672/lecture_notes/linux/c/c_dev.html}{\emph{C programming
  environment on Linux}};
\item \href{http://cs6.swfu.edu.cn/~wx672/lecture_notes/os/lab.html}{\emph{Linux system programming}}。
\end{itemize}

\subsection{实验报告要求}
\label{sec-3-5}

按规定格式完成,迟交报告每天扣分10\%。

\subsection{成绩考核}
\label{sec-3-6}

实验报告满分100,60分及格。

\subsection{实验指导和参考书目}
\label{sec-3-7}

自编实验指导:
\begin{itemize}
\item
  \url{https://cs6.swfu.edu.cn/~wx672/lecture_notes/linux/bash/shell_basics.html}
\item \url{https://cs6.swfu.edu.cn/~wx672/lecture_notes/linux/c/c_dev.html}
\item \url{http://cs6.swfu.edu.cn/~wx672/lecture_notes/os/lab.html}
\end{itemize}

\subsection{特别说明}
\label{sec-3-8}

无

\section{课程简介}
\label{sec-4}

\begin{itemize}
\item 课程编号: 41100169
\item 学时: 64 (理论: 32; 实验: 32)
\item 学分: 4
\item 实习: 0
\item 面向专业: 计算机科学与技术,电子信息工程,信息与计算机技术
\item 前期课程:英语,计算机组成原理,Linux应用基础,C编程,汇编知识
\item 课程性质和要求:《Linux应用编程》是一门重要的专业基础课。熟悉Linux平台上的软件开发环
  境对学生在软件编程、开发方面具有重大意义。 本课程介绍给同学如下内容:
  \begin{itemize}
  \item Shell basics and programming
  \item Linux programming environmnt
  \item OS basics
  \item File operations
  \item Processs and Threads
  \item Inter-process communication
  \item Sockets
  \end{itemize}
\item 参考教材\hfill
  \nocite{cs241, matthew2008beginning, cooper10bash, raymond2003art, stevens2013advanced,
    love:2007:lsp:1205435, kerrisk:2010:lpi:1869911, bryant2010computersystems,
    silberschatz11essentials, tanenbaum2008modern, bovet2005understanding}
  \printbibliography[heading=none]{}
\end{itemize}
\end{document}

%%% Local Variables:
%%% mode: latex
%%% TeX-master: t
%%% End:
