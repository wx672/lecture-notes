\documentclass{wx672article}

\author{王晓林}
\date{\today}
\title{Linux平台的C开发环境介绍}

\begin{document}

\maketitle
\tableofcontents


\section{Linux平台的经典IDE}

IDE这个词大家都不陌生吧,Integrated Development Environment, 集成开发环境。
说到IDE,估计你会想到Windows平台的Visual Studio,或者Java开发常用的Eclipse。
今天我向大家推荐一下世界上最好的IDE。最好的IDE?你一定开始怀疑我是在吹牛了吧。
那么,好吧,我们先看看到底什么是IDE。很简单,一个IDE无非包括如下一些功能模块:
\begin{itemize}
\item 一个编辑器;
\item 一个编译器;
\item 一个调试器;
\item 其它一些辅助功能,比如用鼠标拖控件。
\end{itemize}

什么是最好的IDE?那肯定是:
\begin{center}
最好的IDE = 最好的编辑器 + 最好的编译器 + 最好的调试器
\end{center}

上面我们提到的VS和Eclipse都没能把最好的东西集成到一起,所以,我说它们不是最好的。
有哪个IDE做到这一点了吗?只有Emacs。

Emacs是一个有数十年悠久历史的开源编辑器。不折不扣地说,它也是世界上最强大的编辑器,
因为它是模块化设计,你如果觉得它还缺少什么功能,那么你可以给它添加一个新的功能模块。
如此日积月累几十年,凡是你能想到的功能都已经添加好了。

Emacs可以很方便地调用世界上最牛的编译器(gcc)和调试器(gdb)。
\begin{center}
Emacs + gcc + gdb
\end{center}
这就是世界上最好的IDE。也许你会说「Emacs不能拖控件啊」。
没错,但在我看,拖控件并不总是一个受人欢迎的功能,至少在系统编程的时候,它毫无用处。

而且,从学习的角度来说,「用鼠标编程」绝对是一个非常恶劣的习惯,因为这根本就是在逃避学习。
「鼠标化的IDE」隐藏了很多学生应该了解的技术细节。
我们学院的绝大多数学生居然不知道C程序是要编译之后才能运行的。他们以为写好了程序,
只要「按那个“感叹号”按钮」就可以了。这就是「鼠标教学」的成果。
Emacs可以帮助你克服「鼠标依赖」,强迫你熟练地使用键盘。

更重要的是,Emacs不只是个IDE,它是个ICE(Integrated Computing Environment,
这名字是我刚编出来的)。Emacs的设计目标就是,你装了个Unix或者Linux系统,
不需要装任何其它软件,只要装一个Emacs就够了,它能帮助你完成所有的任务。
也就是说,除了编程,你还可以用它写论文、做幻灯片、浏览网页、收发邮件、聊天、听歌、
看照片、玩游戏……目前,好像除了直接在Emacs里看电影还不行,其它的都实现了。

Emacs如此「大一统」的设计目标显然有违Unix的设计原则,do one thing, and do it well
(做一件事,并且把它真正做好)。
但好在Emacs是模块化的,它的每一个功能模块都绝对遵循do one thing, and do it well原则。
你不需要哪个功能,可以不装那个模块。

另外,还是从学习的角度来说,Emacs的学习曲线貌似比其他IDE要长不少,但是你
\begin{itemize}
\item 不必学习VC去写C/C++,
\item 不必学习eclipse去写Java,
\item 不必学习MS-Word去写报告、幻灯片,
\item 不必学习……
\end{itemize}

一句话,``Everything Emacs'',你可以省下大量不必要的学习时间。
人生苦短,何必让你的生活被VC/eclipse/MS-Word 搞得头昏脑胀呢?
\textbf{简单而强大,本就是计科专业学生和非专业学生应有的不同。} 

Emacs绝对强大,但是否「方便」就不好说了。因为「方便」是一个很主观的概念。
反正,作为一个18年的老用户,我肯定觉得方便。其他IDE太无聊了,那么花哨而庞大的东西,
却只适用于应用层编程。既不能用来写论文,又不能做幻灯片,更不能用来听歌、玩游戏。
生活也太没有乐趣了。

最后一点,Emacs还是一个巨大的开放社区,在这里你能结识到更酷一些的程序员。

Emacs入门还是很简单的,它自带了一个基础教程。打开Emacs,按 \texttt{Ctrl-h t},
教程就出现在你面前了。照着它边看边练,英文不太困难的话,一个小时应该可以走一遍了。
之后,
\begin{itemize}
\item \texttt{Ctrl-h i m emacs} 就可以调出详细的Emacs使用手册;
\item \texttt{Ctrl-h i m emacs lisp intro} 可以调出Emacs Lisp入门教程;
\item \texttt{Ctrl-h i m elisp} 可以调出完整的elisp编程手册。
\end{itemize}

当然,Google永远是你最好的帮手。Happy emacsing!

\section{GCC使用入门}

GNU Compiler Collection (GCC) 是GNU自由软件项目开发出的一整套编译器,包括
\begin{itemize}
\item gcc: c编译器
\item g++: c++编译器
\item gcj: java编译器
\item gfortran: Fortran编译器
\item GNAT: ada编译器
\item gccgo: Go编译器
\item 更多其它语言的编译器
\end{itemize}

注意GCC和gcc的区别。当我们说GCC的时候,是在说那一整套编译器;而当我们说gcc的时候,
是在说c编译器。gcc的开发者就是GNU自由软件运动的创始人,大名鼎鼎的Richard Stallman。
gcc于1987年问世,自诞生以来就广受推崇。现在,它仍然是诸多Unix-like操作系统的标准配置。
Linux内核的数千万行源代码就是用gcc编译的。

gcc很强大,它支持包括x86, ARM在内的数十种硬件架构,并且支持交叉编译,也就是说,
你可以在x86平台上写程序,然后把它编译成能在ARM平台上运行的二进制文件。
如果你的系统里已经装好了gcc,那么你可以用 \texttt{man gcc | wc -l} 命令来数一数gcc手册的长度。
这一万六千多行的庞大手册从侧面说明了gcc功能的强大。

不过,作为初学者,我们并不必关心gcc有多强大,少数几个简单的命令选项,
就足以应付我们的c程序编译了。假设你有个c文件叫 \texttt{hello.c},

\begin{minted}[mathescape=true,linenos=true,numbersep=5pt,framesep=2mm]{c}
#include <stdio.h>

int main(void)
{
  printf("Hello, world!\n");
  return 0;
}
\end{minted}

那么,在最通常的情况下,你只需要:
\begin{verbatim}
gcc hello.c
\end{verbatim}
就可以得到一个名叫 \texttt{a.out} 的可执行文件了。如果你不喜欢 \texttt{a.out} 这个古老的名字,那么
\begin{verbatim}
gcc hello.c -o hello
\end{verbatim}
就可以得到一个名叫 \texttt{hello} 的可执行文件了。你应该猜到了,选项 "\texttt{-o}" 代表 \texttt{output},
“输出”的意思。要运行 \texttt{hello} 看看结果的话,
\begin{verbatim}
./hello
\end{verbatim}
就可以了。 \texttt{./} 代表当前目录,也就是你的 \texttt{hello.c} 所在的目录。

当然,生活并不总是像\texttt{Hello, world!}这样简单。比如说,还是 \texttt{hello.c},

\begin{minted}[mathescape=true,linenos=true,numbersep=5pt,framesep=2mm]{c}
#include <stdio.h>

int main(void)
{
  printf("Hello, world!\n")
  return 0;
}
\end{minted}

你敲完 \texttt{gcc hello.c}, 一回车,意外地看到了如下一些东西:
\begin{verbatim}
1  hello.c: In function ‘main’:
2  hello.c:6:3: error: expected ‘;’ before ‘return’
3    return 0;
4    ^
\end{verbatim}

OMG! 怎么办?第一,别慌;第二,别懒。其实,上面这几行输出并没有几个单词,
而且差不多都认识,静下心来仔细看看,还是很好理解的嘛。
\begin{enumerate}
\item 第一行意思是说,在函数 \texttt{main} 里面发现了点问题;
\item 第二行和第三行具体给你指出了出错的地方,在第6行,第3列,\texttt{r}的前面应该有个分号;
\item 第四行的 \texttt{\textasciicircum{}} 就是个向上的箭头,指向 \texttt{r}, 也就是问题点所在。
\end{enumerate}

怎么样,不太困难吧?重新编辑你的 \texttt{hello.c}, 在 \texttt{return} 与 \texttt{)} 之间加上分号,

\begin{minted}[mathescape=true,linenos=true,numbersep=5pt,framesep=2mm]{c}
#include <stdio.h>

int main(void)
{
  printf("Hello, world!\n");
  return 0;
}
\end{minted}

然后再编译一下,
\begin{verbatim}
1  hello.c: In function ‘main’:
2  hello.c:5:3: error: stray ‘\357’ in program
3      printf("Hello, world!\n");
4      ^
5  hello.c:5:3: error: stray ‘\274’ in program
6  hello.c:5:3: error: stray ‘\233’ in program
7  hello.c:6:3: error: expected ‘;’ before ‘return’
8     return 0;
9     ^
\end{verbatim}

OMG!!! 怎么问题越来越多了?第一,别慌;第二,别懒;第三,别马虎。
上面出错信息的最后三行你都见过了,显然,分号还是有问题。再仔细看看,那居然是个中文分号!
你怎么可以这样马虎呢?

在初学者当中,类似上面这样的小错误是屡见不鲜的。怎么办呢?别慌、别懒、别马虎,
静下心来做事情,这就够了。

\subsection{从hello.c到hello.exe}

我们在命令行敲完 \texttt{gcc hello.c -o hello.exe}, 然后一回车,不出错误的话,一个可执行文件,
\texttt{hello.exe}, 就诞生了。现在我们简要了解一下,敲完回车之后,电脑里到底发生了什么。
换句话说,就是了解一下我们常说的「编译」到底是怎么回事。

\includegraphics[width=.6\linewidth]{compilation}

上图中,椭圆框里面放的都是工具,包括
\begin{description}
\item[{编辑器}] 我们用的是Emacs,世界上最强大的编辑器。当然,你也可以用nano, vim,
或者其它什么编辑器。毕竟写一个 \texttt{Hello, world!} 并不必非要用那么高大上的工具。
但如果你以后想当个程序员,那么Emacs就应该是你的首选利器。

\item[{C预处理器}] C预处理器(c preprocessor)负责处理程序中以 \texttt{\#} 开头的程序语句,比如说:
\begin{itemize}
\item \texttt{\#include<stdio.h>} 。怎么处理?你肯定知道 \texttt{include} 是「包含」的意思,
也肯定知道 \texttt{stdio.h} 是一个文件的名字\footnote{在我们的Debian GNU/Linux系统中,
       它的完整路径是 \texttt{/usr/include/stdio.h} 。}。那么, \texttt{\#include<stdio.h>}
显然就是要把 \texttt{stdio.h} 文件的内容包含到你的程序中来。我们用的预处理器是cpp,
在命令行敲:
\begin{verbatim}
cpp hello.c
\end{verbatim}
看到了吗?原来程序中的第一行 \texttt{\#include<stdio.h>} 被扩展成了800多行。再比如说,
\item \texttt{\#define SQR(x) (x * x)}, 这是一句「宏定义」,意思是说,在后面的程序里,凡是遇到
\texttt{SQR(x)}, 就都给替换成 \texttt{(x * x)}. 这个替换工作,也是由 \texttt{cpp} 完成的。
\end{itemize}

\item[{C编译器}] 我们用的是gcc。编译器(Compiler)的工作是把 \texttt{cpp} 
处理过的源程序翻译成一个汇编程序。
如果你是头一次听说汇编语言,那么你应该立即去Google一下「汇编语言」。
简而言之,汇编语言是一种比C更底层,也就是更靠近CPU,的编程语言。
如果你想搞操作系统开发,或者硬件驱动开发,那么汇编就是你必须掌握的编程语言。

与其把C翻译成汇编,为什么我们不直接用汇编来写程序呢?两个原因,
\begin{enumerate}
\item 用汇编写程序比用C要累得多。和汇编相比,C是一门高级(或者说「高层」)语言。
所谓「高级」,通过上面的大流程图来看,就是离上面的那个程序员更近。
其实也就是对人更友好的意思。举个简单的例子,
\begin{verbatim}
i++
\end{verbatim}
如果用汇编来写,大概就成了下面这样
\begin{verbatim}
movl    $0, -4(%rbp)
addl    $1, -4(%rbp)
movl    -4(%rbp), %eax
\end{verbatim}
显然,还是 \texttt{i++} 更友好些吧。通常来讲,用高级语言(比如C)
写一条程序语句就相当于好几条,甚至几十条,汇编语句。
而一条汇编语句通常就对应一条机器指令。所以,用汇编些程序要累得多。

通过下面的命令,就可以把我们的 \texttt{hello.c} 翻译成一个汇编程序 \texttt{hello.s}:
\begin{verbatim}
gcc -S hello.c
\end{verbatim}
生成的 \texttt{hello.s} 就是下面这副样子:

\begin{minted}[mathescape=true,linenos=true,numbersep=5pt,frame=lines,framesep=2mm]{asm}
	.file   "hello.c"
	.section        .rodata
.LC0:
	.string "Hello, world!"
	.text
	.globl  main
	.type   main, @function
main:
.LFB0:
	.cfi_startproc
	pushq   %rbp
	.cfi_def_cfa_offset 16
	.cfi_offset 6, -16
	movq    %rsp, %rbp
	.cfi_def_cfa_register 6
	movl    $.LC0, %edi
	call    puts
	movl    $0, %eax
	popq    %rbp
	.cfi_def_cfa 7, 8
	ret
	.cfi_endproc
.LFE0:
	.size   main, .-main
	.ident  "GCC: (Debian 5.4.0-6) 5.4.0 20160609"
	.section        .note.GNU-stack,"",@progbits
\end{minted}

和前面的 \texttt{hello.c} 相比,我想你肯定和我一样,更愿意用C来写程序吧。

\item 用汇编语言写出的程序针对性很强,通常不能跨平台使用,可移植性差。
针对x86机器写的汇编程序,在其他机器(比如ARM, PowerPC, M68000, ...)上就不能用。
因为一条汇编指令就对应一条机器指令,而各CPU架构所支持的机器指令集都不一样,
那么对应的汇编程序当然就无法跨平台使用了。        
反之,用高级语言写的程序就比较容易跨平台,可移植性比较好。

所以,基于上述原因,通常我们会尽可能地选用高级语言来编程。
之后,再利用汇编器,把对人友好的高级程序语句翻译成对机器友好的底层程序语句。
\end{enumerate}

\item[{汇编器}] 汇编器(Assembler)是用来把人能看懂的汇编程序翻译成人读不懂,但机器能读懂的二进制程序,
通常我们把它叫做目标文件(object file),通常以 \texttt{.o} 结尾。 
我们用的汇编器是 GNU Assembler (gas)。

通常如果想要生成一个 \texttt{.o} 文件的话,我们用下面的命令:               
\begin{verbatim}
gcc -c hello.c
\end{verbatim}
这样,gcc会调用gas帮我们生成一个目标文件。
这时,你的目录里应该多了一个 \texttt{hello.o} 文件了。好奇的话,你可以
\texttt{cat hello.o} 来看看它的内容。我肯定你读不懂它,除非你是CPU。
用 \texttt{hd hello.o} 来看看它的内容,感觉会稍好些,虽然还是看不懂。

\item[{链接器}] 链接器(Linker)是用来把若干个 \texttt{.o} 文件结合成一个可执行文件。
我们用的链接器是ld。

也许你会问,「我只写了一个 \texttt{hello.c}, 经过编译和汇编之后,
只产生了一个而不是多个 \texttt{.o} 文件,还需要链接吗」?的确,
你只有一个 \texttt{.o} 文件,但在你的 \texttt{hello.c} 
文件里还用到了别人的 \texttt{.o} 文件,比如说, \texttt{printf()} 函数并不是你写的吧?
它存在于系统自带的函数库里。系统函数库里装的其实就是一大堆 \texttt{.o} 文件。
这些 \texttt{.o} 文件里都是供我们调用的一个个函数,其中就包括 \texttt{printf()} 函数。
所以,你自己的 \texttt{hello.o} 必须要和系统函数库中包含 \texttt{printf()} 的那个
\texttt{.o} 文件\footnote{实际上是 \texttt{libc.so} 文件。 \texttt{so} 代表shared object,
               是Unix平台通用的动态链接函数库。想了解更多?去Google一下"shared object"
               就知道了。}链接,然后才能得到最终的可执行文件。

\item[{调试器}] 调试器(Debugger)是用来帮助我们找出可执行文件中的bug。我们用的调试器是gdb,
它可以
\begin{itemize}
\item 分步执行程序
\item 设置断点
\item 追踪变量的值
\item 查看堆栈
\item 还有很多高深的功能
\end{itemize}
如果你的程序像 \texttt{Hello, world!} 那样简单,那么通常也就不会有什么bug,
自然也就不需要调试器了。
\end{description}

上面我们简单介绍了一下从 \texttt{hello.c} 到 \texttt{hello.exe} 的过程。了解这些基础知识,
有助于我们加深对编程的理解,可以让我们在今后的学习中少走弯路。

\subsection{常用编译选项}

在上一节,我们看到在使用gcc编译C程序的时候,可以跟上一些选项,比如 \texttt{-o},
后面可以给出可执行文件的名字。下面我们再介绍几个常用的编译选项。

\subsubsection{\texttt{-Wall}}
\label{sec-2-2-1}
这个选项非常有用,应该随时都带着。大写的 \texttt{W} 代表warning, \texttt{all} 就代表all,那么 \texttt{-Warning}
就代表打开所有的告警。也就是说,编译过程中发现的算不上是「错误」
的小毛病也都会被提示出来。有的时候,这些小毛病还是挺要命的,比如下面这个小程序 \texttt{wall.c}

\begin{minted}[mathescape=true,linenos=true,numbersep=5pt,framesep=2mm]{c}
#include <stdio.h>

int main (void)
{
  printf ("Two plus two is %f\n", 4);
  return 0;
}
\end{minted}

如果不带 \texttt{-Wall} 直接编译的话, \texttt{gcc wall.c}, 看不到任何错误提示。
可是运行 \texttt{a.out} 的输出结果却是:
\begin{verbatim}
Two plus two is 0.000000
\end{verbatim}
这个结果明显是错误的。如果编译时带上 \texttt{-Wall} 选项, \texttt{gcc -Wall wall.c}, 会看到如下输出:
\begin{verbatim}
wall.c: In function ‘main’:
wall.c:5:11: warning: format ‘%f’ expects argument of type ‘double’,
       but argument 2 has type ‘int’ [-Wformat=]
printf ("Two plus two is %f\n", 4);
        ^
\end{verbatim}
问题被提示了出来,用 \texttt{\%f} 的格式来输出整型数是不合情理的。记住,编译时永远带上
\texttt{-Wall}, 而且 \texttt{W} 必须大写!

\subsubsection{\texttt{-E}}

\texttt{-E} 这个选项是告诉 \texttt{gcc}, 调用完 \texttt{cpp} 就停下来。也就是说 \texttt{gcc -E hello.c} 和 \texttt{cpp
    hello.c} 是一回事。关于 \texttt{cpp} 我们前面已经提到过,它是一个c preprocessor,
作用之一就是把源程序中的宏定义(Macro)扩展还原成它本来的字符串。
宏定义是C编程中经常要用到的强大武器,而且编程大师们可以把它用得非常复杂。
比如在Linux的内核源代码里就有下面这样的宏定义:
\begin{minted}[mathescape=true,linenos=true,numbersep=5pt,framesep=2mm]{c}
#define INIT_LIST_HEAD(ptr) do { \
	   (ptr)->next = (ptr);(ptr)->prev= (ptr);  \
	} while(0)
\end{minted}
一个完整的 \texttt{do-while} 结构被 \texttt{INIT\_LIST\_HEAD(prt)} 代表了。你知道 \texttt{do \{\} while(0)}
中的花括号里是可以放很多程序语句的,那么你肯定也想到了,宏定义可以用来代表非常非常复杂的东西。

常识告诉我们,越复杂的东西越容易隐藏着bug。 \texttt{-E} 这个选项可以帮助我们排除宏定义中的bug。
比如下面这个小程序 \texttt{macro.c} 里用到了一个很简单的宏定义 \texttt{SQR(x)}

\begin{minted}[mathescape=true,linenos=true,numbersep=5pt,framesep=2mm]{c}
#include <stdio.h>
#define SQR(x) (x * x)
int main()
{
    int counter;    /* counter for loop */
    for (counter = 0; counter < 5; ++counter)
    {
	printf("x %d, x squared %d\n",
	    counter+1, SQR(counter+1));
    }
    return (0);
}
\end{minted}

\texttt{SQR(x)} 显然是要对 \texttt{x} 做平方运算。编译一下, \texttt{gcc -Wall macro.c}, 很顺利,
没有任何出错迹象。现在运行一下 \texttt{a.out}, 看看结果:
\begin{verbatim}
x 1, x squared 1
x 2, x squared 3
x 3, x squared 5
x 4, x squared 7
x 5, x squared 9
\end{verbatim}
显然是错误的!问题就出在 \texttt{SQR(x)} 。借助 \texttt{-E} 把它还原扩展开,程序变成了这样:
\begin{minted}[mathescape=true,linenos=true,numbersep=5pt,framesep=2mm]{c}
/* 前面省略无数行 */
int main()
{
    int counter;
    for (counter = 0; counter < 5; ++counter)
    {
	printf("x %d, x squared %d\n",
	    counter+1, (counter+1 * counter+1));
    }
    return (0);
}
\end{minted}
看明白了吗?在第8行, \texttt{SQR(counter+1)} 被扩展成了 \texttt{(counter+1 * counter+1)},
而我们真正想要的是 \texttt{((counter+1) * (counter+1))}, 所以程序中的宏定义不该是
\begin{minted}[mathescape=true,linenos=true,numbersep=5pt,frame=lines,framesep=2mm]{c}
#define SQR(x) (x * x)
\end{minted}
而应该是
\begin{minted}[mathescape=true,linenos=true,numbersep=5pt,frame=lines,framesep=2mm]{c}
#define SQR(x) ((x) * (x))
\end{minted}
如果你也喜欢宏定义,那么一定要记住 \texttt{-E} 这个编译选项。

\subsubsection{\texttt{-D}}

Debug的时候,我们经常会在程序里加入一些 \texttt{printf()} 语句,
借助它输出某些关键变量的值,帮助我们思考。在bug被解决之后,这些 \texttt{printf()}
语句也就没用了,如果要一个个地删除掉,实在是一件麻烦的事情。想省点事的话,你可以借助一
下 \texttt{-D} 这个编译选项。比如,下面这个小程序 \texttt{stackoverflow.c},

\begin{minted}[mathescape=true,linenos=true,numbersep=5pt,framesep=2mm]{c}
#include<stdio.h>

int i=0;

int main(void)
{
#ifdef DEBUG     
  printf("%d\t",i++);
#endif   
  main();
  return 0;
}
\end{minted}

主函数递归地调用它自己,这实在不是件有意义的工作。
但如果你想知道调用多少次之后栈才会溢出,那么可以像上面程序中的第8行那样,利用
\texttt{printf()} 来输出计数器变量 \texttt{i} 的值。

第7、9两行是干什么用的呢?如果你正常编译这个小程序
\begin{verbatim}
gcc -Wall stackoverflow.c
\end{verbatim}
然后运行 \texttt{./a.out}, 你只能看到如下一行输出,那就是著名的
\begin{verbatim}
Segmentation fault
\end{verbatim}
很显然,栈溢出,递归程序就结束了,而且 \texttt{printf()} 没有发挥作用。
但是,如果你像下面这样编译:
\begin{verbatim}
gcc -Wall -DDEBUG stackoverflow.c
\end{verbatim}
之后再运行 \texttt{./a.out}, 怎么样?在我的电脑上,Segmentation fault之前,=i= 最后的值是
523629。很显然,带上编译选项 \texttt{-DDEBUG} 之后, \texttt{printf()} 起作用了。现在,你该猜到
\texttt{-D} 和程序中的
\begin{verbatim}
#ifdef DEBUG
...
#endif
\end{verbatim}
之间的关系了吧? \texttt{DEBUG} 也是个Macro, \texttt{-DDEBUG} 就相当于在程序里写上
\begin{verbatim}
#define DEBUG 1
\end{verbatim}
程序中的 \texttt{\#ifdef DEBUG} 就是说「如果 \texttt{DEBUG} 有定义的话」,那么就执行之后的程序语句,
直到看见 \texttt{\#endif} 为止。

所以,编译时如果不带 \texttt{-D}, 那么 \texttt{DEBUG} 就没定义,于是 \texttt{\#ifdef DEBUG}
这句判断结果就是 \texttt{false}, 于是它后面的 \texttt{printf()} 就不会被执行。反之,编译时带上 \texttt{-DDEBUG},
那么 \texttt{DEBUG} 就有了定义,于是 \texttt{\#ifdef DEBUG} 判断就返回 \texttt{true}, 于是 \texttt{printf()}
就发挥作用了。

如此一来,你再也不用为删除多余的 \texttt{printf()} 操心了,只需要操控 \texttt{-D}
这个小开关就可以达到目的了。

\subsubsection{更多选项}
\label{sec-2-2-4}
\texttt{gcc} 的编译选项多如牛毛,但做为初学者,知道上面这些就算是入门了。随着学习的深入,
更多的选项也会逐渐变成我们的常用选项。比如,
\begin{description}
\item[{\texttt{-g}}] 如果你想用gdb来debug程序的话,编译时一定要带上它。
\item[{\texttt{-l}}] 如果你用到了外部函数库里的函数,那么编译时就要带上它, \texttt{l} 代表link,
链接的意思。
\end{description}

「那么,我怎么知道我到底要链接哪个函数库呢」?答案很简单,「看手册」。比如说,
我在程序里调用了 \texttt{pthread\_create()} 函数用来产生一个新的线程,那么,
显然你该看看 \texttt{pthread\_create()} 的手册,具体了解一下这个函数的应用细节。
\begin{verbatim}
man pthread_create
\end{verbatim}
手册的前几行如下:
\begin{verbatim}
NAME
   pthread_create - create a new thread

SYNOPSIS
   #include <pthread.h>

   int pthread_create(pthread_t *thread, const pthread_attr_t *attr,
                      void *(*start_routine) (void *), void *arg);

   Compile and link with -pthread.
\end{verbatim}
它告诉你
\begin{enumerate}
\item 在程序中一定要有 \texttt{\#include <pthread.h>};
\item 从函数原型你知道
\begin{enumerate}
\item \texttt{pthread\_create()} 一定要返回一个 \texttt{int};
\item 调用这个函数必须提供4个指针类型的参数。
\end{enumerate}
\item 上面的最后一行 \texttt{Compile and link with -pthread}, 明确告诉你编译的时候要带上
\texttt{pthread} 选项。
\end{enumerate}

养成看手册的习惯,这可是程序员的基本功。

关于Linux平台上的C开发环境,我们简单介绍了编辑器Emacs,和编译器gcc。掌握了这两样神器,
你就是个相当有前途的程序员了。另外还剩下一个调试神器gdb我们没有介绍。
当年我的老师这样对我说,「像你这样两三百行的小程序,最好不要用调试器,静下心来,
一行一行地读你自己的代码,用你自己的大脑来找出程序中的bug,这是对你最好的训练」。
Happy hacking!

\end{document}