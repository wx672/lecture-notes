% Created 2020-02-22 Sat 19:44
% Intended LaTeX compiler: pdflatex
\documentclass{wx672ctexart} [NO-DEFAULT-PACKAGES] \usepackage{wx672hyperref}
\usepackage{amsmath}
\usepackage{amsfonts}
\usepackage{amssymb}
\usepackage{graphicx}
\pagestyle{plain}
\usepackage{wx672minted}


\usepackage[utf8]{inputenc}
\usepackage[T1]{fontenc}
\usepackage{graphicx}
\usepackage{grffile}
\usepackage{longtable}
\usepackage{wrapfig}
\usepackage{rotating}
\usepackage[normalem]{ulem}
\usepackage{amsmath}
\usepackage{textcomp}
\usepackage{amssymb}
\usepackage{capt-of}
\usepackage{hyperref}
\author{王晓林}
\date{\today}
\title{课堂讲稿}
\hypersetup{
 pdfauthor={王晓林},
 pdftitle={课堂讲稿},
 pdfkeywords={},
 pdfsubject={},
 pdfcreator={Emacs 26.1 (Org mode 9.3.2)}, 
 pdflang={Cn}}
\begin{document}

\maketitle
\tableofcontents


\section{\textit{[2020-02-20 Thu] } 老生常谈}
\label{sec:orga552b0f}

我提供的讲义、幻灯片、参考书等所有资料都是英文的,这有几方面原因。首先,英文真的很重要。怎
么算是重要?我们来举个例子,假设我现在给你留一个作业,比如写一个简单的system call吧。面对
这个作业,你会发现有诸多困难要克服:
\begin{enumerate}
\item 不知道什么是system call;(缺乏操作系统的相关知识)
\item C编程啊?不会;
\item 你说的这个编辑器(Emacs, Vim)我也没用过;
\item 没标准答案?要我上网搜什么?
\item 你提供的参考资料都是英文的?
\item 这作业太难了……(我没这份耐心)
\end{enumerate}

也就是说,想完成作业的话,你要克服上面这6个困难才行。而且,上面这6条,是我按它们的重要程度
排好了顺序的。其中最重要的是“耐心”,最不重要的是“操作系统知识”。为什么?因为,你们敷衍完这
门课之后,恐怕没有谁会去搞操作系统开发。可是,无论你将来做什么,只要你想稍微做出点样子,你
都必须要有“耐心”。
\begin{enumerate}
\item 耐心,无论干什么,都需要;
\item 英文,只要想找个好工作,就需要;
\item 上网搜,也就是科研能力,如果你将来搞技术工作,那么肯定需要;
\item 编辑器,也就是软件工具,只要你用电脑就需要;
\item 编程,只有程序员才需要;
\item 操作系统,只有搞系统开发才需要。
\end{enumerate}

也就是说,最通用的才最有用。尽管你以后不搞系统开发,甚至都不搞技术,但既然选了这门课,总该
有点收获,那就借着这门课的机会,补习一下你的英文吧。如果你打算将来吃技术饭,那么英文就更加
重要了。现代IT技术,都是西方人搞出来的,低头看看键盘,上面就没有一个中国字,不老老实实把英
文学好,你的专业技术也不会有什么前途。

面对这份作业,我相信大家完成作业的速度和质量也会大不一样。通常,好同学会完成得又快又好,为
什么?因为学习有“捷径”呗。好同学的捷径是什么?是这样:
\begin{enumerate}
\item 他做事的耐心早就锻炼出来了;
\item 他的英文已经学得不错了;
\item 他经常用Google查英文资料,而不是百度中文资料(没错,google比百度强太多了;英文技术资料
的数量和质量都比中文的强);
\item 他Linux用得很熟;
\item 他C编程也不错;
\item 他唯一需要现学的就是一点操作系统知识。
\end{enumerate}

也就是说,貌似你们是同时开始做作业,同时开始克服这6个困难,但实际上,他已经提前把5个困难克
服掉了,而你是从现在才开始启动,所以他当然比你做得快、做得好。所谓捷径,无非是笨鸟先飞罢了。

「也不全是吧,老师,那孩子真的比我聪明」。聪明不重要,至少在本科学习阶段,真的不重要。这不
是说聪明不好,就好比电脑的CPU,当然是越快越好。但只是硬件好,软件很烂,电脑也做不出什么正
经事。如果你真感觉聪明不足,那么可以学点不需要聪明的东西,比如英文。英国的傻子都比我们英
文好,不是吗?可你的英文为什么这么烂?很简单,因为你懒。

「英国的傻子有良好的“语境”」。那么,你的高中同学英语都和你一样烂吗?和你同龄的年轻人,他们
考进了清华、北大、云大……英语显然比你强。所以,别扯什么“语境”的淡了,懒就是懒。发现问题,正
视问题,才可能解决问题。真想学好,就好好学。学习是需要耐心的,像长跑一样,只有气喘吁吁、
大汗淋漓,才算是锻炼,才能锻炼出强健的体魄。

\section{\textit{[2020-02-21 Fri] } 关于作业}
\label{sec:orgfb40ea9}

下周一,我应该可以收到你们的作业了吧?「还什么都没讲,怎么做作业啊?」因为我布置的作业很简
单,就是每周问一个问题而已。因为我还什么都没讲,你还什么都没学,所以你满脑子都应该是问题,
问一个出来有什么难的?话虽如此,但提问有提问的规矩。在第4页幻灯片上,我把规矩列出来了。提
问要有条有理:
\begin{enumerate}
\item 我要做一件什么事情?
\item 我是怎样做的?(也就是步骤)
\item 做到哪一步的时候,我期待看到怎样的结果,可实际上看到的却是……
\item 于是我尝试了如此种种若干办法去解决它;
\item 我在这个问题上花费了多少小时?
\end{enumerate}

很显然,这个作业的意义就在于“做事情”。如果你能保证每周做一件稍微像样的事情,花费若干小时真
正去克服一些困难,那么你在毕业的时候就可以傲视身边这群懒蛋了。“傲视懒蛋”,这真的算不上有出
息,但这至少是第一步。想有出息,就要时刻提醒自己「我未来的竞争对手绝不是身边这些比我更懒的
游戏帅哥、追剧美眉」。想有出息,就要去和校外比,和省外比,和国外比。

「可是,我真的想不到有什么正经事要做啊」。如果你打算有出息,只是对操作系统缺乏兴趣,那么没
关系,我说过了,操作系统是学习中最不重要的事情。你尽管去做你感兴趣的事情,然后在作业里告诉
我,你是怎样做的,花了多少小时,克服了哪些困难?更直白地说,只要你给我一个好印象,期末考试
是不成问题的。

「我还是不知道该做啥,你分配个任务让我做吧」。千万不要找我要任务,因为我对「这是老师让我写
的,那是家长逼我学的」深恶痛绝。为什么?因为这些都是“借口”,是为失败找的借口,是为敷衍了事、
推卸责任找的借口。扯淡的是,从幼儿园到大学,我们的教育一直在为你提供这些借口。换言之,我们
的教育一直在培养“敷衍了事、推卸责任、毫无担当”的人。只有“尊重个人权利、尊重个人自由”的教育
才能培养出“负责任、有胆当”的人。显然,全世界没有哪个国家100\%实现了这种理想化教育。但我们也
看到,欧美国家的孩子动手能力更强,更有团队精神。为什么?因为这些孩子从小就被鼓励去做他们自
己喜欢做的事情。既然是自己的事情,自然会认真去做。「我要做什么,我该怎样做」这样的问题对他
们来讲,是家常便饭。久而久之,就养成了认真、负责的做事习惯。

我一个人没本事改变教育现状,但至少在我的课堂上,我们可以尝试一下。现在我给你机会,自由自在
地去做自己想做的事情。我希望你珍惜它。

\section{\textit{[2020-02-21 Fri] } 幻灯片第7页,什么是操作系统?}
\label{sec:org7ca60d0}

我们还是按着幻灯片的顺序来上课吧。其实,大家心里都明白,上课和学习是两码事,就好像“做一天
和尚”和“撞一天钟”是两码事一样。来课堂不等于就学习,学习也未必非要上课。毕竟,在课堂上我能
告诉你的充其量就是「你该学些什么」,而真正的学习肯定是课下进行的。“修行”不是“撞钟”能取代的。

好了,我们现在步入正题。前面说了,正题(也就是操作系统知识)并不重要,如果你真要学习的话,
千万不要绕过前面五个困难,直奔第六个,也就是最不重要的操作系统。你应该静下心来,克服幻灯片
里的每一个生词,认真理解这张幻灯片要表达的意思,这需要你去查阅参考书里的相关章节。然后,在
你认为「我终于搞明白了」的时候,把书合上,把幻灯片关掉,然后用自己的话把幻灯片里的内容复述
出来。没错,这才叫“学习”。

好了,假装你还没被吓跑,我开始讲第7张幻灯片,What's an OS? 第一句说,OS就是当你网购一个OS
的时候,人家寄给你的东西,那肯定就是OS。没错,这话很正确啊,虽然只是个玩笑,但并非毫无意义。
起码这句话让我意识到,要和流氓正经讲道理该有多困难。人家只要回你一句这么“正确”的话,就能噎得
你想撞墙。人嘴两张皮,人嘴是多么邪恶的两张皮啊!所以说,道理永远是讲给“讲道理的人”听的。对
于不讲道理的人呢?用他听得懂的语言去教训他!说白了,给他不及格呗。

好了,再看第二句,OS是从一开机就开始跑,直到关机(或者死机,如果你用Windows的话)才会结束
的程序。这句话算是很讲道理了。这的确是OS的重要特征之一,可以算是操作系统定义的一部分了。

第三句,它是资源的管理者。那么,什么是资源?电脑诞生之前,资源这个词就存在了。水、土地、煤
炭、石油、空气、人、动物……貌似没有什么东西不能被当作资源。没错,就连“垃圾”也可以被认为是
“放错了地方的资源”。但通常,当我们谈到资源的时候,“空气”和“垃圾”都不太容易被想到,为什么?
因为它们不“紧缺”。当我们谈资源的时候,通常是在谈那些“大家都想抢的东西”。

回到电脑里面来,也一样,everything is a resource,但大家(众多进程)无时无刻不想抢的东西就
两样,一是time,时间;二是space,空间。和时间相关的资源就是CPU,和空间相关的就是内存。好了,
现在打开终端,跟我学一个小命令:

\begin{minted}[mathescape=true,linenos=true,numbersep=5pt,frame=lines,framesep=2mm]{sh}
ps aux | wc -l
\end{minted}

用这条命令可以数出你的电脑里正运行着多少个进程。我的是191个。而我的CPU是8核,也就是说,191
个进程要抢着用8个CPU,没人管肯定是不行的吧。任何一个程序要运行的话,先要把它加载到内存中去,
而我的内存只有8G,如果不够用怎么办呢?操作系统最重要的工作就是负责CPU和内存的分配与管理,
它是电脑里的resource manager。

第四句,它是个控制程序。假如那191个进程里,谁和谁发生了不愉快,比如一个流氓进程非要往我的
地址空间里写东西,那么,操作系统肯定要出手干预,或者在流氓得手之后,帮它洗地。每当此时,你
心爱的Windows就会呈现出著名的Blue screen of death。Unix没这么夸张,它通常会在你启动流
氓程序的瞬间,就告诉你“Segmentation fault”,也就是所谓“段错误”,这通常是访问非法内存地址造
成的。

最后一句,关于操作系统,并不存在一个放之四海而皆准的定义。为什么呢?因为有个问题,操作系统除
了要一直转着不停,除了要管理资源,除了要洗地之外,还应该有那些功能呢?这就见仁见智了。举例
而言吧,苹果和微软都把图形界面放到了操作系统内核里,因为这样“打开窗户的速度”更快;而其它的
Unix,还有Linux,内核里都不包含图形界面,因为并不是所有人都需要图形界面,比如服务器就不需
要图形界面。庞大的图形界面会给内核带来更多的bug,降低系统的稳定性、安全性、和效率。因此,
专业的服务器都不太会选用Windows或者苹果系统,毕竟Windows和苹果的设计初衷,就是面向个人电脑
用户。服务器的话,基本上都是Linux和Unix的天下(除了苹果,虽然它也算是Unix)。Linux、
Windows、苹果,这些都是通用型操作系统,这世界上还有很多专用型系统呢,比如用于流水线控制的
单板机。不同的系统,面向不同的工作场景,有不同的设计需求,所以,操作系统该如何定义呢?还是
见仁见智吧。

好了,现在关掉幻灯片,把你的理解复述出来吧,最好用英文。

\section{\textit{[2020-02-22 Sat] } 幻灯片第8页,OS的功能模块}
\label{sec:org8df8309}

第8页上这张图,简化一下,就是下面这副样子:

\verbatimfont{\dejavu}
\begin{verbatim}
┌──────────┐
│  Users   │      ┌───────────────────┐
├──────────┤      │      ┌─────────┐  │
│  APPs    │      │      │    ┌──┐ │  │
├──────────┤      │ APPs │ OS │HW│ │  │
│   OS     │      │      │    └──┘ │  │
├──────────┤      │      └─────────┘  │
│ Hardware │      └───────────────────┘
└──────────┘
\end{verbatim}

左、右两张图是一回事,表达的是同一个意思,我们,生物意义上的人,从来不直接使用操作系统,我
们只使用应用程序,应用程序才会去和操作系统打交道。任何应用程序如果想使用硬件(比如键盘、鼠
标、显示器)的话,都要向操作系统发出请求,然后操作系统帮你把键盘输入的字符显示到屏幕上。

「干吗这么费事?没有操作系统不行吗?」,其实前面我们已经回答过这个问题了,电脑里的资源(比
如键盘、鼠标、显示器)很紧缺,若干进程都要抢着用,所以必然需要有人来维持一下秩序。换言之,
如果你的系统里只跑一个程序的话,也就是所谓“单任务系统”,那么操作系统的确就显得多余了。

把上图中的OS放到显微镜下,看到的就是第8页的幻灯片。片中上下两条虚线之间就是我们最关心的部
分,操作系统。它是个软件,可以很庞大而复杂,也可以小巧而简单,因设计需求的不同而变化。通常,
讲课的时候,都选庞大而复杂的来说,而具体编程实现的时候,都是怎么简单怎么来。为什么?因为说
起来容易,做起来难呗。

无论如何,一个通常意义的操作系统,它里面会有进程控制、内存管理、文件系统、输入输出等功能模
块。教科书上,一般也是着重讲这几个模块。我们16周的课,通常只能讲完前三个,输入输出就靠自学
了。本来嘛,上课也就能告诉你“该学什么”,不是吗?

现在来看看,APPs是如何向OS请求服务的?APPs和OS都是软件,软件之间怎么相互通信,或者说传递数
据啊?函数调用呗。所以说,APPs只要调用OS提供的函数,就可以把信息发送给OS了。每个操作系统都
为用户进程提供了一整套函数,或者说一个“函数库”,这个函数库就是图中的system call interface。
Linux的库比较小,里面有大约400个syscall。Meanwhile,Windows的库里有4000多个。是大点好,还
是小点好呢?其实很好回答,就问问你自己,「做为一个程序员,我是愿意看一本400页的手册呢,还
是看4000页的?」。分手吧,她真的太胖了!而且,不止是手册的厚薄问题,软件的代码量越大,
bug数量必然就越多。这可是操作系统啊,它蓝屏,我一点都不觉得意外。

再注意一下,图中的syscall interface有两条路通往用户程序。一条路是直接的;另一条是间接的,
要通过libraries(函数库)。其实,这个函数库,差不多就是专指C函数库。它不在OS里面,它在用户
层(user level),是一个普通用户就能随意安装、卸载、替换的软件包。既然有一条直接的路径,为
什么还需要它?有两个主要原因:
\begin{enumerate}
\item 跨平台。前面说了,不同的OS提供了不同的syscall函数库。那么,如果你在windows平台写了个程
序,里面自然要用到Windows提供的syscall,比如说 \texttt{CreateProcess()} ,用来产生一个子进程。
写好了之后,你编译、运行,一切良好。于是,你把它拿到Linux平台,直接运行肯定是不行的。换
了平台,要先编译。「完蛋了!在Windows上一切都好好的,为什么到Linux上就编译通不过?Linux
太难用了!」没错,Linux对你不友好,也只是对“你”不友好。为什么?因为(此处略去500脏字)。

想想看,Windows提供了4000多个syscall,其中包括 \texttt{CreateProcess()}; 而Linux只提供了400个,
你保证它也有 \texttt{CreateProcess()} 吗?Too simple, sometimes naive!在Linux平台,想要产生子
进程的话,你要调用 \texttt{fork()} 。于是,累了,你不得不把程序中用到的成百上千个Windows
syscall都替换成相应的Linux syscall!(我相信你的脏字也会很多的)结论,由于你直接调用了
OS提供的syscall,导致你的软件可移植性极差,根本不跨平台。

如果你不直接调用syscall,而是“走弯路”,调用Library里的函数,生活就美好多了。比如说,我
们最常用的Library是POSIX提供的LIBC,它既有Windows版,也有Linux版。于是,你只要在Windows
和Linux上都装好POSIX LIBC,编程的时候调用里面的函数,让它去帮你调用底层的syscall,就没
问题了。
\item 方便。函数库存在的意义,不止是把syscall包装一下,以便于你的软件可以跨平台。它还提供了很
多广受用户欢迎,而syscall没有提供的功能。比如说 \texttt{printf()} 吧,
\begin{minted}[mathescape=true,linenos=true,numbersep=5pt,frame=lines,framesep=2mm]{c}
printf("Hello, world!\n");
\end{minted}
说简单了,它的功能就是“屏幕输出”。但直接调用syscall(不考虑跨平台的话) \texttt{write()} 也可
以实现屏幕输出。实际上, \texttt{printf()} 最终就是通过调用 \texttt{write()} 来完成屏幕输出的。
\begin{minted}[mathescape=true,linenos=true,numbersep=5pt,frame=lines,framesep=2mm]{c}
write(1, "Hello, world!\n", 14);
\end{minted}

这么绕弯的好处是什么呢? \texttt{printf()} 提供了带格式的输出,而 \texttt{write()} 不行。为什么
\texttt{write()} 不提供格式支持呢?原因既浅显又重要,Linux的设计者认为,OS只应该提供“不得不提供
的功能”,所有非必须的功能都应该由用户层软件提供。这也是Linux内核不提供GUI(图形界面)支
持的原因。这也是它更适合用来做服务器的原因。
\end{enumerate}

再来看看图中的这两条路,边上都有一个单词,trap,当名词用的时候被翻译成“陷阱”。但是,在计算
机专业,它经常被用做动词,是“触发”的意思。一个陷阱挖好了,只要别踩它,什么事都没有。一旦你
踩上去,就会触发你的噩梦……同样,用户程序以调用函数的方式触发操作系统的某个功能,所以这里用
trap一词。

后面的课程里,我们还会经常接触system call。课上用到的例子,自己去尝试一下吧。
\begin{itemize}
\item \url{https://cs6.swfu.edu.cn/\~wx672/lecture\_notes/os/src.tgz}
\item \url{https://cs6.swfu.edu.cn/\~wx672/lecture\_notes/linux-app/src.tgz}
\end{itemize}

\section{\textit{[2020-02-22 Sat] } 幻灯片第9页,选个OS}
\label{sec:orgf1caf62}

这张画在技术圈里还是挺著名的,而且有好几个版本。不论哪个版本,都是个笑话。能看懂的,就笑呗;
看不懂的呢,“加油!不哭!”,画一个自己的版本,咱们再笑呗。

「我是学技术的,到底该用哪个系统啊?」。肯定不该用Windows。为什么?
\begin{enumerate}
\item 学外科的,总得解剖过尸体吧;学操作系统的,总得解剖过操作系统吧。Windows根本就不开源,你
看不到它里面的东西,所以无法拿它来解剖、学习。如果你真的有本事把它解剖了,那么你就要收
到法院的传票了,因为这犯法。
\item Windows是要花钱买的,不便宜,而且在它的版权声明里还罗列着种种限制,包括不能修改、不能送
人、不能私自买卖……而Linux是自由软件,你可以自由获取、自由使用、自由学习、自由送人、自由
买卖、自由修改…… 你一个穷学生,不该为爹妈省点钱吗?而且,这绝不只是钱的事情,「自由」才
是最重要的。武汉的李文亮大夫用生命告诉我们「言论自由关系到每个人的生命安全」。西谚有云
「Live free or die」,先贤译之为「不自由,毋宁死」。看看李大夫,我的翻译是「不自由,真
要命」。
\item 「可是我身边的同学和老师,他们都用Windows啊」。没错,而且我劝过他们了,就像我劝你一样。
他们会说「自由了,早晚不也得死」,「不觉不自由,也就自由了」,「现在不挺舒服的,自由又
不能当饭吃」,「干吗非得跟人不一样啊」……的确,各有各的活法,至少我也要尊重他们「选择不
自由」的自由。而且,被这些人环绕着,你会感觉自己才是个怪物。又想起一句西谚「Birds born
in a cage think flying is an illness」,笼子里长大的鸟会认为飞翔是一种病。你的翅膀还在
吗?愿意做一个飞翔的“怪物”吗?还是收起翅膀,和他们一起做“正常人”呢?西洋人又说了,「Why
fit in, when you were born to stand out?」,一个生来就该与众不同、卓尔不群的人,干吗又
要去随大流呢?
\item 用Linux的同学,毕业后的前景都很好。这并不意外吧。首先,肯用Linux的同学都普遍比较好学;
其次,用Windows的人太多,竞争自然就大。竞争大,老板就会把工资压得很低。而用Linux的人少,
所以需求缺口巨大,工作自然要容易找,工资也相对较高。
\end{enumerate}

好了,晓之以理、动之以情、诱之以利,我都做到了。「好吧,我试试,那我到底该装哪款Linux呢?」。
很简单,你周围的人用哪个,你就用哪个。为什么?容易得到帮助呗。所以,最好是我用哪个,你就用
哪个。跟着我的安装指导一步步走,应该不会很费事。
\begin{itemize}
\item \url{https://cs6.swfu.edu.cn/\~wx672/debian-install/install.html}
\end{itemize}
\end{document}