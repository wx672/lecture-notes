% Created 2015-12-06 Sun 21:50
\documentclass[11pt]{article}
\usepackage[utf8]{inputenc}
\usepackage[T1]{fontenc}
\usepackage{fixltx2e}
\usepackage{graphicx}
\usepackage{longtable}
\usepackage{float}
\usepackage{wrapfig}
\usepackage{rotating}
\usepackage[normalem]{ulem}
\usepackage{amsmath}
\usepackage{textcomp}
\usepackage{marvosym}
\usepackage{wasysym}
\usepackage{amssymb}
\usepackage{hyperref}
\tolerance=1000
\usepackage{fullpage}
\usepackage{indentfirst}
\usepackage[indentafter,pagestyles]{titlesec}
\usepackage{minted}
\usepackage{xltxtra}
\usepackage{xeCJK}
\tolerance=1000
\setlength{\parindent}{2.5em}
\setmainfont{DejaVu Serif}
\setsansfont{DejaVu Sans}
\setmonofont{DejaVu Sans Mono}
\setCJKmainfont[BoldFont={WenQuanYi Zen Hei}]{SimSun}
\setCJKfamilyfont{hei}{WenQuanYi Zen Hei}
\setCJKfamilyfont{song}{SimSun}
\XeTeXlinebreaklocale "zh"
\XeTeXlinebreakskip = 0pt plus 1pt
\graphicspath{{./figs/}{../figs/}{./}{../}}
\renewcommand{\contentsname}{目录}
\renewcommand{\listfigurename}{插图目录}
\renewcommand{\listtablename}{表格目录}
\renewcommand{\abstractname}{摘要}
\renewcommand{\appendixname}{附录}
\renewcommand{\indexname}{索引}
\renewcommand{\figurename}{图}
\renewcommand{\tablename}{表}
\author{王晓林}
\date{2015-06-20}
\title{《操作系统原理》教案}
\hypersetup{
  pdfkeywords={},
  pdfsubject={},
  pdfcreator={Emacs 24.5.1 (Org mode 8.2.10)}}
\begin{document}

\maketitle
\tableofcontents


\clearpage
\begin{itemize}
\item 课程名称: 操作系统原理
\item 面向专业: 计算机相关本科专业
\item 教材参考:
\begin{enumerate}
\item A.S. Tanenbaum. Modern Operating Systems, 3e. Pearson Prentice Hall, 2008.
\item Silberschatz, Galvin, and Gagne. Operating System Concepts Essentials. John Wiley \& Sons, 2011.
\end{enumerate}
\item 授课学期:大三第一学期
\item 任课教师:王晓林
\item 编写时间:\textit{[2016-06-20]}
\end{itemize}
\section{操作系统简介}
\label{sec-1}
\subsection{教学目标及基本要求}
\label{sec-1-1}
了解什么是操作系统,及相关基本概念。
\subsection{教学内容及学时分配}
\label{sec-1-2}
\begin{itemize}
\item \href{./slides/intro-a.pdf}{讲义}
\item \href{./slides/intro-b.pdf}{幻灯片}
\item \href{./lab.html#sec-3}{作业}
\end{itemize}
\subsubsection{What's an OS? (0.5h)}
\label{sec-1-2-1}
\paragraph{教学内容(具体到知识点)}
\label{sec-1-2-1-1}
\begin{itemize}
\item Resource manager
\item Control program
\item System goals -- convenient vs. efficient
\item History of OS
\item Various OSes
\end{itemize}
\paragraph{教学方式(手段)}
\label{sec-1-2-1-2}
理论 + 实验
\subsubsection{OS services (0.5h)}
\label{sec-1-2-2}
\paragraph{教学内容(具体到知识点)}
\label{sec-1-2-2-1}
\begin{itemize}
\item Helping the user
\item Keeping the system efficient
\end{itemize}
\paragraph{教学方式(手段)}
\label{sec-1-2-2-2}
理论 + 实验
\subsubsection{Bootstrapping (1h)}
\label{sec-1-2-3}
\paragraph{教学内容(具体到知识点)}
\label{sec-1-2-3-1}
\begin{itemize}
\item Intel x86 bootstrapping
\end{itemize}
\paragraph{教学方式(手段)}
\label{sec-1-2-3-2}
理论 + 实验
\subsubsection{Hardware (0.5h)}
\label{sec-1-2-4}
\paragraph{教学内容(具体到知识点)}
\label{sec-1-2-4-1}
\begin{itemize}
\item CPU working cycle
\item CPU registers
\item System bus
\item Controllers and Peripherals
\end{itemize}
\paragraph{教学方式(手段)}
\label{sec-1-2-4-2}
理论 + 实验
\subsubsection{Interrupt (1h)}
\label{sec-1-2-5}
\paragraph{教学内容(具体到知识点)}
\label{sec-1-2-5-1}
\begin{itemize}
\item Why interrupt?
\item Interrupt timeline
\item Concepts: 
\begin{itemize}
\item hardware interrupt
\item software interrupt
\item trap
\item interrupt vector
\end{itemize}
\item Interrupt processing
\end{itemize}
\paragraph{教学方式(手段)}
\label{sec-1-2-5-2}
理论 + 实验
\subsubsection{System calls (1h)}
\label{sec-1-2-6}
\paragraph{教学内容(具体到知识点)}
\label{sec-1-2-6-1}
\begin{itemize}
\item What's a system call?
\item How a system call works?
\item Various system calls
\end{itemize}
\paragraph{教学方式(手段)}
\label{sec-1-2-6-2}
理论 + 实验
\subsection{重点和难点}
\label{sec-1-3}
\begin{itemize}
\item Bootstrapping
\item Interrupt
\item System calls
\end{itemize}
\subsection{深化和拓宽}
\label{sec-1-4}
暂无
\subsection{教学方式(手段)及教学过程中应注意的问题}
\label{sec-1-5}
\begin{itemize}
\item 理论 + 实验
\end{itemize}
\section{进程与线程}
\label{sec-2}
\subsection{教学目标及基本要求}
\label{sec-2-1}
了解进程和线程相关的基本概念。
\subsection{教学内容及学时分配}
\label{sec-2-2}
\begin{itemize}
\item \href{./slides/process-thread-a.pdf}{讲义}
\item \href{./slides/process-thread-b.pdf}{幻灯片}
\item \href{./lab.html#sec-4}{作业}
\end{itemize}
\subsubsection{What's a process? (.5h)}
\label{sec-2-2-1}
\paragraph{教学内容(具体到知识点)}
\label{sec-2-2-1-1}
\begin{itemize}
\item logical view of a process
\end{itemize}
\paragraph{教学方式(手段)}
\label{sec-2-2-1-2}
理论 + 实验
\subsubsection{Process creation (1h)}
\label{sec-2-2-2}
\paragraph{教学内容(具体到知识点)}
\label{sec-2-2-2-1}
\begin{itemize}
\item fork(), exec(), wait(), exit()
\end{itemize}
\paragraph{教学方式(手段)}
\label{sec-2-2-2-2}
理论 + 实验
\subsubsection{Process state (.5h)}
\label{sec-2-2-3}
\paragraph{教学内容(具体到知识点)}
\label{sec-2-2-3-1}
\begin{itemize}
\item running, blocked, ready
\end{itemize}
\paragraph{教学方式(手段)}
\label{sec-2-2-3-2}
理论 + 实验
\subsubsection{Process Control Block(PCB) (1h)}
\label{sec-2-2-4}
\paragraph{教学内容(具体到知识点)}
\label{sec-2-2-4-1}
\begin{itemize}
\item How to describe a process?
\item \texttt{task\_struct} in Linux
\end{itemize}
\paragraph{教学方式(手段)}
\label{sec-2-2-4-2}
理论 + 实验
\subsubsection{CPU switch from process to process (.5h)}
\label{sec-2-2-5}
\paragraph{教学内容(具体到知识点)}
\label{sec-2-2-5-1}
\begin{itemize}
\item The OS role in process switch
\end{itemize}
\paragraph{教学方式(手段)}
\label{sec-2-2-5-2}
\subsubsection{Processes vs. threads (.5h)}
\label{sec-2-2-6}
\paragraph{教学内容(具体到知识点)}
\label{sec-2-2-6-1}
\begin{itemize}
\item A process is a unit of resource ownership
\item A thread is a unit of scheduling
\end{itemize}
\paragraph{教学方式(手段)}
\label{sec-2-2-6-2}
理论 + 实验
\subsubsection{Why thread? (.5)}
\label{sec-2-2-7}
\paragraph{教学内容(具体到知识点)}
\label{sec-2-2-7-1}
\begin{itemize}
\item Advantages
\end{itemize}
\paragraph{教学方式(手段)}
\label{sec-2-2-7-2}
理论 + 实验
\subsubsection{Thread characteristics (.5)}
\label{sec-2-2-8}
\paragraph{教学内容(具体到知识点)}
\label{sec-2-2-8-1}
\begin{itemize}
\item Thread state
\item A thread has its own stack
\end{itemize}
\paragraph{教学方式(手段)}
\label{sec-2-2-8-2}
理论 + 实验
\subsubsection{Thread operation and POSIX threads (1h)}
\label{sec-2-2-9}
\paragraph{教学内容(具体到知识点)}
\label{sec-2-2-9-1}
\begin{itemize}
\item Thread library calls
\item POSIX threads examples
\end{itemize}
\paragraph{教学方式(手段)}
\label{sec-2-2-9-2}
理论 + 实验
\subsubsection{User-level threads vs. kernel-level threads (.5h)}
\label{sec-2-2-10}
\paragraph{教学内容(具体到知识点)}
\label{sec-2-2-10-1}
\begin{itemize}
\item pros and cons
\item Hybrid implementations
\item Programming complications
\end{itemize}
\paragraph{教学方式(手段)}
\label{sec-2-2-10-2}
理论 + 实验
\subsubsection{Linux threads (1h)}
\label{sec-2-2-11}
\paragraph{教学内容(具体到知识点)}
\label{sec-2-2-11-1}
\begin{itemize}
\item clone()
\end{itemize}
\paragraph{教学方式(手段)}
\label{sec-2-2-11-2}
理论 + 实验
\subsection{重点和难点}
\label{sec-2-3}
\begin{itemize}
\item 进程的一生
\item 线程的一生
\end{itemize}
\subsection{深化和拓宽}
\label{sec-2-4}
暂无
\subsection{教学方式(手段)及教学过程中应注意的问题}
\label{sec-2-5}
理论 + 实验
\subsection{思考题和习题}
\label{sec-2-6}
\section{进程间同步}
\label{sec-3}
\subsection{教学目标及基本要求}
\label{sec-3-1}
了解进程间协调的基本概念和方法。
\subsection{教学内容及学时分配}
\label{sec-3-2}
\begin{itemize}
\item \href{./slides/process-sync-a.pdf}{讲义}
\item \href{./slides/process-sync-b.pdf}{幻灯片}
\item \href{./lab.html#sec-4-3}{作业}
\end{itemize}
\subsubsection{Inter-process communication (.5h)}
\label{sec-3-2-1}
\paragraph{教学内容(具体到知识点)}
\label{sec-3-2-1-1}
\begin{itemize}
\item IPC issues
\item Two models of IPC
\end{itemize}
\paragraph{教学方式(手段)}
\label{sec-3-2-1-2}
理论 + 实验
\subsubsection{Shared memory (1h)}
\label{sec-3-2-2}
\paragraph{教学内容(具体到知识点)}
\label{sec-3-2-2-1}
\begin{itemize}
\item Producer-consumer problem
\end{itemize}
\paragraph{教学方式(手段)}
\label{sec-3-2-2-2}
理论 + 实验
\subsubsection{Race condition and mutual exclusion (2h)}
\label{sec-3-2-3}
\paragraph{教学内容(具体到知识点)}
\label{sec-3-2-3-1}
\begin{itemize}
\item Race scenarios
\item Critical regions
\item Algorithms: 
\begin{itemize}
\item busy-waiting
\begin{itemize}
\item disable interrupts
\item Peterson's algorithm
\item Hardware solution --- TSL instruction
\end{itemize}
\item No busy-waiting
\begin{itemize}
\item sleep/wake up
\end{itemize}
\end{itemize}
\end{itemize}
\paragraph{教学方式(手段)}
\label{sec-3-2-3-2}
理论 + 实验
\subsubsection{Semaphores (2h)}
\label{sec-3-2-4}
\paragraph{教学内容(具体到知识点)}
\label{sec-3-2-4-1}
\begin{itemize}
\item What's a semaphore?
\item Why semaphore?
\item How to use semaphore?
\item Mutex
\item Barriers
\item Examples
\end{itemize}
\paragraph{教学方式(手段)}
\label{sec-3-2-4-2}
理论 + 实验
\subsubsection{Monitors (1h)}
\label{sec-3-2-5}
\paragraph{教学内容(具体到知识点)}
\label{sec-3-2-5-1}
\begin{itemize}
\item What's a monitor?
\item Examples
\end{itemize}
\paragraph{教学方式(手段)}
\label{sec-3-2-5-2}
理论 + 实验
\subsubsection{Message passing (1h)}
\label{sec-3-2-6}
\paragraph{教学内容(具体到知识点)}
\label{sec-3-2-6-1}
\begin{itemize}
\item What's messaging?
\item Examples
\end{itemize}
\paragraph{教学方式(手段)}
\label{sec-3-2-6-2}
理论 + 实验
\paragraph{师生活动设计}
\label{sec-3-2-6-3}
\subsubsection{Classical IPC problems (2h)}
\label{sec-3-2-7}
\paragraph{教学内容(具体到知识点)}
\label{sec-3-2-7-1}
\begin{itemize}
\item The dining philosophers problem
\item The readers-writers problem
\item The sleeping barber problem
\end{itemize}
\paragraph{教学方式(手段)}
\label{sec-3-2-7-2}
理论 + 实验
\subsection{重点和难点}
\label{sec-3-3}
\begin{itemize}
\item Semaphores
\item Classical IPC problems
\end{itemize}
\subsection{深化和拓宽}
\label{sec-3-4}
暂无
\subsection{教学方式(手段)及教学过程中应注意的问题}
\label{sec-3-5}
理论 + 实验
\subsection{思考题和习题}
\label{sec-3-6}
\section{CPU调度}
\label{sec-4}
\subsection{教学目标及基本要求}
\label{sec-4-1}
了解进程调度的基本算法
\subsection{教学内容及学时分配}
\label{sec-4-2}
\begin{itemize}
\item \href{./slides/cpu-sched-a.pdf}{讲义}
\item \href{./slides/cpu-sched-b.pdf}{幻灯片}
\end{itemize}
\subsubsection{Scheduling introduction (1h)}
\label{sec-4-2-1}
\paragraph{教学内容(具体到知识点)}
\label{sec-4-2-1-1}
\begin{itemize}
\item Process scheduling queues
\item Different system has different scheduling algorithm
\item Process behavior
\item Process classification
\item Process schedulers
\end{itemize}
\paragraph{教学方式(手段)}
\label{sec-4-2-1-2}
理论 + 实验
\subsubsection{Scheduling algorithms (1h)}
\label{sec-4-2-2}
\paragraph{教学内容(具体到知识点)}
\label{sec-4-2-2-1}
\begin{itemize}
\item Scheduling in batch systems
\item Scheduling in interactive systems
\end{itemize}
\paragraph{教学方式(手段)}
\label{sec-4-2-2-2}
理论 + 实验
\subsubsection{Thread scheduling (.5h)}
\label{sec-4-2-3}
\paragraph{教学内容(具体到知识点)}
\label{sec-4-2-3-1}
\begin{itemize}
\item kernel-threads vs. user-threads
\end{itemize}
\paragraph{教学方式(手段)}
\label{sec-4-2-3-2}
理论 + 实验
\subsubsection{Linux scheduling (1h)}
\label{sec-4-2-4}
\paragraph{教学内容(具体到知识点)}
\label{sec-4-2-4-1}
\begin{itemize}
\item 140 priorities
\item O(1)
\item active array, expired array
\end{itemize}
\paragraph{教学方式(手段)}
\label{sec-4-2-4-2}
理论 + 实验
\subsection{重点和难点}
\label{sec-4-3}
\begin{itemize}
\item Scheduling algorithms
\end{itemize}
\subsection{深化和拓宽}
\label{sec-4-4}
暂无
\subsection{教学方式(手段)及教学过程中应注意的问题}
\label{sec-4-5}
理论 + 实验
\subsection{思考题和习题}
\label{sec-4-6}
\section{死锁}
\label{sec-5}
\subsection{教学目标及基本要求}
\label{sec-5-1}
了解死锁产生的原因和处理方法。
\subsection{教学内容及学时分配}
\label{sec-5-2}
\begin{itemize}
\item \href{./slides/deadlock-a.pdf}{讲义}
\item \href{./slides/deadlock-b.pdf}{幻灯片}
\end{itemize}
\subsubsection{Resources (.5h)}
\label{sec-5-2-1}
\paragraph{教学内容(具体到知识点)}
\label{sec-5-2-1-1}
\begin{itemize}
\item Processes need access to resources in reasonable order
\item Preemptable and non-preemptable resources
\end{itemize}
\paragraph{教学方式(手段)}
\label{sec-5-2-1-2}
理论 + 实验
\subsubsection{Introduction to deadlock (.5h)}
\label{sec-5-2-2}
\paragraph{教学内容(具体到知识点)}
\label{sec-5-2-2-1}
\begin{itemize}
\item four conditions
\end{itemize}
\paragraph{教学方式(手段)}
\label{sec-5-2-2-2}
理论 + 实验
\subsubsection{Deadlock modeling (.5h)}
\label{sec-5-2-3}
\paragraph{教学内容(具体到知识点)}
\label{sec-5-2-3-1}
\begin{itemize}
\item resource requirements graph
\end{itemize}
\paragraph{教学方式(手段)}
\label{sec-5-2-3-2}
理论 + 实验
\subsubsection{Dealing with deadlocks (2h)}
\label{sec-5-2-4}
\paragraph{教学内容(具体到知识点)}
\label{sec-5-2-4-1}
\begin{itemize}
\item Deadlock detection and recovery
\item Deadlock avoidance
\item Deadlock prevention
\item The ostrich algorithm
\end{itemize}
\paragraph{教学方式(手段)}
\label{sec-5-2-4-2}
理论 + 实验
\subsection{重点和难点}
\label{sec-5-3}
\begin{itemize}
\item 死锁处理方法
\end{itemize}
\subsection{深化和拓宽}
\label{sec-5-4}
暂无
\subsection{教学方式(手段)及教学过程中应注意的问题}
\label{sec-5-5}
理论 + 实验
\subsection{思考题和习题}
\label{sec-5-6}
\section{内存管理}
\label{sec-6}
\subsection{教学目标及基本要求}
\label{sec-6-1}
了解内存管理的相关概念
\subsection{教学内容及学时分配}
\label{sec-6-2}
\begin{itemize}
\item \href{./slides/mm-a.pdf}{讲义}
\item \href{./slides/mm-b.pdf}{幻灯片}
\item \href{./lab.html#sec-5}{作业}
\end{itemize}
\subsubsection{Real-mode vs. protected-mode memory management (1.5h)}
\label{sec-6-2-1}
\paragraph{教学内容(具体到知识点)}
\label{sec-6-2-1-1}
\begin{itemize}
\item Relocation problem
\item Process' memory is divided into logical segments
\item Memory allocation, who/when/how?
\item Swapping
\end{itemize}
\paragraph{教学方式(手段)}
\label{sec-6-2-1-2}
理论 + 实验
\subsubsection{Contiguous memory allocation (.5h)}
\label{sec-6-2-2}
\paragraph{教学内容(具体到知识点)}
\label{sec-6-2-2-1}
\begin{itemize}
\item first-fit, best-fit, worst-fit
\item internal-fragmentation, external-fragmentation
\end{itemize}
\paragraph{教学方式(手段)}
\label{sec-6-2-2-2}
理论 + 实验
\subsubsection{Virtual memory (5h)}
\label{sec-6-2-3}
\paragraph{教学内容(具体到知识点)}
\label{sec-6-2-3-1}
\begin{itemize}
\item Logical memory vs. physical memory
\item Paging
\begin{itemize}
\item Demand paging
\item Copy-on-write
\item Memory-mapped file
\item Page replacement algorithm
\item Allocation of frames
\item Thrashing and working set model
\end{itemize}
\item Segmentation
\end{itemize}
\paragraph{教学方式(手段)}
\label{sec-6-2-3-2}
理论 + 实验
\subsection{重点和难点}
\label{sec-6-3}
\begin{itemize}
\item Virtual memory
\end{itemize}
\subsection{深化和拓宽}
\label{sec-6-4}
暂无
\subsection{教学方式(手段)及教学过程中应注意的问题}
\label{sec-6-5}
理论 + 实验
\subsection{思考题和习题}
\label{sec-6-6}
\section{文件系统}
\label{sec-7}
\subsection{教学目标及基本要求}
\label{sec-7-1}
了解文件系统的基本工作原理
\subsection{教学内容及学时分配}
\label{sec-7-2}
\begin{itemize}
\item \href{./slides/fs-a.pdf}{讲义}
\item \href{./slides/fs-b.pdf}{幻灯片}
\item \href{./lab.html#sec-6}{作业}
\end{itemize}
\subsubsection{Files (1.5h)}
\label{sec-7-2-1}
\paragraph{教学内容(具体到知识点)}
\label{sec-7-2-1-1}
\begin{itemize}
\item What's a file?
\item File system design issues
\item File system models --- layered design
\item File attributes and types
\item File operations
\end{itemize}
\paragraph{教学方式(手段)}
\label{sec-7-2-1-2}
理论 + 实验
\subsubsection{Directories (1h)}
\label{sec-7-2-2}
\paragraph{教学内容(具体到知识点)}
\label{sec-7-2-2-1}
\begin{itemize}
\item Single level, multi-level directories
\item Operations
\end{itemize}
\paragraph{教学方式(手段)}
\label{sec-7-2-2-2}
理论 + 实验
\subsubsection{File system implementation (3h)}
\label{sec-7-2-3}
\paragraph{教学内容(具体到知识点)}
\label{sec-7-2-3-1}
\begin{itemize}
\item File system layout
\item Implementing files
\begin{itemize}
\item Linked-list allocation
\item Indexed allocation
\end{itemize}
\item Implementing directories
\item Shared files
\item Disk space management
\end{itemize}
\paragraph{教学方式(手段)}
\label{sec-7-2-3-2}
理论 + 实验
\subsubsection{Ext2 file system (1.5h)}
\label{sec-7-2-4}
\paragraph{教学内容(具体到知识点)}
\label{sec-7-2-4-1}
\begin{itemize}
\item Ext2 fs layout
\item Ext2 superblock
\item Ext2 inode
\item Ext2 directory
\end{itemize}
\paragraph{教学方式(手段)}
\label{sec-7-2-4-2}
理论 + 实验
\subsubsection{Virtual file system (1.5h)}
\label{sec-7-2-5}
\paragraph{教学内容(具体到知识点)}
\label{sec-7-2-5-1}
\begin{itemize}
\item Why VFS?
\item FS mounting
\item Linux VFS
\end{itemize}
\paragraph{教学方式(手段)}
\label{sec-7-2-5-2}
理论 + 实验
\subsection{重点和难点}
\label{sec-7-3}
\begin{itemize}
\item File system implementation
\item Ext2 FS
\end{itemize}
\subsection{深化和拓宽}
\label{sec-7-4}
暂无
\subsection{教学方式(手段)及教学过程中应注意的问题}
\label{sec-7-5}
理论 + 实验
\subsection{参考书目}
\label{sec-7-6}
\subsection{思考题和习题}
\label{sec-7-7}


\bibliographystyle{plain}
\bibliography{os}
% Emacs 24.5.1 (Org mode 8.2.10)
\end{document}