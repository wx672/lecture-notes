% Created 2015-12-06 Sun 21:48
\documentclass[11pt]{article}
\usepackage[utf8]{inputenc}
\usepackage[T1]{fontenc}
\usepackage{fixltx2e}
\usepackage{graphicx}
\usepackage{longtable}
\usepackage{float}
\usepackage{wrapfig}
\usepackage{rotating}
\usepackage[normalem]{ulem}
\usepackage{amsmath}
\usepackage{textcomp}
\usepackage{marvosym}
\usepackage{wasysym}
\usepackage{amssymb}
\usepackage{hyperref}
\tolerance=1000
\usepackage{fullpage}
\usepackage{indentfirst}
\usepackage[indentafter,pagestyles]{titlesec}
\usepackage{minted}
\usepackage{xltxtra}
\usepackage{xeCJK}
\tolerance=1000
\setlength{\parindent}{2.5em}
\setmainfont{DejaVu Serif}
\setsansfont{DejaVu Sans}
\setmonofont{DejaVu Sans Mono}
\setCJKmainfont[BoldFont={WenQuanYi Zen Hei}]{SimSun}
\setCJKfamilyfont{hei}{WenQuanYi Zen Hei}
\setCJKfamilyfont{song}{SimSun}
\XeTeXlinebreaklocale "zh"
\XeTeXlinebreakskip = 0pt plus 1pt
\graphicspath{{./figs/}{../figs/}{./}{../}}
\renewcommand{\contentsname}{目录}
\renewcommand{\listfigurename}{插图目录}
\renewcommand{\listtablename}{表格目录}
\renewcommand{\abstractname}{摘要}
\renewcommand{\appendixname}{附录}
\renewcommand{\indexname}{索引}
\renewcommand{\figurename}{图}
\renewcommand{\tablename}{表}
\author{王晓林}
\date{2015-06-20}
\title{《操作系统原理》考核大纲}
\hypersetup{
  pdfkeywords={},
  pdfsubject={},
  pdfcreator={Emacs 24.5.1 (Org mode 8.2.10)}}
\begin{document}

\maketitle
\tableofcontents
\clearpage

\begin{itemize}
\item 课程编号:A05025, A05026
\item 课程名称:操作系统原理
\item 任课教师:王晓林
\item 适应专业:电信、信计、计算机等本科专业
\item 授课学时:64
\item 考试方式:闭卷笔试
\item 命题规则:按《西南林学院考试命题规则》(试行)(1996年10月修订)执行
\item 考试时间:120分钟
\item 推荐教材:
\begin{enumerate}
\item A.S. Tanenbaum. Modern Operating Systems, 3e. Pearson Prentice Hall, 2008.
\item Silberschatz, Galvin, and Gagne. Operating System Concepts Essentials. John Wiley \& Sons, 2011.
\end{enumerate}
\item 课程性质及教学目的:本课程为计算机专业本科专业基础课。要求学生全面了解操作系统的工作原理。
\end{itemize}

各章节考核目标如下:
\section{操作系统简介}
\label{sec-1}
\subsection{考核知识点}
\label{sec-1-1}
\begin{itemize}
\item What's an OS?
\item OS services
\item Bootstrapping
\item Hardware
\item Interrupt
\item System calls
\end{itemize}
\subsection{考核要求}
\label{sec-1-2}
\begin{itemize}
\item 了解什么是操作系统
\item 了解操作系统提供哪些服务
\item 了解计算机启动的过程
\item 了解什么是中断
\item 了解什么是system call
\end{itemize}
\section{进程与线程}
\label{sec-2}
\subsection{考核知识点}
\label{sec-2-1}
\begin{itemize}
\item 进程的概念
\item 进程的产生
\item 进程的状态
\item 进程的PCB
\item 进程的切换
\item 进程与线程
\item 线程的概念
\item 线程的特点
\item 线程的操作
\item 线程分类(user level vs. kernel level)
\item Linux中的线程实现
\end{itemize}
\subsection{考核要求}
\label{sec-2-2}
\begin{itemize}
\item 了解什么是进程
\item 了解进程的创建和进程的状态变化
\item 了解什么是PCB
\item 了解CPU切换的过程
\item 了解什么是线程,线程的特点,及相关操作
\end{itemize}
\section{进程间同步}
\label{sec-3}
\subsection{考核知识点}
\label{sec-3-1}
\begin{itemize}
\item Concepts
\item Shared memory
\item Race condition and mutual exclusion
\item Semaphores
\item Monitors
\item Message passing
\item Classical IPC problems
\end{itemize}
\subsection{考核要求}
\label{sec-3-2}
\begin{itemize}
\item 了解进程间同步的概念
\item 了解共享内存的工作原理
\item 了解进程间冲突的处理方式
\item 了解什么是信号量,及其工作机制
\item 了解什么是消息传递,及其工作过程
\item 了解经典的IPC问题及解决方案
\end{itemize}
\section{CPU调度}
\label{sec-4}
\subsection{考核知识点}
\label{sec-4-1}
\begin{itemize}
\item Scheduling introduction
\item Scheduling algorithms
\item Thread scheduling
\item Linux scheduling
\end{itemize}
\subsection{考核要求}
\label{sec-4-2}
\begin{itemize}
\item 了解什么是调度
\item 了解各种调度算法
\item 了解线程的调度
\item 了解Linux系统的调度机制
\end{itemize}
\section{死锁}
\label{sec-5}
\subsection{考核知识点}
\label{sec-5-1}
\begin{itemize}
\item Resources
\item Introduction to deadlocks
\item Deadlock modeling
\item Dealing with deadlocks
\end{itemize}
\subsection{考核要求}
\label{sec-5-2}
\begin{itemize}
\item 了解什么是死锁
\item 了解死锁的处理机制
\end{itemize}
\section{内存管理}
\label{sec-6}
\subsection{考核知识点}
\label{sec-6-1}
\begin{itemize}
\item Real-mode vs. protected-mode memory management
\item Contiguous memory allocation
\item Virtual memory
\end{itemize}
\subsection{考核要求}
\label{sec-6-2}
\begin{itemize}
\item 了解什么是虚拟内存
\item 了解分页内存管理机制
\item 了解页替换算法
\end{itemize}
\section{文件系统}
\label{sec-7}
\subsection{考核知识点}
\label{sec-7-1}
\begin{itemize}
\item Files
\item Directories
\item File system implementation
\item Ext2 file system
\item Virtual file system
\end{itemize}
\subsection{考核要求}
\label{sec-7-2}
\begin{itemize}
\item 了解文件及目录的概念
\item 了解文件的实现方式
\item 了解ext2文件系统的工作原理
\item 了解什么是虚拟文件系统,及其工作原理
\end{itemize}
% Emacs 24.5.1 (Org mode 8.2.10)
\end{document}