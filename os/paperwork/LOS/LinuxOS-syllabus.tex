\documentclass{wx672ctexart}

\usepackage{wx672bib}
\addbibresource{../os.bib}

\author{王晓林}
\title{《Linux操作系统原理与实践》课程大纲}

\begin{document}

\maketitle
\tableofcontents
\clearpage

\begin{itemize}
\item 课程编号: 40000001
\item 学时: 64 (理论: 32; 实验: 32)
\item 学分: 4
\item 实习: 0
\item 面向专业: 计算机科学与技术,电子信息工程,信息与计算机技术
\end{itemize}

\section{课程大纲}

\subsection{课程内容}

\begin{enumerate}
\item Introduction
  \begin{itemize}
  \item What's an OS?
  \item OS services
  \item Bootstrapping
  \item Hardware
  \item Interrupt
  \item System calls
  \end{itemize}
\item Processes and Threads
  \begin{itemize}
  \item What's a process?
  \item Process creation
  \item Process state
  \item Process Control Block (PCB)
  \item CPU switch from process to process
  \item Processes vs. threads
  \item Why threads?
  \item Thread characteristics
  \item Thread operation and POSIX threads
  \item User-level threads vs. kernel-level threads
  \item Linux threads
  \end{itemize}
\item Process synchronization
  \begin{itemize}
  \item Concepts
  \item Shared memory
  \item Race condition and mutual exclusion
  \item Semaphores
  \item Monitors
  \item Message passing
  \item Classical IPC problems
  \end{itemize}
\item CPU Scheduling
  \begin{itemize}
  \item Scheduling introduction
  \item Scheduling algorithms
  \item Thread scheduling
  \item Linux scheduling
  \end{itemize}
\item Deadlocks
  \begin{itemize}
  \item Resources
  \item Introduction to deadlocks
  \item Deadlock modeling
  \item Dealing with deadlocks
  \end{itemize}
\item Memory Management
  \begin{itemize}
  \item Real-mode vs. protected-mode memory management
  \item Contiguous memory allocation
  \item Virtual memory
  \end{itemize}
\item File Systems
  \begin{itemize}
  \item Files
  \item Directories
  \item File system implementation
  \item Ext2 file system
  \item Virtual file system
  \end{itemize}
\end{enumerate}

\subsection{实验内容}
\label{sec-1-2}

参见第\ref{sec:lab}节《Linux操作系统原理与实践》实验教学大纲。

\subsection{实习}
\label{sec-1-3}

无

\subsection{考核}
\label{sec-1-4}

\begin{itemize}
\item 考试: 50\%
\item 作业: 50\%
\end{itemize}

\subsection{参考教材}
\label{sec-1-5}

\nocite{silberschatz11essentials,tanenbaum2008modern,bovet2005understanding}
\printbibliography[heading=none]{}

\section{课程说明}
\label{sec-2}

\subsection{课程性质和要求}
\label{sec-2-1}

《Linux操作系统原理与实践》是一门重要的专业基础课。深入理解操作系统的工作原理,了解Linux平
台的软件开发环境,对学生在软件编程、开发方面具有重大意义。 本课程介绍给同学如下内容:
\begin{itemize}
\item 操作系统简介
\item 进程与线程
\item 进程间通信
\item CPU调度
\item 死锁
\item 内存管理
\item 文件系统
\end{itemize}

\subsection{课程重点}
\label{sec-2-2}

\begin{itemize}
\item 进程
\item 内存管理
\item 文件系统
\end{itemize}

\subsection{作业、实习要求}
\label{sec-2-3}
作业迟交一天扣分10\%。

\subsection{与其它课程的关系}
\label{sec-2-4}

\begin{itemize}
\item 前期课程:计算机组成原理,Linux应用基础,C编程,汇编编程
\item 后期课程:Linux系统分析
\end{itemize}

\subsection{课时安排}
\label{sec-2-5}

\begin{center}
  \begin{tabular}{lrr}
    \hline
    课程内容 & 理论学时 & 实验学时\\
    \hline
    简介 & 4 & 4\\
    进程与线程 & 4 & 4\\
    进程间同步 & 6 & 6\\
    CPU调度 & 4 & 4\\
    死锁 & 4 & 4\\
    内存管理 & 6 & 6\\
    文件系统 & 4 & 4\\
    \hline
  \end{tabular}
\end{center}

\subsection{特殊说明}
\label{sec-2-6}

无

\section{实验教学大纲}
\label{sec:lab}

\begin{itemize}
\item 课程编号: 40000001
\item 学时: 64 (理论: 32; 实验: 32)
\item 学分: 4
\item 实习: 0
\item 授课对象: 计算机科学与技术,电子信息工程,信息与计算机技术
\end{itemize}

\subsection{实验教学的目的和要求}
\label{sec-3-1}

通过编程实践,让学生深入了解Linux操作系统的工作原理。

\subsection{实践教学大纲}
\label{sec-3-2}

\begin{center}
  \begin{tabular}{lr}
    \hline
    实验安排 & 学时\\
    \hline
    了解Linux内核 & 8\\
    进程管理 & 8\\
    内存管理 & 8\\
    文件系统 & 8\\
    \hline
  \end{tabular}
\end{center}

\subsection{实验设备要求}
\label{sec-3-3}

\begin{itemize}
\item Debian PC
\end{itemize}

\subsection{实验内容}
\label{sec-3-4}

\begin{itemize}
\item 参见\href{http://cs6.swfu.edu.cn/~wx672/lecture_notes/os/lab.html}{《Linux操作系统原
    理与实践》实验指导}。
\end{itemize}

\subsubsection{Approaching to the Linux kernel}
\label{sec-3-4-1}

\begin{enumerate}
\item proc file-system
\item Play with the kernel
\item Hello, kernel module!
\item System calls
\end{enumerate}

\subsubsection{Process Management}
\label{sec-3-4-2}

\begin{enumerate}
\item Process creation
\item Thread
\item IPC
  \begin{enumerate}
  \item Signals
  \item Pipe
  \item FIFO
  \item File Locking
  \item Message Queues
  \item Semaphores
  \end{enumerate}
\end{enumerate}

\subsubsection{Memory management}
\label{sec-3-4-3}

\begin{enumerate}
\item Basic commands
\item Shared Memory Segments
\item Memory Mapped Files
\end{enumerate}

\subsubsection{File System}
\label{sec-3-4-4}

\begin{enumerate}
\item File system creation
\item Finding a file with \texttt{hexdump}
\end{enumerate}

\subsection{实验报告要求}
\label{sec-3-5}

按规定格式完成,迟交报告每天扣分10\%。

\subsection{成绩考核}
\label{sec-3-6}

\begin{itemize}
\item 实验报告满分100,60分及格
\end{itemize}

\subsection{实验指导和参考书目}
\label{sec-3-7}

\begin{itemize}
\item 自编\href{http://cs6.swfu.edu.cn/~wx672/lecture_notes/os/lab.html}{《实验指导》}
\end{itemize}

\subsection{特别说明}
\label{sec-3-8}

无

\section{课程简介}
\label{sec-4}

\begin{itemize}
\item 课程编号: 40000001
\item 学时: 64 (理论: 32; 实验: 32)
\item 学分: 4
\item 实习: 0
\item 面向专业: 计算机科学与技术,电子信息工程,信息与计算机技术
\item 前期课程:英语,计算机组成原理,Linux应用基础,C编程,汇编知识
\item 课程性质和要求《Linux操作系统原理与实践》是一门重要的专业基础课。深入理解操作系统的工
  作原理,了解Linux平台上的软件开发环境对学生在软件编程、开发方面具有重大意义。 本课程介绍
  给同学如下内容:
  \begin{itemize}
  \item 操作系统简介
  \item 进程与线程
  \item 进程间通信
  \item CPU调度
  \item 死锁
  \item 内存管理
  \item 文件系统
  \end{itemize}
\item 参考教材\hfill
  \nocite{silberschatz11essentials,tanenbaum2008modern,bovet2005understanding}
  \printbibliography[heading=none]{}
\end{itemize}

\end{document}

%%% Local Variables:
%%% mode: latex
%%% TeX-master: t
%%% End:
