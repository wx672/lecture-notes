\documentclass{wx672article} % $HOME/texmf/tex/latex/wx672article.cls


\usepackage{wx672ctex}
\usepackage{wx672bib}
\addbibresource{os.bib}

\title{操作系统原理教学大纲}
\author{}\date{}

\pagestyle{empty}

\begin{document}

\begin{description}
\item[操作系统原理]
\item[每周课次: 4]
\item[学分:] 4
\item[前导课程:] C编程,Linux基础
\item[课程简介:] 本课程涵盖操作系统设计与实现方面的重要问题。操作系统是用户应用程
  序与计算机硬件之间的重要接口,它负责系统资源的分配与管理,为用户应用程序提供必要的服务,
  并要防止进程间发生冲突。本课程从操作系统五十多年的历史演变开始,然后介绍它的重要组成部分,
  包括进程管理,内存管理,文件系统,输入/输出等内容。
\item[主要内容:] 课程主要内容包括:
  \begin{itemize}
  \item[第1周:] 操作系统简介
  \item[第2周:] 硬件相关概念
  \item[第3周:] 进程的概念
  \item[第4周:] 进程的控制与管理
  \item[第5周:] 进程间通信
  \item[第6周:] 进程的调度
  \item[第7 -- 10周:] 进程间通信
  \item[第11周:] 死锁
  \item[第12 -- 14周:] 内存管理
  \item[第15、16周:] 文件系统
  \end{itemize}
\item[教材与参考书目:]\hfill
  \nocite{silberschatz11essentials,tanenbaum2008modern,bovet2005understanding}
  \printbibliography[heading=none]{}
\end{description}

\end{document}

%%% Local Variables:
%%% mode: latex
%%% TeX-master: t
%%% End:
